% !Mode:: "TeX:UTF-8" 

\BiAppendixChapter{致\quad 谢}{Acknowledgements}

三年时间匆匆过去,硕士生涯即将结束。然而三年时光是不可逆的,这份感谢也是永恒的。

首先感谢带领我学术研究和完成这篇论文的老师,我的硕士阶段指导老师,贾晨老师。在我的上一位导师离职的时候,他刚好入职,我也有幸成为了他的第一名研究生。由于我的研究方向发生变化,长时间都处在入门阶段,最后能够完成这篇论文,非常感谢贾老师两年来的悉心指导。

中物院留下过许多英雄的足迹,每当和别人谈到我是在这里读研究生,都觉得这是段会让我难忘的经历。在此感谢招收我进来的导师,胡广辉老师,还有计科中心,是你们给了我这段经历。特别是胡老师,多次让我外出参加会议和外地的暑期学校,即使离职后,还是很关心我,对我的学习和生活都提供了很多帮助。在中物院学习期间,有幸去清华,北大上课,也有幸多次听到各种前沿问题的学术报告。这些学习机会,让我受益良多,充分满足了我的学术好奇心,也增添了对自己未来道路的思考。

三年的时光,还要感谢和我一起走过来的同窗们,是你们让我在枯燥科研过程中不孤独,状态低沉时不迷茫,遇到失败时不灰心。此外,还要感谢我的父母和我过去的同学朋友,是你们给了我信心和肯定,让我面对各种挑战。最后一段时光,我是在公司实习中度过,感谢超参数公司,以及在这里遇到的同事,带我认识象牙塔外面的世界,开启新的征程。

回想这三年,确实经历了很多人和事,感谢有你们,让我认清生活,还依然热爱生活。
