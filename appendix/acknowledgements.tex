% !Mode:: "TeX:UTF-8" 

\BiAppendixChapter{致\quad 谢}{Acknowledgements}

% 三年的硕士生涯即将画上句号。在这里,我要向这期间曾经帮助过我的人,表达最诚挚的谢意。

% 首先,我要感谢我的导师张智民老师,还有在学术研究中指导过我的 李会元老师,许志钦老师,孙继广老师,蔡伟老师,贾晨老师(排名不分先后)。这些老师们无论是在学术方面还是生活方面,都给了我很多的指导和关心。而我也在这些老师身上学到的也不只是学术知识,更多的是严谨求实的态度和为人处世的道理。他们的言传身教使我终生受益。

% 同时,我也要感谢这三年来与我一起学习的各位同窗。在我面对困难灰心丧气的时候,他们总是给陪在我身边,给了我许许多多的鼓励。感谢他们陪我度过了一段永生难忘的岁月。

% 此外,我还要感谢我的家人和我的父母,他们是我在科研道路上的坚实后盾。感谢他们在硕士期间给了我充分的自由。我衷心希望自己能真正地成为家人眼中的骄傲。

% 最后,我要感谢我未来的女朋友。感谢她在我三年的硕士生涯中从来没有和我恋爱,使我可以专心地从事学术研究工作。

三年时间匆匆过去,硕士生涯即将结束。陪我走过这段路的人有很多,或许我会在下面提到,也或许我已经忘了他们的名字,这仅代表写这篇致谢时的状态。然而三年时光是不可逆的,这份感谢也是永恒的。

首先感谢带领我学术研究和完成这篇论文的老师,我的硕士阶段指导老师,贾晨老师。在我的上一位导师离职的时候,他刚好入职,我也有幸成为了他的第一名研究生。由于我的研究方向发生变化,长时间都处在入门阶段,最后能够完成这篇论文,非常感谢贾老师两年来的悉心指导。

中物院留下过许多英雄的足迹,每当和别人谈到我是在这里读研究生,都觉得这是段会让我难忘的经历。在此感谢招收我进来的导师,胡广辉老师,还有计科中心,是你们给了我这段经历。特别是胡老师,多次让我外出参加会议和外地的暑期学校,即使离职后,还是很关心我,对我的学习和生活都提供了很多帮助。在中物院学习期间,有幸去清华,北大上课,也有幸多次听到各种前沿问题的学术报告。这些学习机会,让我受益良多,充分满足了我的学术好奇心,也增添了对自己未来道路的思考。

三年的时光,还要感谢和我一起走过来的同窗们,是你们让我在枯燥科研过程中不孤独,状态低沉时不迷茫,遇到失败时不灰心。此外,还要感谢我的父母和我过去的同学朋友,是你们给了我信心和肯定,让我面对各种挑战。最后一段时光,我是在公司实习中度过,感谢超参数公司,以及在这里遇到的同事,带我认识象牙塔外面的世界,开启新的征程。

回想这三年,确实经历了很多人和事,感谢有你们,让我认清生活,还依然热爱生活。
