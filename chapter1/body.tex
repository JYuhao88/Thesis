% !Mode:: "TeX:UTF-8"

\BiChapter{绪论}{Introduction}

\BiSection{问题的背景和研究现状}{Background and research status}

本文中研究的椭圆方程具有如下的一般形式。在$d$维空间中的求解区域$\Omega \subset \mathbb{R}^d$上,椭圆方程表示为
\begin{equation}\label{prob1}
\left\{
\begin{split}
& -\nabla(a(x) \nabla u(x)) + c(x) \; u(x) = f(x) & \text{for} \; x \in \Omega \\
& \quad u(x) = u_0(x) & \text{for} \; x \in \partial \Omega
\end{split}
\right.
\end{equation}
其中$a(x) > a_0 > 0, c(x) \geq 0, f(x), u_0(x)$是已知函数,$u(x)$是方程的解。

椭圆特征值问题表示为
\begin{equation}\label{prob2}
\left\{
\begin{split}
& -\nabla(a(x) \nabla u(x)) + c(x) \; u(x) = \lambda \; u(x) & \text{for} \; x \in \Omega \\
& \quad u(x) = 0 & \text{for} \; x \in \partial \Omega
\end{split}
\right.
\end{equation}
其中,$u(x)$是待求解的特征函数,$\lambda$是对应的特征值。

椭圆方程是应用很广泛的一类方程,例如:描述溶液中蛋白质电势分布的Poisson-Boltzmann方程\cite{cai2013},描述量子效应的Schrodinger方程等,都可以看作是椭圆方程。目前已经有很多传统方法(如:有限差分法,有限元方法,有限体积方法,谱方法等)可以高效且高精度地求解椭圆方程。在本文中,我们主要关注用深度学习方法求解椭圆方程。

深度学习方法是目前研究的热点问题。在许多领域内(如:图像处理,语音识别等)深度学习方法已经被广泛应用,并且取得了超越传统方法的效果。在科学计算问题中,深度学习方法也已经取得了一定的进展。由于神经网络的逼近性质和维数的关系不大,深度学习方法能够一定程度上缓解维数灾难,在求解高维方程时有很好的效果。此外,由于深度学习方法不需要像传统的有限元或有限差分法那样使用结构化网格,因此它可以轻易处理复杂的求解区域。文献\cite{weinan2017deep, han2018solving, wang2020mesh}中讨论了深度学习方法在求解微分方程中的应用。文献\cite{han2018deep, strofer2019data}中讨论了如何通过深度学习方法模拟物理模型。文献\cite{he2018relu, hamilton2019dnn}中讨论了深度神经网络和传统有限元方法的联系,从数学角度解释了深度神经网络的逼近性质。

对于深度学习方法在求解椭圆方程上的具体应用,文献\cite{weinan2018deep, liao2019deep}中提出了Deep-Ritz方法,通过极小化能量泛函的方式求解方程。同时,文献\cite{raissi2019physics}中提出了PINN方法,通过极小化残量的方式求解方程。

然而,要将常用的深度神经网络应用于计算科学和工程问题,我们面临着一些挑战。其中最突出的问题是,深度学习方法通常只适合于处理低频数据。文献\cite{xu_training_2018, rahaman2018spectral, xu2019frequency}中提出了频率准则。频率准则指出,许多深度神经网络学习低频数据的速度很快,效果很好,但当处理高频数据时,它们就失去了这些优势。文献\cite{luo2019theory, zhang2019explicitizing, basri2019convergence, cao_towards_2020}在理论上严格地论证了频率准则。因此,对于解具有高震荡的问题,我们需要采取特殊的处理。

目前用深度学习方法求解偏微分方程的工作已经有许多,但从频域的角度探索深度学习方法的研究还不完善。文献\cite{cai2019phasednn}中提出了Phase-DNN。它通过在频域上平移的方式求解带有高震荡的方程,取得了很好的效果。但美中不足的是,Phase-DNN在求解过程中需要事先知道精确解的频率,因此无法求解频率未知的方程。

神经网络在不同频率下的表现不同,利用这种差异设计神经网络结构可以有利于提高网络的性能。在计算机视觉领域,已经有一系列的工作,如:图像恢复\cite{deng2018learning}、超分辨率\cite{pan2018learning}、分类问题\cite{wu2020multigrid}等,通过利用不同图像频率的学习差异,提高了学习性能(包括泛化性能和训练速度)。在多尺度变换方面,文献\cite{Fan2019Hmat}和文献\cite{Fan2019BCR}中分别基于快速多极算法和小波变换的思想设计了多尺度网络,使网络性能得到了很大提升。

\BiSection{本文主要研究内容}{Main contents}

本文主要研究内容可以分为以下三个方面:

首先,在实际问题中,我们通常希望模型和输入数据的尺度无关。例如在图像处理中,我们希望输入的灰度值无论是在$[0,255]$还是$[0,1]$之间,得到的结果应该类似。很多机器学习方法(如:简单的线性模型,SVM等)都有这种性质,但是并没有理论保证深度学习具有类似的性质。本文论述了尺度变换(在输入数据或者输出数据上)对网络的影响,得到了一些初步的结论。

其次,深度学习是基于优化算法的方法,它把方程求解的问题转化成一个优化问题。但是由于神经网络本身的复杂性,这样的优化问题通常是非凸且病态的。想要高效求解这样的优化问题并不容易,一般来说,我们只能找到问题中靠近初始点的局部极小值。因此,在这样的问题中,初值的选取十分关键,好的初值可以提高网络的精度(使训练过程收敛到误差更小的局部最优值),也可以加速训练过程(使初始值离局部最优值更近)。本文基于尺度变换,提出了一种有效的初值选取方式,同时基于这种初值选取方式提出了多尺度网络结构。

最后,本文实验了多尺度网络在求解椭圆方程和分子计算模拟问题中的效果,同时指出了深度学习方法在求解椭圆特征值问题中的潜力。经过各方面的实验,我们可以验证,相比于传统的全连接网络,多尺度网络求解椭圆方程的速度更快,误差也更小。这些实验清楚地表明,多尺度网络是一种高效且易于实现的无网格椭圆方程求解器。
