\BiChapter{绪论}{Introduction}

\BiSection{问题的背景和研究现状}{Background and research status}

随机过程,是从物理问题中抽象出的一种数学模型,并逐渐应用于化学,生物,金融,人工智能领域,其中的数学理论为很多学科建立理论基础。随着各学科发展,背后的数学结构更加多样化,同时也带给了数学领域新挑战,新问题。过去二十年,随机热力学取得显著进展,并且逐渐成为了非平衡态统计物理的重要分支 \cite{annurev-conmatphys,Seifert_2012,VANDENBROECK20156} 。随机热力学系统通常被建模为马尔科夫过程,如郎之万方程,主方程和福克-普朗克方程等。马尔科夫链的状态空间是离散的,并且因为任何马尔科夫过程总是可以被马尔科夫链所近似,所以它也是最基本和最重要的动态模型。依据此观点,平衡状态可以由可逆的马氏过程定义,进一步用熵的产生定量刻画偏离平衡的程度,并写成热力学通量和力的双线性函数 \cite{PhysRev.91.1505}。Kolmogorov \cite{Math.Ann.112} 很早就认识到,马尔科夫系统的可逆性可以通过其环动态性来描述:当且仅当沿每个环的转移概率的过渡概率的乘积完全相同于沿其反向的环时,该系统是可逆的,这概括了细致平衡化学反应网络的 Wegscheider 条件。一个更为深刻观点是:熵的产生可以沿着环分解,热力学通量可以表示为环流(也叫环通量),热动力可以表示为环关系 \cite{Schnakenberg1976NetworkTO}。

马氏链的环表示法在物理学、化学和生物学中应用广泛 \cite{ZHANG20121,GE201287}。事实上,环流有几种不同的定义方法,其中生成树和环消除是两个常见的方法。Hill \cite{Hill1966StudiesIIa,Hill1966StudiesIIb,hill2013free} 和 Schnakenberg \cite{Schnakenberg1976NetworkTO} 研究了一套网络理论,其中定义了环流的基环族,并把生成树与马氏系统的有向转移图相关联。图中每条不属于生成树的边,被称为弦,由此产生一个基环。一个基环的环流被定义为单位时间内相应的弦形成的次数。钱敏平\cite{minping1982circulation,jian1984circulations,jiang2004mathematical}和 Kalpazidou \cite{kalpazidou2007cycle} 则是研究了环表示理论,并定义了图中所有简单环的环流,即除了开始和结束的顶点,没有重复顶点的环。其中马氏系统的轨道被持续追踪,一旦一个环形成,该环就会从轨道中抹去,接着追踪剩余的轨道,等待下一个环形成。由此,一个简单环的环流就被定义为单位时间内形成的次数。近年来,也有学者基于序列匹配的思想 \cite{roldan2019exact,biddle2020reversal,pietzonka2021cycle} 提出了其他类型的环流,即图中所有回路的环流,也就是第一和最后一个顶点相等的有向路径的环流。

环流也可以沿着单一的随机轨道定义。大量热力学量的发现是随机热力学领域的重大突破,如熵的产生,绝热和非绝热,以及环流所满足各种类型的涨落定理 \cite{annurev-conmatphys,Seifert_2012,VANDENBROECK20156},这些定理以等式式的方式给出了热力学第二定律的非平凡概括。对于生成树方式定义的环流,Andrieux和Gaspard \cite{andrieux2007fluctuation}证明了涨落定理在某些限制条件下对于净环流成立。此外,Polettini 和 Esposito \cite{polettini2014transient}表明,如果稍作修正环流的定义,任何有限时间内,暂态涨落定理都是成立的。对于环擦除方式定义的环流,Andrieux 和 Gaspard \cite{andrieux2007network}以及贾晨等人 \cite{jia2016cycle} 证明了环流和净环流都满足各有类型的涨落定理和对称关系。对于序列匹配方式定义的环流,相应的涨落定理和对称关系最近也被提出了 \cite{pietzonka2021cycle}。针对某些具有连续状态空间的随机过程,一些学者也在研究相应的环流涨落定理,例如圆周上的 Langevin动力学\cite{ge2017cycle}。

数学层面还会有一个问题,沿单一随机轨道定义的各种热力学量是否满足大偏差原理\cite{varadhan1984large,den2000large}。大偏差关注的是小概率的随机过程的长期涨落行为,它与无穷时间极限下的涨落定理密切相关。对于马氏系统,经验测度的大偏差(单位时间内通过图中顶点的次数的大偏差)和经验流的大偏差(单位时间内通过图中边的次数的大偏差)已被广泛研究,然而经验环流的大偏差(单位时间内通过图中环的次数的大偏差)很少受到关注。关于生成树方式定义的环流,相关的大偏差理论已经被建立,这是由于经验环流恰好是弦的经验流 \cite{bertini2015flows,bertini2015large}。而对于循环消除方式定义的环流,大偏差率函数的表达方式目前还没有相关的研究成果,即使对简单拓扑结构的系统也是如此。

针对以生成树和循环消除方式定义的环流,本文进行了全面的比较研究,并阐明了它们之间的联系和区别。本文的结构安排如下。第2节中,回顾了这两类环流的定义,并对它们进行了简单的比较。第3节中,研究了这两类环流的大偏差,并用环插入法得到了单环马氏系统的环擦除环流的联合分布和速率函数,同时也得到了一般马氏系统的生成树环流的速率函数表达式。第4节中,陈述并比较了这两类环流所满足的涨落定理及对称关系。进一步阐明了这些涨落定理的应用范围,并表明所有针对生成树环流的结果都可以从环擦除环流的结果中导出。第5节中,得出最后结论。