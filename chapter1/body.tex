\BiChapter{绪论}{Introduction}

\BiSection{问题的背景和研究现状}{Background and research status}

过去二十年,随机热力学取得显著进展,并且这个领域逐渐成为了非平衡态统计物理的重要分支 \cite{Fan2019Hmat,annurev-conmatphys,Seifert_2012} 。该领域中,热力学系统通常被建模为马氏过程。马尔科夫链的状态空间是离散的,并且它是最基本和最重要的动态模型,因为任何马尔科夫过程总是可以被马尔科夫链所近似。依据这一思路,平衡状态被定义为一个可逆的马氏过程,对平衡的偏差通常由熵的产生这一概念定量刻画,它可以写成热力学通量和力的双线性函数 \cite{VANDENBROECK20156}。Kolmogorov \cite{PhysRev.91.1505} 早就注意到,马尔科夫系统的可逆性可以通过其环动力学来描述:当且仅当沿每个环的转移概率的过渡概率的乘积完全相同于沿其反向的环时,该系统是可逆的,这概括了细致平衡化学反应网络的 Wegscheider 条件。可以深刻认识到,熵的产生可以沿着环分解,热力学通量可以表示为环流(也叫环通量),热动力可以表示为环关系 \cite{Math.Ann.112}。

//
马氏链的环表示法在物理学、化学和生物学中得到了广泛的应用 \cite{Schnakenberg1976NetworkTO,ZHANG20121}。事实上,环流可以用几种不同的方式来定义。生成树和环消除是两个常见的定义方法。Hill \cite{GE201287,Hill1966StudiesIIa,Hill1966StudiesIIb} 和 Schnakenberg \cite{Math.Ann.112} 发展了一套网络理论,并定义了环流的基环族。该理论把生成树与马氏系统的有向转移图相关联。图中每条不属于生成树的边,被称为弦,并由此产生一个基环。一个基环的环流被定义为单位时间内相应的弦形成的次数。Qians \cite{minping1982circulation,jian1984circulations,jiang2004mathematical}和 Kalpazidou \cite{kalpazidou2007cycle} 进一步发展了环表示理论,并定义了图中所有简单环的环流,即除了开始和结束的顶点,没有重复顶点的环。该理论中,马氏系统的轨迹被追踪。一旦一个环形成,该环就会从轨迹中抹去,接着追踪剩余的轨迹,直到下一个环形成。那么,一个简单环的环流被定义为单位时间内形成的次数。近年来,基于序列匹配的思想 \cite{roldan2019exact,biddle2020reversal,pietzonka2021cycle} 提出了另一种类型的环流,即图中所有环的环流,即第一和最后一个顶点相等的有向路径的环流。

所有类型的环流也可以沿着单一的随机轨迹来定义。随机热力学的重大突破之一是发现一大类热力学量,如熵的产生及其绝热和非绝热部分,以及环流所满足各种类型的涨落定理 \cite{Fan2019Hmat,annurev-conmatphys,Seifert_2012},这些定理以等式,而不是不等式的方式给出了热力学第二定律的非平凡概括。对于以生成树方式定义的环流,Andrieux和Gaspard \cite{andrieux2007fluctuation}证明了涨落定理在限制下对于净环流成立。此外,Polettini 和 Esposito \cite{polettini2014transient}表明,如果对缓落定义稍作修正,在任何有限时间内,瞬时涨落定理都是成立的。对于环擦除方式定义的环流,Andrieux 和 Gaspard \cite{{andrieux2007network}以及Jia等人 \cite{jia2016cycle} 证明所有类型的涨落定理和对称关系,对于绝对的和净环流都满足。对于以序列匹配方式定义的环流,相应的涨落定理和对称关系最近也被提出了 \cite{pietzonka2021cycle}。对于一些具有连续状态空间的随机过程,环流的涨落定理也得到了发展,例如圆周上的 Langevin动力学\cite{ge2017cycle}。

从数学角度看,另一个重要的问题是沿单一随机轨迹定义的各种热力学量是否满足大偏差原理\cite{varadhan1984large,den2000large}。大偏差关注的是小概率的随机过程的长期涨落行为,在无穷时间极限下,它与涨落定理密切相关。对于马氏系统,经验测度的大偏差(即图中一个顶点在单位时间内通过的次数的大偏差)已被广泛研究,而经验环流的大偏差(即图的一个顶点在单位时间内被通过的次数)以及经验流量(即图的一条边在单位时间内被通过的次数)则相对来说很少受到关注。对于以生成树方式定义的环流,大偏差理论已经被建立。因为在这种情况下,经验环流恰好是弦的经验流 \cite{bertini2015flows,bertini2015large}。对于以循环消除方式定义的环流,其大偏差率函数的明确表达方式仍然是未知的,即使对具有简单拓扑结构的系统也是如此。

在本文中,我们对以生成树和循环消除方式定义的环流进行了全面的比较研究,并阐明了它们之间的联系和区别。本文的结构安排如下。第2节中,回顾了这两类环流的定义,并对它们进行了简单的比较。第3节中,研究了这两类环流的大偏差,并用环插入法得到了单环马氏系统的环擦除环流的精确联合分布和速率函数,同时也得到了一般马氏系统的生成树环流的精确速率函数。第4节中,陈述并比较了这两类环流所满足的各种类型的涨落定理及对称关系。进一步阐明了这些涨落定理的应用范围,并表明所有针对生成树环流的结果都可以从环擦除环流的结果中到处。我们在第5节得出最后结论。