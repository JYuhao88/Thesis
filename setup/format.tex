% !Mode:: "TeX:UTF-8"

\theoremstyle{plain}
\theorembodyfont{\song\rmfamily}
\theoremheaderfont{\hei\rmfamily}
\newtheorem{definition}{\hei 定义}[chapter]
\newtheorem{example}{\hei 例}[chapter]
\newtheorem{algo}{\hei 算法}[chapter]
\newtheorem{theorem}{\hei 定理}[chapter]
\newtheorem{axiom}{\hei 公理}[chapter]
\newtheorem{proposition}{\hei 命题}[chapter]
\newtheorem{lemma}{\hei 引理}[chapter]
\newtheorem{corollary}{\hei 推论}[chapter]
\newtheorem{remark}{\hei 注解}[chapter]
\newenvironment{proof}{\noindent{\hei 证明:}}{\hfill $ \square $ \vskip 4mm}
\theoremsymbol{$\square$}
\setlength{\theorempreskipamount}{0pt}
\setlength{\theorempostskipamount}{-2pt}

\allowdisplaybreaks[4]

%\CJKcaption{gb_452}
%\CJKtilde
\setlength{\parindent}{2em}

\arraycolsep=1.6pt

% s ########################################################
\titleformat{\chapter}{\center\bfseries\xiaosan\song}{\chaptertitlename}{0.5em}{}
\titlespacing{\chapter}{0pt}{-5.5mm}{8mm}   % 小三,黑宋
\titleformat{\section}{\bfseries\sihao\song}{\thesection}{0.5em}{}
\titlespacing{\section}{0pt}{4.5mm}{4.5mm}   % 四号,黑宋
\titleformat{\subsection}{\bfseries\xiaosi\song}{\thesubsection}{0.5em}{}
\titlespacing{\subsection}{0pt}{4mm}{4mm}   % 小四,黑宋
\titleformat{\subsubsection}{\xiaosi\kai}{\thesubsubsection}{0.5em}{}
\titlespacing{\subsubsection}{0pt}{0pt}{0pt}  % 小四,楷体

% 缩小目录中各级标题之间的缩进,使它们相隔一个字符距离,也就是12pt

%% 以下内容设置英文目录格式 %%%%%%%%%%%%%%%%%%%%%%%%%%%%%%%%%%%
\makeatletter
\renewcommand*\l@chapter{\@dottedtocline{0}{0em}{4em}}% 控制英文目录: 细点\@dottedtocline  粗点\@dottedtoclinebold
\renewcommand*\l@section{\@dottedtocline{1}{0em}{1.8em}} % 目录无缩进
\renewcommand*\l@subsection{\@dottedtocline{2}{0em}{2.5em}} % 目录无缩进

\def\@dotsep{0.75}           % 定义英文目录的点间距
\setlength\leftmargini {0pt}
\setlength\leftmarginii {0pt}
\setlength\leftmarginiii {0pt}
\setlength\leftmarginiv {0pt}
\setlength\leftmarginv {0pt}
\setlength\leftmarginvi {0pt}

\def\chicontentsname{\bfseries\xiaosan 目~\quad~录}   % 中文目录名
\newcommand\tableofchicontents{
   \pdfbookmark[0]{目~~~~录}{mulu}
     \@restonecolfalse
   \chapter*{\chicontentsname  %chapter*上移一行,避免在toc中出现。
       \@mkboth{%
          \chicontentsname}{\chicontentsname}}
   \@starttoc{toc}%
   \if@restonecol\twocolumn\fi
   }

\def\engcontentsname{\bfseries\xiaosan Contents}   % 英文目录名
\newcommand\tableofengcontents{
   \pdfbookmark[0]{Contents}{econtent}
     \@restonecolfalse
   \chapter*{\engcontentsname  %chapter*上移一行,避免在toc中出现。
       \@mkboth{%
          \engcontentsname}{\engcontentsname}}
   \@starttoc{toe}%
   \if@restonecol\twocolumn\fi
   }

\urlstyle{same}  % 论文中引用的网址的字体默认与正文中字体不一致,这里修正为一致的。
\renewcommand\endtable{\vspace{-4mm}\end@float}

% 定义页眉和页脚,根据布尔变量画正文和首页不同的页眉
\newboolean{first}                             % 定义布尔变量
\setboolean{first}{true}                       % 附初值为真
\pagestyle{fancy}                              % 设定正文页眉

\fancypagestyle{plain}{                        % 设定首页页眉
\setboolean{first}{false}                      % 布尔变量为假
\fancyhf{}
\fancyhead[CO]{}                               % 无页眉
\fancyfoot[OR,EL]{\xiaowu \thepage}}
\newcommand{\makefirstpageheadrule}{           % 首页页眉线
\makebox[0pt][l]{\rule[0.55\baselineskip]{\headwidth}{0.0pt}}  %文武线 1.0pt
\rule[0.7\baselineskip]{\headwidth}{0.0pt}}                    %文武线 0.5pt

\newcommand{\makeheadrule}{                        % 正文页眉线
\fancyhead[CE]{\song \xiaowu \leftmark}            % 正文页眉
\fancyhead[CO]{\song \xiaowu \chinesethesistitle}  % 正文页眉
\fancyfoot[OR,EL]{\xiaowu \thepage}
\rule[0.7\baselineskip]{\headwidth}{0.5pt}}

\renewcommand{\headrule}{
\ifthenelse{\boolean{first}}{\makeheadrule}
{\makefirstpageheadrule}}

%去掉章节标题中的数字
%%不要注销这一行,否则页眉会变成:“第1章1  绪论”样式
\renewcommand{\chaptermark}[1]{\markboth{\chaptertitlename~\ #1}{}}
\fancyhf{}

\renewcommand\frontmatter{\cleardoublepage
  \@mainmatterfalse
  \pagenumbering{Roman}}

% 调整罗列环境的布局
\setitemize{leftmargin=3em,itemsep=0em,partopsep=0em,parsep=0em,topsep=-0em}
\setenumerate{leftmargin=3em,itemsep=0em,partopsep=0em,parsep=0em,topsep=0em}

\newcommand{\citeup}[1]{\textsuperscript{\cite{#1}}}

% 定制浮动图形和表格标题样式
\captionnamefont{\bfseries\wuhao}
\captiontitlefont{\bfseries\wuhao}
\captiondelim{~~}
\captionstyle{\centering}
\renewcommand{\subcapsize}{\wuhao}
\setlength{\abovecaptionskip}{0pt}
\setlength{\belowcaptionskip}{0pt}

% 自定义项目列表标签及格式 \begin{publist} 列表项 \end{publist}
\newcounter{pubctr} %自定义新计数器
\newenvironment{publist}{%%%%% 定义新环境
\begin{list}{[\arabic{pubctr}]} %% 标签格式
    {
     \usecounter{pubctr}
     \setlength{\leftmargin}{2.5em}     % 左边界 \leftmargin =\itemindent + \labelwidth + \labelsep
     \setlength{\itemindent}{0em}       % 标号缩进量
     \setlength{\labelsep}{1em}         % 标号和列表项之间的距离,默认0.5em
     \setlength{\rightmargin}{0em}      % 右边界
     \setlength{\topsep}{0ex}           % 列表到上下文的垂直距离
     \setlength{\parsep}{0ex}           % 段落间距
     \setlength{\itemsep}{0ex}          % 标签间距
     \setlength{\listparindent}{0pt}    % 段落缩进量
    }}
{\end{list}}%%%%%

% 默认字体
\renewcommand\normalsize{
  \@setfontsize\normalsize{12pt}{12pt}
  \setlength\abovedisplayskip{4pt}
  \setlength\abovedisplayshortskip{4pt}
  \setlength\belowdisplayskip{\abovedisplayskip}
  \setlength\belowdisplayshortskip{\abovedisplayshortskip}
  \let\@listi\@listI}
% 默认五号字体
\newcommand\normalsmallsize{
  \@setfontsize\normalsize{10.5pt}{10.5pt}
  \setlength\abovedisplayskip{4pt}
  \setlength\abovedisplayshortskip{4pt}
  \setlength\belowdisplayskip{\abovedisplayskip}
  \setlength\belowdisplayshortskip{\abovedisplayshortskip}
  \let\@listi\@listI}

% 设置行距和段落间垂直距离,根据规定,20磅
\def\defaultfont{\renewcommand{\baselinestretch}{1.64}\normalsize\selectfont}
\def\defaultfontsmall{\renewcommand{\baselinestretch}{1.64}\normalsmallsize\selectfont}

\renewcommand{\CJKglue}{\hskip 0.56pt plus 0.08\baselineskip}  %加大字间距,使每行36个字。
\predisplaypenalty=0  %公式之前可以换页,公式出现在页面顶部

% 封面、摘要、版权、致谢格式定义
\def\ctitle#1{\def\@ctitle{#1}}\def\@ctitle{}
\def\cdegree#1{\def\@cdegree{#1}}\def\@cdegree{}
\def\csubject#1{\def\@csubject{#1}}\def\@csubject{}
\def\cauthor#1{\def\@cauthor{#1}}\def\@cauthor{}
\def\csupervisor#1{\def\@csupervisor{#1}}\def\@csupervisor{}
\def\cassosupervisor#1{\def\@cassosupervisor{副导师: #1\\}}\def\@cassosupervisor{}
\def\ccosupervisor#1{\def\@ccosupervisor{联合导师: #1}}\def\@ccosupervisor{}
\def\ccommitteechairman#1{\def\@ccommitteechairman{#1}}\def\@ccommitteechairman{}
\def\creviewers#1{\def\@creviewers{#1}}\def\@creviewers{}
\def\csdate#1{\def\@csdate{#1}}\def\@csdate{}
\def\cdate#1{\def\@cdate{#1}}\def\@cdate{}
\long\def\cabstract#1{\long\def\@cabstract{#1}}\long\def\@cabstract{}
\def\ckeywords#1{\def\@ckeywords{#1}}\def\@ckeywords{}

\def\etitle#1{\def\@etitle{#1}}\def\@etitle{}
\def\edegree#1{\def\@edegree{#1}}\def\@edegree{}
\def\esubject#1{\def\@esubject{#1}}\def\@esubject{}
\def\eauthor#1{\def\@eauthor{#1}}\def\@eauthor{}
\def\esupervisor#1{\def\@esupervisor{#1}}\def\@esupervisor{}
\def\eassosupervisor#1{\def\@eassosupervisor{Associate Supervisor: #1\\}}\def\@eassosupervisor{}
\def\ecosupervisor#1{\def\@ecosupervisor{Co Supervisor: #1\\}}\def\@ecosupervisor{}
\def\ecommitteechairman#1{\def\@ecommitteechairman{#1}}\def\@ecommitteechairman{}
\def\ereviewers#1{\def\@ereviewers{#1}}\def\@ereviewers{}
\def\esdate#1{\def\@esdate{#1}}\def\@esdate{}
\def\edate#1{\def\@edate{#1}}\def\@edate{}
\long\def\eabstract#1{\long\def\@eabstract{#1}}\long\def\@eabstract{}
\long\def\NotationList#1{\long\def\@NotationList{#1}}\long\def\@NotationList{}
\def\ekeywords#1{\def\@ekeywords{#1}}\def\@ekeywords{}
\def\classifiedindex#1{\def\@classifiedindex{#1}}\def\@classifiedindex{}
\def\unidecimalclass#1{\def\@unidecimalclass{#1}}\def\@unidecimalclass{}
\def\statesecrets#1{\def\@statesecrets{#1}}\def\@statesecrets{}
\def\thesisnum#1{\def\@thesisnum{#1}}\def\@thesisnum{}

% 定义封面
\def\makecover{
    \begin{titlepage}
    % 封面一
\begin{center}

	{\xiaosi
	\begin{tabular}{@{}r@{:}l@{}}
	   分类号 & \uline{\makebox[51mm]{\@classifiedindex}}\\
 	   U.D.C.  & \uline{\makebox[51mm]{\@unidecimalclass}}
	\end{tabular}}\hfill
	{\xiaosi
	\begin{tabular}{@{}r@{:}l@{}}
	   密级 &  \uline{\makebox[51mm]{\@statesecrets}}\\
 	   编号 &  \uline{\makebox[51mm]{\@thesisnum}}
	\end{tabular}}
    \parbox[t][36pt][t]{\textwidth}{\begin{center} \end{center} }

    \parbox[t][44pt][t]{\textwidth}{\chuhao
    \begin{center} {\song \bfseries 中国工程物理研究院} \end{center} }
    \parbox[t][44pt][t]{\textwidth}{\begin{center} \end{center} }
    \begin{center} {\song \yihao \bfseries 学 位 论 文 }\end{center}

    \parbox[t][70pt][t]{\textwidth}{ \bfseries
    \begin{center} \renewcommand{\arraystretch}{2.0}
    \begin{tabular}{c}
        \kai \xiaoer \uline{\makebox[93mm]{\@ctitle}}\\
        \kai \xiaoer \uline{\makebox[93mm]{}}\\
        \song\sanhao \uline{\makebox[93mm]{\@cauthor}}
    \end{tabular} \renewcommand{\arraystretch}{1}
    \end{center} }
    \parbox[t][40pt][t]{\textwidth}{\begin{center} \end{center} }

	\parbox[t][150pt][t]{\textwidth}{\bfseries
    \begin{center} \renewcommand{\arraystretch}{2.0} \song \xiaosi
    \begin{tabular}{r}
    {\kai \sihao 指导教师姓名}  \uline{\makebox[85mm]{\@csupervisor}}\\
                                \uline{\makebox[85mm]{\@cassosupervisor \@ccosupervisor}}\\
    {\kai \sihao 申请学位级别} \uline{\makebox[26mm]{\@cdegree}}  专业名称 \uline{\makebox[41mm]{\@csubject}}  \\
    {\kai \sihao 论文提交日期} \uline{\makebox[26mm]{\number\year 年\number\month 月}}  论文答辩日期 \uline{\makebox[32mm]{\@cdate}}  \\
    {\kai \sihao 授予学位单位和日期} \uline{\makebox[70mm]{中国工程物理研究}}\\
    {\kai \sihao 答辩委员会主席 \uline{\makebox[45mm]{\@ccommitteechairman}}}\\
    {\kai \sihao 评阅人 \uline{\parbox[b]{45mm}{\@creviewers}}}
    \end{tabular} \renewcommand{\arraystretch}{1}
    \end{center} }
    \parbox[t][130pt][t]{\textwidth}{\begin{center} \end{center} }

    {\xiaosi \bfseries \number\year 年\number\month 月\number\day 日}
\end{center}

%%%%%%增加一空白页
    \newpage
    ~~~\vspace{1em}
    \thispagestyle{empty}

    % 英文封面
    \newpage
    \thispagestyle{empty}

        % 封面一
\begin{center}

	{\xiaosi
	\begin{tabular}{@{}r@{:}l@{}}
	   Classif\/ied Index & \uline{\makebox[51mm]{\@classifiedindex}}\\
 	   U.D.C.  & \uline{\makebox[51mm]{\@unidecimalclass}}
	\end{tabular}}\hfill
	{\xiaosi
	\begin{tabular}{@{}r@{:}l@{}}
	   Secret State &  \uline{\makebox[51mm]{Public}}\\   % 英文密级
 	   Number &  \uline{\makebox[51mm]{\@thesisnum}}
	\end{tabular}}
    \parbox[t][24pt][t]{\textwidth}{\begin{center} \end{center} }

    \parbox[t][74pt][t]{\textwidth}{\chuhao
    \begin{center} {\bfseries China Academy of Engineering Physics} \end{center} }
    \parbox[t][44pt][t]{\textwidth}{\begin{center} \end{center} }
    \begin{center} {\yihao \bfseries Dissertation for the {\exueweier} Degree in Engineering }\end{center}

    \parbox[t][70pt][t]{\textwidth}{ \bfseries
    \begin{center} \renewcommand{\arraystretch}{2.0}
    \begin{tabular}{c}
        \sanhao \uline{\makebox[93mm]{\@etitle}}\\
        \sanhao \uline{\makebox[93mm]{}}\\
        \sanhao \uline{\makebox[93mm]{\@eauthor}}
    \end{tabular} \renewcommand{\arraystretch}{1}
    \end{center} }
    \parbox[t][40pt][t]{\textwidth}{\begin{center} \end{center} }

	\parbox[t][150pt][t]{\textwidth}{\bfseries
    \begin{center} \renewcommand{\arraystretch}{2.0} \xiaosi
    \begin{tabular}{r}
    {\sihao Supervisor:}  \uline{\makebox[85mm]{\@esupervisor}}\\
                                \uline{\makebox[85mm]{\@eassosupervisor \@ecosupervisor}}\\
    {\sihao Academic Degree Applied for:} \uline{\makebox[72mm]{\@edegree}}\\  Specialty: \uline{\makebox[118mm]{\@esubject}}  \\
    {\sihao Date of Submitting:} \uline{\makebox[30mm]{\@esdate}}  Date of Defence: \uline{\makebox[32mm]{\@edate}}  \\
    {\sihao Degree-Conferring-Institution:} \uline{\makebox[70mm]{China Academy of Engineering Physics}}\\
    {\sihao Chairman of defence committee \uline{\makebox[45mm]{\@ecommitteechairman}}}\\
    {\sihao Paper Reviewers \uline{\parbox[b]{45mm}{\@ereviewers}}}
    \end{tabular} \renewcommand{\arraystretch}{1}
    \end{center} }
    \parbox[t][120pt][t]{\textwidth}{\begin{center} \end{center} }

    {\xiaosi \bfseries {\number\day}th \number\month, \number\year}
\end{center}
    \end{titlepage}

%%%%%%增加一空白页
    \newpage
    ~~~\vspace{1em}
    \thispagestyle{empty}
\clearpage
}

\makeatother
% e ########################################################
