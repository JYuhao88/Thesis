\BiChapter{总结和讨论}{}
本文考察了LE和ST方式定义的环流,并从大偏差和涨落定理角度,对两种环流做比较性探讨。LE环流定义适用于环空间中的所有类型的环,而ST环流只能在由生成树的弦产生的基本集中定义。基本集的规模远小于一般马氏系统的环空间。然而,对于一个环拓扑结构的系统,最多只有一个环会出现在环空间,但不会在基本集中。相比于ST环流,LE环流对环动态性给出了更完整和详细的描述。在周期性边界条件下,任意环的ST环流都能通过LE环流的加权和来表示。

此外,本文建立了相应的大偏差原理,并给出了单环马氏链中LE经验环流的速率大偏差函数的明确表达。该证明是基于封闭形式下LE经验环流的联合分布,并在计算联合分布过程里,提出了环插入方法的三个步骤。(i) 第一步是将所有经过初始状态的环插入轨迹中,(ii) 第二步是将所有不包含初始状态的二元环插入轨迹中,(iii) 第三步是将所有不包含初始状态的一元环插入轨迹中。此外,还证明了速率函数与马氏系统的初始分布无关。对于一般的单环系统,速率函数的解析表达式是很复杂的。然而,对于三状态系统和相邻状态之间的转移被禁止的单环系统,该公式却可以简化。仿照 \cite{bertini2015flows} 中提出的ST经验净环流,本文也给出了一般系统的经验ST环流的速率函数的精确表达式,并阐明了经验性LE和ST环流的速率函数之间的关系。

最后,还研究了LE(ST)经验环流是否满足的各种类型的涨落定理,并阐述了它们的适用范围。其中表明了LE经验(净)环流满足所有类型的涨落定理和对称关系。特别是,本文还引入了相似环的概念,得到了暂态涨落定理的强形式。(i)当某一对相似环被交换时,经验LE环流的联合分布满足对称关系;(ii)当某个环的净环流的值取其相反数时,经验性LE净环流的联合分布满足对称关系。由于在周期性边界下,ST经验环流可以用的经验LE环流的加权和来表示,进一步表明,经验ST环流不满足任何形式的涨落定理。然而只有在周期性边界条件下,ST经验净环流满足弱形式的暂态涨落定理:当基本集合中所有环的净环流取其相反数时,经验净ST净环流的联合分布满足对称关系。

在本文中,一些结果只针对单环马氏链得出。希望这些结果可以被推广到更普遍的马氏系统,甚至是半马氏或非马氏系统。此外,本文只对LE和ST的环流进行了比较。这两类环流与以序列匹配方式定义的环流 \cite{roldan2019exact,biddle2020reversal,pietzonka2021cycle} 之间的关系还不清楚,这些工作尚处于调研中。