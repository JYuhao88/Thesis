\BiChapter{总结和讨论}{}
本文对以 LE 和 ST 方式定义的马氏链的经验环流的大偏差和涨落定理作了比较研究,并对以LE和ST方式定义的马氏链的经验环流进行比较研究。LE环流可以针对环空间中的所有环定义,而ST环流只能定义在由生成树的和弦产生的基本集中。基本集可能比远小于一般马氏系统的环空间。然而,对于一个周期拓扑结构的系统,最多只有一个环包含在环空间中,但在基本集中却没有。与ST环流相比,LE环流对环动态性提供了更完整和详细的描述。在周期性边界条件下,任何环的ST环流都可以用LE环流的加权和来表示。

此外,本文建立了相应的大偏差原理,并给出了单环马氏链中经验LE环流的相关速率函数的明确表达。该证明是基于,封闭形式下所有环的经验LE环流的联合分布。在计算联合分布时,提出了三步环插入的方法。(i) 第一步是将所有经过初始状态的环插入轨迹中,(ii) 第二步是将所有不包含初始状态的两状态环插入轨迹中,(iii) 第三步是将所有不包含初始状态的单状态环插入轨迹中。此外,还证明了速率函数与马氏系统的初始分布无关。速率函数的解析表达式对于一般的单环系统来说是复杂的。然而,对于三态系统和某种相邻状态之间的转移被禁止的单环系统,该公式却可以简化。仿照 \cite{bertini2015flows} 中提出的经验净ST环流的方法,本文也给出了一般系统的经验(绝对)ST环流的速率函数的精确表达式,并阐明了经验性LE和ST环流的速率函数之间的关系。

最后,还研究了由经验LE和ST环流所满足的各种类型的涨落定理,并阐述了它们的适用范围。其中表明了经验的绝对(净)LE 环流满足所有类型的涨落定理和对称关系。特别是,我们引入了相似环的概念,并得到暂态涨落定理的强形式。(i) 当相似环中的某一对被交换时,经验LE环流的联合分布满足一个对称关系;(ii)当任何一对环的净环流取其相反数时,经验性LE净环流的联合分布满足对称关系。当任何周期的净环流被其相反的数字取代时,满足对称关系。由于在周期性边界下,经验的 ST 环流可以用的经验LE环流的加权和来表示。进一步表明,经验ST环流不满足任何形式的涨落定理,然而经验ST净环流只在周期性边界条件下满足弱形式的暂态涨落定理:当基本集合中所有环的净环流取其相反数时,经验净ST净环流的联合分布满足一个对称关系。

在本文中,一些结果只针对单环马氏链得出。希望这些结果可以被推广到更普遍的马氏系统,甚至是半马氏或非马氏系统。此外,本文只对LE和ST的环流进行了比较。这两类环流与以序列匹配方式定义的环流 \cite{roldan2019exact,biddle2020reversal,pietzonka2021cycle} 之间的关系还不清楚,这些工作尚处于调研中。