% !Mode:: "TeX:UTF-8"

\BiChapter{模型介绍}{}
在此考虑基于离散时间马氏链$\xi = (\xi_n)_{n \ge 0}$建模的随机热力学系统,该模型的状态空间是$S = \{1, \dots N\}$转移概率矩阵是$P=(p_{ij})_{i,j \in S}$,其中$p_{ij}$表示从状态$i$到状态$j$的转移概率。该马氏链对应的转移图为$G=(S, E)$,其中顶点集合$S$是状态空间,$E$是转移概率有向边的集合 \ref{figure:transitiongraph}。本文用$\langle i, j\rangle$表示状态$i$到状态$j$的有向边,因此有$E = \{\langle \langle i, j\rangle \in S \times S: p_{ij}>0\}$,并且令$|E| = M$,其中$|E|$表示集合$E$中元素的数量。本文所考虑的马氏链都是不可约的,也就是有向图$G$是连通的。因此对某个状态,图$G$不仅包含其他状态到其的边,还包括它到其自身的边,也就是自循环 \ref{figure:transitiongraph}。

转移图$G$中如果有一个如图 \ref{figure:transitiongraph}所示的图拓扑(除了自循环),将被视为一种特殊情形,在以往的研究中称该系统为单环马氏链。具体来说,如果马氏链$\xi$满足 $p_{ij}=0$且$|i-j| \ge 2$(其中的$i,j$是模$N$运算后的结果),则称为单环马氏链。单环系统在生物学方面具有特殊的意义。许多重要的生化过程,如酶和离子通道的构象变化\cite{cornish2013fundamentals,sakmann2013single},磷酸化-磷酸化循环\cite{beard2008chemical},甲基化-去甲基化循环\cite{jia2017nonequilibrium},以及染色体重塑导致的启动子的激活\cite{pedraza2008effects,jia2022analytical}都可以被建模为单环马氏链。下文中,主要关注单环系统,大部分的结论也可以扩展到一般系统。

\begin{figure}[h]\label{figure:transitiongraph}
\centering
\includegraphics[scale=0.5]{chart/transitiongraph.pdf}
\caption{各类马氏链的转移图和相应的生成树 (a) 一般转移图的马氏链, 绿色线表示根节点为4的生成树$T$,并且蓝色线表示$T$的弦 (b) 四状态全连接马氏链,其中每个状态可以转移到自身和其他状态 (c) 
$N$状态单环马氏链。系统有一个环拓扑。每个状态转移到自身和它两个相邻节点 (d) 一个N状态的单环马尔科夫链,该系统无法从状态$1$转移到状态$N$ (b)-(d)中,绿色箭头表示生成树T}
\end{figure}

\BiSection{环擦除模式定义的环流}{}
本文调查和比较了马氏链中两种类型的环流。首先回顾环擦除模式定义的环流\cite{jiang2004mathematical,kalpazidou2007cycle}。马氏链中的回路是一个状态到自身的路径,比如路径$i_1 \to i_2 \to\cdots\to i_s \to i_1$(其中$i_1, i_2 , \cdots, i_s$是顶点集合$S$中不同点),其中$p_{i_1i_2}p_{i_2i_3}\cdots p_{i_si_1}>0$。令$j_1 \to j_2 \to\cdots\to j_r \to j_1$为另一个回路,若上述两个环满足存在一个整数$k$使得
\begin{equation*}
    j_1 = i_{k+1},j_2 = i_{k+2},\cdots,j_n = i_{k+s},
\end{equation*}
且$r=s$则称两个回路是等价的,其中指标$k+1,k+2,\cdots,k+s$被视为模$n$同余的。回路$i_1 \to i_2 \to\cdots\to i_s \to i_1$在上述等价关系下所属的等价类被表示为环$c = (i_1,i_2,\cdots,i_s)$。例如,$(1,2,3)$,$(2,3,1)$和$(3,1,2)$表示相同的环。定义环$(i_1,i_2,\cdots,i_s)$的反环为$(i_1,i_s,\cdots,i_2)$。系统中所用环的集合称为环空间$\mathcal{C}$。

沿着马氏链轨道的会不断形成各种类型的环。直观看,如果我们抛弃马氏链$\xi$中环,并且在该过程中,持续聚焦在轨道中剩余状态的轨迹,那么将会获得一个新的马氏链$\tilde{\xi} = (\tilde{\xi}_n)_{n\geq 0}$,称为导出链。例如,如果原始的马氏链$\xi$的轨道为$\{1,2,3,3,2,3,4,1,4,\cdots\}$,那么相应的导出链$\tilde{\xi}$的轨迹和形成的环为表 \ref{trajectory}所示。
\begin{table}[htb!]
    \renewcommand\arraystretch{1}\centering
    \begin{tabular}{cccccccccc} \hline\hline
   $n$            & 0 & 1 & 2 & 3   & 4     & 5 & 6 & 7         & 8 \\ \hline
   $\xi_n$         & 1 & 2 & 3 & 3   & 2     & 3 & 4 & 1         & 4 \\ \hline
   $\tilde{\xi}_n$& {[}1{]} & {[}1,2{]} & {[}1,2,3{]} & {[}1,2,3{]} & {[}1,2{]} & {[}1,2,3{]} & {[}1,2,3,4{]} & {[}1{]} & {[}1,4{]} \\ \hline
    \text{形成的环} &   &   &   & (3) & (2,3) &   &   & (1,2,3,4) &   \\ \hline\hline
    \end{tabular}
    \caption{导出链和形成环的案例}\label{trajectory}
\end{table}

特别地,导出链的状态用$S$的中状态组成的有限序列表示,即$[i_1,i_2,\cdots,i_s]$。假设$\tilde{\xi}_{n-1}=[i_1,i_2,\cdots,i_s]$且$\xi_n = i_{s+1}$。若$i_{s+1}$不同于$i_1,i_2,\cdots,i_s$,那么$\tilde{\xi}_n$被定义为$\tilde{\xi}_n = [i_1,i_2,\cdots,i_s,i_{s+1}]$。其次,若$i_{n+1}=i_r$,那么$\tilde{\xi}_n$被定义为$\tilde{\xi}_n = [i_1,i_2,\cdots,i_r]$。对于这种情况,称马氏链在时间$n$形成环$(i_r,i_{r+1},\cdots,i_s)$。令$N^c_n$为环$c$在时间$n$时形成的次数。那么在时间$n$时,环$c$的经验环流可以被定义为:
\begin{equation*}
    J_n^c = \frac{1}{n}N^c_n,
\end{equation*}
并且在时间$n$时,经验净环流可以定义为:
\begin{equation*}
    \tilde{J}^c_n = J^c_n-J^{c-}_n.
\end{equation*}
用更直观的表述,$J^c_n$表示环$c$平均每个单位时间形成的数量,$\tilde{J}^c$表示环$c$平均每个单位时间形成的净数量。
若令$n\rightarrow\infty$,则有经验环流$J^c_n$和 经验净环流$\tilde{J}^c$分别以概率为 1 趋近于$J^c$和$\tilde{J}^c$。
极限$J^c$和$\tilde{J}^c$分别被视为环$c$的环流和净环流。关于$J^c_n$和$\tilde{J}^c$更为细致的描述,可以参考文献 [1]。著名的环流分解定理可以用上述定义表示为:
\begin{equation}\label{decomposition}
    \pi_ip_{ij} = \sum_{c\ni\langle i,j\rangle}J^c,
\end{equation}
其中$c \ni \langle i, j\rangle$表示环$c$中有边$\langle i, j\rangle$。

\BiSection{生成树定义下的环流}{}
环的环流可以用生成树的方式定义。令$T$为$G$的一个有向子图,也就是说$T$的所有边也是$G$的边,再令$\hat{T}$表示$\bar{T}$表示与$T$有关的无向图。
满足下列三个条件的$T$被称为图$G$的生成树:
\begin{itemize}
    \item$T$是$G$的覆盖子图,也就是说$T$包含$G$的所有顶点。
    \item$\bar{T}$是连通的。
    \item$\bar{T}$没有回路,其中无向图的回路是顶点到自身的无向路径。
\end{itemize}

易知生成树$T$包含$G$的所有顶点,并且$|T| = N -1$。接下来,会通过$T$表示生成树和它的边集合。图的生成树并不唯一,一个图可以有很多完全不同的生成树。

若有向边$l \notin T$,则被称为$T$的弦。(图1 a)。因为$|T|= N-1$,生成树$T$有$M-N+1$个弦,也就是说明图$G$中有$M-N+1$条边不会在$T$中出现。同时$\bar{T}$是连通的,并且没有环的,如果添加一根弦$l$到$T$,则导出的子图$\overline{T \cup \{l\}}$会恰巧有一个回路,表示其为$C_l$。记$c_l$是从 回路$C_l$中获得的环,并且和$l$保持同样的指向。由弦生成的环的集合$\mathcal{L} = \{c_l: l\in E\setminus T\}$被称为基本集。很明显,弦和基本集之间没有一一对应的关系,可以用$c_l$形成的次数定义通过弦$l$的次数。那么经验时间$n$时刻,$c_l$的经验环流可以被定义为:
\begin{equation*}
    Q^{c_l}_n = \frac{1}{n}\sum_{i=1}^n1_{\{\langle\xi_{i-1},\xi_i\rangle=l\}}.
\end{equation*}
$Q^{c_l}_n$表示单位时间通过弦$l$的次数。生成树方式和环擦除方式有很大的不同,环擦除可以很便捷的定义环流,生成树只能定义基本集的环流。

类似的,也可以用生成树定义净环流。最终,假设转移概率满足$p_{ij}>0$,当且仅当$p_{ji}>0$,这个条件可以保障熵增量是有限的。【1】考虑$l
$,在时间步$n$,$c_l$的净环流定义为:
\begin{equation*}
    \tilde{Q}^{c_l}_n=Q^{c_l}_n-Q^{c_l-}_n.
\end{equation*}
如果$c_l$中只有一个或两个状态,那么易知$c_l = c_{l^-}$,因此$\tilde{Q}_n^{c_l}=0$。对于弦$l=\langle i,j \rangle$,如果$c_l$中有三个以上的状态,那么$l^-$也是一个弦,并且$c_{l^-}$是由弦$l^-$生成的环。在文献 [3-6],净环流的定义只是考虑有三个及以上状态的的环,本文参考文献【2】中的定义,使得净环流的定义也可以包含只有一个或两个状态的环。

若$n \to \infty$, 则经验环流$Q_n^{c_l}$和经验净环流$\tilde{Q}_n^{c_l}$将会分别以概率为$1$趋于$Q^{c_l}$和$\tilde{Q}^{c_l}$。极限$Q^{c_l}$和$\tilde{Q}^{c_l}$分别被当做环$c$的环流与净环流。根据马氏链的遍历性,有$Q^{c_l} = \pi_i p_{ij}$。

\BiSection{两种类型环流的比较}{}
下面将简述两种类型环流的差异。为叙述简便,由环擦除方式定义的环流称为$LE$环流,由生成树方式定义的环流称为$ST$环流。$LE$环流是对环空间$\mathcal{C}$中所有环定义的,然而$ST$环流仅是针对基本集$\mathcal{L}$定义的。因此,对于马氏链的环动态性,$LE$环流相较于$ST$环流给出了更完整的描述。而且,由于生成树不具有唯一性,不同的生成树选择会对应不同的$ST$环流。对比之下,$LE$环流并不依赖生成树的选择。

经过上述的比对,自然会问到基本集$\mathcal{L}$和环空间$\mathcal{C}$的差距会有多大。因为每根弦对应集合$\mathcal{L}$唯一一个元素,即$|\mathcal{L}| = |E\setminus T| = M-N+1$,所以很难对$|\mathcal{C}|$给出一般性的表述。下面将主要考虑几个重要的特例。首先考虑转移图是全连接的马氏链(图1b),也就是$p_{ij}>0, \forall i,j \in S$。有$k$个状态的环的数量是$\frac{N (N-1) \cdots (N-K+2)(N-k+1)}{k}$,因此
\begin{equation*}
    |\mathcal{C}| = \sum_{k=1}^N\frac{N\cdots (N-k+1)}{k}.
\end{equation*}
特别地,若$N=4$,可得$|\mathcal{C}|=24$,环空间为:
\begin{align*}
    \mathcal{C} = \{&(1),(2),(3),(4),(1,2),(1,3),(1,4),(2,3),(2,4),(3,4),\\
    &(1,2,3),(1,2,4),(1,3,2),(1,3,4),(1,4,2),(1,4,3),(2,3,4),(2,4,3),\\
    &(1,2,3,4),(1,2,4,3),(1,3,2,4),(1,3,4,2),(1,4,2,3),(1,4,3,2)\}.
\end{align*}
若选择 $T = 1\to 2\to 3\to 4$作为生成树,那么有基本集$|\mathcal{L}|=13$,且可以表示为:
\begin{align*}
    \mathcal{L} = \{&(1),(2),(3),(4),(1,2),(2,3),(3,4)\\
    &(1,2,3),(1,3,2),(2,3,4),(2,4,3),(1,2,3,4),(1,4,3,2)\}.
\end{align*}
这也说明了$LE$环流的值可能会和$ST$的值差很多。

接下来考虑单环马氏链 【图1 c]】。为了叙述方便,假设任意一对相邻状态$i$和$j$满足$p_{ii}>0$和$p_{ij}>0$。对于这种情况,$|\mathcal{C}| = 2N +2$,并且环空间为:

其中前$N$个环是只包含一个状态的环,也就是自循环的。中间$N$个环是两状态的环,最后两个环是$N$个状态的环。如果选择$T = 1\to 2\to\cdots \to N$作为生成树,那么$|\mathcal{L}| = 2N + 1$,并且基本集是:
\begin{equation*}
    \mathcal{L} = \{(1),\cdots,(N),(1,2),\cdots,(N-1,N),(1,2,\cdots,N),(1,N,\cdots,2)\}.
\end{equation*}
对于这种情况,只有一个环$(N, 1)$在环空间$C$中出现,而没有在基本集$\mathcal{L}$中出现。

最后考虑单环马氏链 [图1d]。从状态$1$到状态$N$的转移概率为$0$。为了叙述方便,假设$p_{1N}=0$,且对于其他相邻状态$i,j$,马氏链满足$p_{ii}>0$并且$p_{ij}>0$。易知$|\mathcal{C}|=2N$,环空间为:
\begin{equation*}
    \mathcal{C} = \{(1),\cdots,(N),(1,2),\cdots,(N-1,N),(1,\cdots,N)\}.
\end{equation*}
若令$T = 1\to 2\to\cdots \to N$为生成树,那么有$\mathcal{L} = \mathcal{C}$,也就是说环空间和基本集是一致的。而且,很容易验证两种类型的环流也是一致的,也就是:
\begin{equation}\label{same cycle current}
    Q_n^{c_l} = J^{c_l}_n,\;\;\;c_l\in \mathcal{L}.
\end{equation}

为了进一步理解经验$LE$环流$J_n^c$和经验$ST$环流$Q_n^c$的关系。下面使用周期边界条件假设,也就是$\xi_0 = \xi_1$,这是文献 [7]中标准的假设条件。基于这个假设,对任意弦$l$,易得:
\begin{equation}\label{conversion}
    Q_n^{c_l} = \sum_{c\in\mathcal{C}}J^c_n1_{\{l\in c\}}.
\end{equation}
注意到上述方程两边都表示弦$l$形成的速度。这表、说明经验$ST$环流可以表示为经验$LE$环流的线性组合。