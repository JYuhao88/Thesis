\begin{appendices}

\BiChapter{附录 A:单环马氏链速率函数 $I_J$ 的表达式 }{} \label{appendix:explicit} \label{appendix:explicit}
因此,可以得到速率函数:
\begin{equation*}\label{ratefunction2}
	\begin{split}
		I_J(\nu) =&\; \left[h\left(\nu^{12}\right)+h\left(\nu^{1N}\right)
		+h\left(\nu^+\right)+h\left(\nu^-\right)-h\left(\nu^{12}+\nu^{1N}+\nu^++\nu^-\right)\right] \\
		&\;+\inf_{X\in V(\nu)}F_{\nu}(X)+\sum_{i\in S}\left[ h\left(\nu_i-\nu^i\right)+h\left(\nu^i\right)
		-h\left(\nu_i\right)\right]-\sum_{c\in\mathcal{C}}\nu^c\log\gamma^c,
	\end{split}
\end{equation*}
下面将使用拉格朗日乘子法化简该式。固定 $\nu\in\mathcal{V}$,令拉格朗日函数 \cite{brian1990optimization} $\mathcal{A}_{\nu}:V(\nu)\times \mathbb{R}^{N-2}\to \mathbb{R}$ 为:
\begin{align*}
    \mathcal{A}_{\nu}(X,\lambda) = F_{\nu}(X) + \sum_{i=2}^{N-1} \lambda_i \left(x^{i} + y^{i} - \nu^{i,i+1}\right),
\end{align*}
其中 $X=(x^i,y^i)_{2\le i\le N-1}\in V(\nu)$ 并且 $\lambda=(\lambda_i)_{2\le i\le N-1}\in \mathbb{R}^{N-2}$。分别对 $x^{i}$,$y^{i}$ 和 $\lambda_i$求导,可以得到下列方程:
\begin{equation}\label{equation1}
	\begin{split}
		\log\left(x^{i}\right) - \log\left(x^{i-1}+x^{i}+\nu^{+}\right)  + \log\left(x^{i}+\nu^{+}\right) -\log\left(x^{i}+x^{i+1}+\nu^{+}\right)+\lambda_i  &= 0, \\
		\log\left(y^{i}+\nu^{-}\right) -\log\left(y^{i-1}+y^{i}+\nu^{-}\right)  + \log\left(y^{i}\right) - \log\left(y^{i}+y^{i+1}+\nu^{-}\right) +\lambda_i &= 0, \\
		x^{i} + y^{i} = \nu^{i,i+1},\qquad 2\le i\le N-1.\qquad\qquad\qquad
	\end{split}
\end{equation}
而且,可以把方程组 \eqref{equation1} 写为:
\begin{equation}\label{equations}
    \begin{split}
    \frac{x^{i}}{x^{i-1}+x^{i}+\nu^+}
    \,\frac{x^{i}+\nu^+}{x^{i}+x^{i+1}+\nu^+}
    &= \frac{y^{i}+\nu^-}{y^{i-1}+y^{i}+\nu^-}
    \,\frac{y^{i}}{y^{i}+y^{i+1}+\nu^-}=e^{-\lambda_i},\\
    x^{i} + y^{i} &= \nu^{i,i+1},\qquad 2\le i\le N-1.
    \end{split}
\end{equation}
其中 $x^1=\nu^{12}$,$x^N=0$,$y^1=0$ 和 $y^N=\nu^{1N}$。
    \begin{lemma}\label{lemma:existence for equations solution}
        方程 \eqref{equations} 有解 $X=(x^i,y^i)\in V(\nu)$。
    \end{lemma}
\begin{proof}
    若对某些 $2\le k\le N-1$ 存在 $\nu^{k,k+1}=0$,则有 $x^{k}=y^{k}=0$ 成立。那么依据指标 $2\le i\le k-1$ 和 $k+1\le i\le N-1$,方程 \eqref{equations} 可以被分为两个方程。因此对 $\nu^{k,k+1}>0, \forall k, 2\le k\le N-1$ 证明引理,下面会从三种不同情况考虑这个引理。

    情况 1:$\nu^{12}=\nu^{+}=\nu^{1N}=\nu^-=0$。
    易知,对每个 $\alpha\in (0,1)$,
    \begin{equation*}
        x^{i}=\alpha\nu^{i,i+1},\quad y^{i}=(1-\alpha)\nu^{i,i+1},\quad 2\le i\le N-1,
    \end{equation*}
    都是 \eqref{equations} 的解。

    情况 2:$\nu^{12}=\nu^{+}=0,\nu^{1N}+\nu^{-}>0$ 或 $\nu^{1N}=\nu^{-}=0,\nu^{12}+\nu^{+}>0$。容易验证若 $\nu^{12}=\nu^{+}=0,\nu^{1N}+\nu^{-}>0$成立,则$x^{i}=0,y^{i}=\nu^{i,i+1}$ 是方程 \eqref{equations} 的解。

    情况 3:$\nu^{12}+\nu^{+}>0$ 或 $\nu^{1N}+\nu^{-}>0$。上述已证明对任意给定的 $x^{k+1}\ge 0$,$y^{k+1}>0$,和 $x^{k+1}+y^{k+1}=\nu^{k+1,k+2}$,下列方程 \eqref{equations2} 满足 $x^{i},y^{i}> 0, \forall i, 2\le i\le k$。
    \begin{equation}\label{equations2}
        \begin{split}
            \frac{x^{i}}{y^{i}}
            =&\frac{x^{i}+x^{i+1}+\nu^+}{x^{i}+\nu^+}\, \frac{y^{i}+\nu^-}{y^{i-1}+y^{i}+\nu^-}
            \,\frac{x^{i-1}+x^{i}+\nu^+}{y^{i}+y^{i+1}+\nu^-},\\
            &\qquad x^{i} + y^{i} = \nu^{i,i+1},\qquad 2\le i\le k.
        \end{split}
    \end{equation}

    下面通过归纳法证明。若 $k=2$,方程 \eqref{equations2} 可以简化为:
    \begin{equation*}\label{equation k=2}
        \frac{x^{2}}{\nu^{23}-x^2}
        =\frac{x^{2}+x^{3}+\nu^+}{x^{2}+\nu^+}\, \frac{\nu^{12}+x^{2}+\nu^{+}}{\nu^{23}-x^2+y^3+\nu^-}.
        %,\qquad \nu^+_{23} + \nu^-_{23} = \nu_{23}.
    \end{equation*}
    易得:
    \begin{equation*}
        \lim_{x^{2}\downarrow 0}\frac{x^{2}}{\nu^{23}-x^2} = 0,\qquad \lim_{x^{2}\downarrow 0}\frac{x^{2}+x^{3}+\nu^+}{x^{2}+\nu^+}\, \frac{\nu^{12}+x^{2}+\nu^{+}}{\nu^{23}-x^2+y^3+\nu^-} \ge  \frac{\nu^{12}+\nu^{+}}{\nu^{23}+y^{3}+\nu^-} > 0.
    \end{equation*} 
    另外,
    \begin{equation*}
        \lim_{x^2\uparrow \nu^{23}}\frac{x^{2}}{\nu^{23}-x^2} = \infty,\qquad \lim_{x^{2}\uparrow \nu^{23}}\frac{x^{2}+x^{3}+\nu^+}{x^{2}+\nu^+}\, \frac{\nu^{12}+x^{2}+\nu^{+}}{\nu^{23}-x^2+y^3+\nu^-} =  \frac{\nu^{23}+x^{3}+\nu^+}{\nu^{23}+\nu^+}\frac{\nu^{12}+\nu^{23}+\nu^{+}}{y^{3}+\nu^-} < \infty.
    \end{equation*}
    通过中值定理,可以找到满足 $x^{i},y^i>0$ 的方程 \eqref{equations2} 的解,并且有 $x^{i}+y^{i}=\nu^{i,i+1}$ 。假设 $

\end{proof}

\end{appendices}
