\section{附录}

\section{附录 A:单环马氏链速率函数 $I_J$ 的表达式 }{} \label{appendix:explicit} \label{appendix:explicit}
因此,可以得到速率函数:
\begin{equation*}\label{ratefunction2}
	\begin{split}
		I_J(\nu) =&\; \left[h\left(\nu^{12}\right)+h\left(\nu^{1N}\right)
		+h\left(\nu^+\right)+h\left(\nu^-\right)-h\left(\nu^{12}+\nu^{1N}+\nu^++\nu^-\right)\right] \\
		&\;+\inf_{X\in V(\nu)}F_{\nu}(X)+\sum_{i\in S}\left[ h\left(\nu_i-\nu^i\right)+h\left(\nu^i\right)
		-h\left(\nu_i\right)\right]-\sum_{c\in\mathcal{C}}\nu^c\log\gamma^c,
	\end{split}
\end{equation*}
下面将使用拉格朗日乘子法化简该式。固定 $\nu\in\mathcal{V}$,令拉格朗日函数 \cite{brian1990optimization} $\mathcal{A}_{\nu}:V(\nu)\times \mathbb{R}^{N-2}\to \mathbb{R}$ 为:
\begin{align*}
    \mathcal{A}_{\nu}(X,\lambda) = F_{\nu}(X) + \sum_{i=2}^{N-1} \lambda_i \left(x^{i} + y^{i} - \nu^{i,i+1}\right),
\end{align*}
其中 $X=(x^i,y^i)_{2\le i\le N-1}\in V(\nu)$ 并且 $\lambda=(\lambda_i)_{2\le i\le N-1}\in \mathbb{R}^{N-2}$。分别对 $x^{i}$,$y^{i}$ 和 $\lambda_i$求导,可以得到下列方程:
\begin{equation}\label{equation1}
	\begin{split}
		\log\left(x^{i}\right) - \log\left(x^{i-1}+x^{i}+\nu^{+}\right)  + \log\left(x^{i}+\nu^{+}\right) -\log\left(x^{i}+x^{i+1}+\nu^{+}\right)+\lambda_i  &= 0, \\
		\log\left(y^{i}+\nu^{-}\right) -\log\left(y^{i-1}+y^{i}+\nu^{-}\right)  + \log\left(y^{i}\right) - \log\left(y^{i}+y^{i+1}+\nu^{-}\right) +\lambda_i &= 0, \\
		x^{i} + y^{i} = \nu^{i,i+1},\qquad 2\le i\le N-1.\qquad\qquad\qquad
	\end{split}
\end{equation}
而且,可以把方程组 \eqref{equation1} 写为:
\begin{equation}\label{equations}
    \begin{split}
    \frac{x^{i}}{x^{i-1}+x^{i}+\nu^+}
    \,\frac{x^{i}+\nu^+}{x^{i}+x^{i+1}+\nu^+}
    &= \frac{y^{i}+\nu^-}{y^{i-1}+y^{i}+\nu^-}
    \,\frac{y^{i}}{y^{i}+y^{i+1}+\nu^-}=e^{-\lambda_i},\\
    x^{i} + y^{i} &= \nu^{i,i+1},\qquad 2\le i\le N-1.
    \end{split}
\end{equation}
其中 $x^1=\nu^{12}$,$x^N=0$,$y^1=0$ 和 $y^N=\nu^{1N}$。
    \begin{lemma}\label{lemma:existence for equations solution}
        方程 \eqref{equations} 有解 $X=(x^i,y^i)\in V(\nu)$。
    \end{lemma}
\begin{proof}
    若对某些 $2\le k\le N-1$ 存在 $\nu^{k,k+1}=0$,则有 $x^{k}=y^{k}=0$ 成立。那么依据指标 $2\le i\le k-1$ 和 $k+1\le i\le N-1$,方程 \eqref{equations} 可以被分为两个方程。因此对 $\nu^{k,k+1}>0, \forall k, 2\le k\le N-1$ 证明引理,下面会从三种不同情况考虑这个引理。

    情况 1:$\nu^{12}=\nu^{+}=\nu^{1N}=\nu^-=0$。
    易知,对每个 $\alpha\in (0,1)$,
    \begin{equation*}
        x^{i}=\alpha\nu^{i,i+1},\quad y^{i}=(1-\alpha)\nu^{i,i+1},\quad 2\le i\le N-1,
    \end{equation*}
    都是 \eqref{equations} 的解。

    情况 2:$\nu^{12}=\nu^{+}=0,\nu^{1N}+\nu^{-}>0$ 或 $\nu^{1N}=\nu^{-}=0,\nu^{12}+\nu^{+}>0$。容易验证若 $\nu^{12}=\nu^{+}=0,\nu^{1N}+\nu^{-}>0$成立,则$x^{i}=0,y^{i}=\nu^{i,i+1}$ 是方程 \eqref{equations} 的解。

    情况 3:$\nu^{12}+\nu^{+}>0$ 或 $\nu^{1N}+\nu^{-}>0$。上述已证明对任意给定的 $x^{k+1}\ge 0$,$y^{k+1}>0$,和 $x^{k+1}+y^{k+1}=\nu^{k+1,k+2}$,下列方程 \eqref{equations2} 满足 $x^{i},y^{i}> 0, \forall i, 2\le i\le k$。
    \begin{equation}\label{equations2}
        \begin{split}
            \frac{x^{i}}{y^{i}}
            =&\frac{x^{i}+x^{i+1}+\nu^+}{x^{i}+\nu^+}\, \frac{y^{i}+\nu^-}{y^{i-1}+y^{i}+\nu^-}
            \,\frac{x^{i-1}+x^{i}+\nu^+}{y^{i}+y^{i+1}+\nu^-},\\
            &\qquad x^{i} + y^{i} = \nu^{i,i+1},\qquad 2\le i\le k.
        \end{split}
    \end{equation}

    下面通过归纳法证明。若 $k=2$,方程 \eqref{equations2} 可以简化为:
    \begin{equation*}\label{equation k=2}
        \frac{x^{2}}{\nu^{23}-x^2}
        =\frac{x^{2}+x^{3}+\nu^+}{x^{2}+\nu^+}\, \frac{\nu^{12}+x^{2}+\nu^{+}}{\nu^{23}-x^2+y^3+\nu^-}.
        %,\qquad \nu^+_{23} + \nu^-_{23} = \nu_{23}.
    \end{equation*}
    易得:
    \begin{equation*}
        \lim_{x^{2}\downarrow 0}\frac{x^{2}}{\nu^{23}-x^2} = 0,\qquad \lim_{x^{2}\downarrow 0}\frac{x^{2}+x^{3}+\nu^+}{x^{2}+\nu^+}\, \frac{\nu^{12}+x^{2}+\nu^{+}}{\nu^{23}-x^2+y^3+\nu^-} \ge  \frac{\nu^{12}+\nu^{+}}{\nu^{23}+y^{3}+\nu^-} > 0.
    \end{equation*} 
    另外,
    \begin{equation*}
        \lim_{x^2\uparrow \nu^{23}}\frac{x^{2}}{\nu^{23}-x^2} = \infty,\qquad \lim_{x^{2}\uparrow \nu^{23}}\frac{x^{2}+x^{3}+\nu^+}{x^{2}+\nu^+}\, \frac{\nu^{12}+x^{2}+\nu^{+}}{\nu^{23}-x^2+y^3+\nu^-} =  \frac{\nu^{23}+x^{3}+\nu^+}{\nu^{23}+\nu^+}\frac{\nu^{12}+\nu^{23}+\nu^{+}}{y^{3}+\nu^-} < \infty.
    \end{equation*}
    通过中值定理,可以找到满足 $x^{i},y^i>0$ 的方程 \eqref{equations2} 的解,并且有 $x^{i}+y^{i}=\nu^{i,i+1}$ 。假设对 $k=n-1$ 命题成立,那么可以考虑方程:
    \begin{equation*}
        \frac{x^{n}}{\nu^{n,n+1}-x^{n}} = \frac{x^{n}+x^{n+1}+\nu^+}{x^{n}+\nu^+}\,\frac{\nu^{n,n+1}-x^{n}+\nu^-}{y^{n-1}+\nu^{n,n+1}-x^{n}+\nu^-}\,\frac{x^{n-1}+x^{n}+\nu^+}{\nu^{n,n+1}-x^{n}+y^{n+1}+\nu^-},
    \end{equation*}
    其中 $x^{n-1}$, $y^{n-1}$ 是方程 \eqref{equations2} 在 $k=n-1$ 时的解。可以得到:
    \begin{equation}\label{limit1}
        \lim_{x^{n}\downarrow 0}\frac{x^{n}}{\nu^{n,n+1}-x^{n}}=0,
    \end{equation}
    并且
    \begin{equation}\label{limit2}
        \begin{split}
            &\;\lim_{x^{n}\downarrow 0}\frac{x^{n}+x^{n+1}+\nu^+}{x^{n}+\nu^+}\,\frac{\nu^{n,n+1}-x^{n}+\nu^-}{y^{n-1}+\nu^{n,n+1}-x^{n}+\nu^-}\,\frac{x^{n-1}+x^{n}+\nu^+}{\nu^{n,n+1}-x^{n}+y^{n+1}+\nu^-}\\
            \ge &\;\frac{\nu^{n,n+1}+\nu^-}{\lim_{x^{n}\downarrow 0}(y^{n-1})+\nu^{n,n+1}+\nu^-}\,\frac{\lim_{x^{n}\downarrow 0}(x^{n-1})+\nu^+}{\nu^{n,n+1}+y^{n+1}+\nu^-}>0.
        \end{split}
    \end{equation}
    利用归纳法得出的结论 $\lim_{x^{n}\downarrow 0}(x^{n-1})>0$,可以得到下面两个式子成立。
    \begin{equation}\label{limit3}
        \lim_{x^{n}\uparrow \nu^{n,n+1}}\frac{x^{n}}{\nu^{n,n+1}-x^{n}}=\infty,
    \end{equation}
    和
    \begin{equation}\label{limit4}
        \begin{split}
            &\;\lim_{x^{n}\uparrow \nu^{n,n+1}}\frac{x^{n}+x^{n+1}+\nu^+}{x^{n}+\nu^+}\,\frac{\nu^{n,n+1}-x^{n}+\nu^-}{y^{n-1}+\nu^{n,n+1}-x^{n}+\nu^-}\,\frac{x^{n-1}+x^{n}+\nu^+}{\nu^{n,n+1}-x^{n}+y^{n+1}+\nu^-}\\
            \le &\;\frac{\nu^{n,n+1}+x^{n+1}+\nu^+}{\nu^{n,n+1}+\nu^+}\,\frac{\lim_{x^{n}\uparrow \nu^{n,n+1}}x^{n-1}+\nu^{n,n+1}+\nu^+}{y^{n+1}+\nu^-}<\infty.
        \end{split}
    \end{equation}
    那么通过中值定理,可以找到方程 \eqref{equations2} 满足条件 $x^{i},y^i>0, x^{i}+y^{i}=\nu^{i,i+1}, \forall i=2,3, \cdots, n$ 的解,就是说上述命题对 $k=n$ 成立。

    当 $k=N-1$ 时,有 $y^{k+1}=y^N=\nu^{1N}$ 成立。注意到 $\nu^{1N}+\nu^->0$,那么可以得到 \eqref{limit1},\eqref{limit2},\eqref{limit3} 和 \eqref{limit4} 成立。通过中值定理,可以找到方程 \eqref{equations2} 满足条件 $x^{i},y^i>0, x^{i}+y^{i}=\nu^{i,i+1}, \forall i=2,3, \cdots, n$ 的解。当 $k=N-1$ 时,方程 \eqref{equations2} 与 方程 \eqref{equations} 等价。证毕。

\end{proof}
\begin{lemma}\label{lemma:mininum}
    令 $X=(x^i,y^i)\in V(\nu)$ 是引理 ~\ref{lemma:existence for equations solution} 的解,那么 $X$ 是 $F_{\nu}(\cdot)$ 定义在 $V(\nu)$ 下的最小值点。 
\end{lemma}
\begin{proof}
    对任意 $a_1,a_2,b_1,b_2\ge 0$,通过 log-sum 不等式(见 \eqref{log sum inequality}),可以得到:
    \begin{align}\label{log sum inequality 2}
        a_1\log\frac{a_1}{a_1+a_2}+a_2\log\frac{a_2}{a_1+a_2}\ge a_1\log\frac{b_1}{b_1+b_2}+a_2\log\frac{b_2}{b_1+b_2}.
    \end{align}
    对任意 $Z=(z^i,w^i)\in V(\nu)$,可以把 \eqref{formula:F} 写为:
    \begin{align*}
        F_{\nu}(Z)
        =&\;\sum_{i=2}^{N-1}\left[ \left(z^{i-1}+\nu^+\right)\log\frac{z^{i-1}+\nu^+}{z^{i-1}+z^{i}+\nu^+} + z^{i}\log\frac{z^{i}}{z^{i-1}+z^{i}+\nu^+}  \right]\\
        &\;+\sum_{i=2}^{N-1}  \left[w^{i}\log\frac{w^{i}}{w^{i} +w^{i+1} +\nu^-} + \left(w^{i+1} +\nu^-\right)\log\frac{w^{i+1} +\nu^-}{w^{i} +w^{i+1} +\nu^-}\right],
    \end{align*}
    其中 $z^{1}=\nu^{12},w^{N}=\nu^{1N}$。再根据方程 \eqref{equations} 和 \eqref{log sum inequality 2}
    \begin{align*}
        F_{\nu}(Z)\ge &\;\sum_{i=2}^{N-1}\left[ \left(z^{i-1}+\nu^+\right)\log\frac{x^{i-1}+\nu^+}{x^{i-1}+x^{i}+\nu^+} + z^{i}\log\frac{x^{i}}{x^{i-1}+x^{i}+\nu^+}  \right]\\
        &\;+\sum_{i=2}^{N-1}  \left[w^{i}\log\frac{y^{i}}{y^{i} +y^{i+1} +\nu^-} + \left(w^{i+1} +\nu^-\right)\log\frac{y^{i+1} +\nu^-}{y^{i} +y^{i+1} +\nu^-}\right]\\
        =&\;\sum_{i=2}^{N-1}\left[ \nu^+\log\frac{x^{i-1}+\nu^+}{x^{i-1}+x^{i}+\nu^+} + z^{i}\log\frac{x^{i}+\nu^{+}}{x^{i-1}+x^{i}+\nu^{+}}\frac{x^{i}}{x^{i-1}+x^{i}+\nu^+}  \right]\\
        &\;+\sum_{i=2}^{N-1}  \left[\nu^-\log\frac{y^{i+1} +\nu^-}{y^{i} +y^{i+1} +\nu^-}+w^{i}\log\frac{y^{i}+\nu^{-}}{y^{i-1}+y^{i}+\nu^{-}}\frac{y^{i}}{y^{i} +y^{i+1} +\nu^-}  \right]\\
        &\;+\nu^{12}\log\frac{\nu^{12}+\nu^{+}}{\nu^{12}+x^{2}+\nu^{+}}+\nu^{1N}\log\frac{\nu^{1N}+\nu^{-}}{\nu^{1N}+y^{N-1}+\nu^{-}}\\
        =&\;\sum_{i=2}^{N-1}\left[-\lambda_i\nu^{i,i+1}+\nu^+\log\frac{x^{i-1}+\nu^+}{x^{i-1}+x^{i}+\nu^+}+\nu^-\log\frac{y^{i+1} +\nu^-}{y^{i} +y^{i+1} +\nu^-}\right]\\
        &\;+\nu^{12}\log\frac{\nu^{12}+\nu^{+}}{\nu^{12}+x^{2}+\nu^{+}}+\nu^{1N}\log\frac{\nu^{1N}+\nu^{-}}{\nu^{1N}+y^{N-1}+\nu^{-}}\\
        =&\;\sum_{i=2}^{N-1}\left[ \left(x^{i-1}+\nu^+\right)\log\frac{x^{i-1}+\nu^+}{x^{i-1}+x^{i}+\nu^+} + x^{i}\log\frac{x^{i}}{x^{i-1}+x^{i}+\nu^+}  \right]\\
        &\;+\sum_{i=2}^{N-1}  \left[y^{i}\log\frac{y^{i}}{y^{i} +y^{i+1} +\nu^-} + \left(y^{i+1} +\nu^-\right)\log\frac{y^{i+1} +\nu^-}{y^{i} +y^{i+1} +\nu^-}\right]\\
        =&\;F_{\nu}(X),
        \end{align*}
        其中 $\lambda_i$ 被表示为 \eqref{equations}。证毕。
\end{proof}

\section{速率函数的对称性} \label{appendix:symmetry}
下面将证明速率函数与初始分布的选择无关(见命题 \ref{corollary:rate function is unrelated to the starting state})。在开始证明之前,先给出下面的定义。

给定 $k=(k^c)_{c\in\mathcal{C}}\in \mathbb{N}^{2N+2}$ 和 $\eta=[i_0,i_1,\cdots,i_t]$,记 $|k|=\sum_{c\in\mathcal{C}}k^c|c|$,并且令 $n=|k|+|\eta|=|k|+t$,其中 $|c|$ 和 $|\eta|$ 分别表示环 $c$ 的长度和导出链的长度。令 $G^{\eta}(k)$ 为所有 $n$ 步长的轨道 $(\xi_m)_{0\le m\le n}$ 组成的集合,这些轨道满足环 $c$ 形成 $k^c$ 次,并且去环遗留的轨道是 $\eta$,定义为
\begin{equation*}
	G^{\eta}(k)=\left\{(\xi_0,\xi_1,\cdots,\xi_n)\in S^{n+1}:N^c_n=k^c (\forall c\in \mathcal{C}),\tilde{\xi}_n=\eta,\text{ and }n=|k|+|\eta|\right\}.
\end{equation*}
在此,对只有一个状态的 $\eta =[i]$,用 $G^{i}(k)$ 表示 $G^{\eta}(k)$。对任意包含状态 $i$ 的环 $\tilde{c}\in \mathcal{C}$,$G^i(k)$ 的一个子集 $G^{i,\tilde{c}}(k)$ 定义为:
\begin{equation*}
	G^{i,\tilde{c}}(k)=\left\{(\xi_0,\xi_1,\cdots,\xi_n)\in S^{n+1}:N^c_n=k^c (\forall c\in \mathcal{C}),\tilde{\xi}_n=[i],T^{\tilde{c}}_{k^{\tilde{c}}}=n,\text{ and }n=|k|\right\},
\end{equation*}
其中 $T^{\tilde{c}}_{k^{\tilde{c}}}$ 表示为第 $k^{\tilde{c}}$ 次形成的环 $\tilde{c}$。即对任意轨道 $(\xi_m)_{0\le m\le n}\in G^{i,\tilde{c}}(k)$,最后一次形成的环是 $\tilde{c}$。

注意到等式 $n=|k|+|\eta|$ 对 $G^{\eta}(k)$ 和 $G^{i,c}(k)$ 成立。因此在章节 ~\ref{LELDP} 省略 $G_n(k)$ 的角标 $n$。

下面将针对 $|G^i(k)|$ 和 $|G^{i,c}(k)|$,给出两个重要的等式。记 
\begin{equation*}
	B^i(\tilde{k})=\left|G^{i,+}(1,0,\tilde{k})\right|,\qquad	C^i(\tilde{k})=\left|G^{i,-}(0,1,\tilde{k})\right|,
\end{equation*}
其中 $\tilde{k}=(k^1,\cdots,k^N,k^{12},\cdots,k^{N-1,N},k^{1N})$,即 $B^i$ ($C^i$) 表示满足下列条件的轨迹数量:1)环 $c$ 形成 $k^c$ 次;2) $k^+=1,k^-=0$ ($k^+=0,k^-=1$);3)从状态 $i$ 出发,状态 $i$ 结束;4)最后形成的环是 $(1,\cdots,N)$ ($(1,N,\cdots,2)$)。

\begin{lemma}\label{lemma:BiBjCiCj}
	对任意 $i,j\in S$,有:
	\begin{equation*}
		B^i\left(\tilde{k}\right)=B^j\left(\tilde{k}\right)=C^i\left(\tilde{k}\right)=C^j\left(\tilde{k}\right).
	\end{equation*}
\end{lemma}

\begin{proof}
    不降低一般性,令 $i=1$,只须证 $B^1(\tilde{k})=C^1(\tilde{k})=B^j(\tilde{k})$。

    首先证明 $B^1(\tilde{k})=C^1(\tilde{k})$。
    \begin{figure}[h]
		\centering
		\includegraphics[scale=0.7]{chart/brokenlinegraph.pdf}
		\caption{$\xi$ 的轨迹对应的折线图。(a)第一种类型的轨迹映射(b)第二种类型的轨迹映射}
		\label{fig:linechart}
	\end{figure}
    注意到该系统是单环马氏链,$\xi$ 的每条轨迹可以被看作是一个折线图(见图 \ref{fig:linechart}(a))折线在状态 $i$ 时间 $m$ 处的水平移动表示环 $(i)$ 在时间 $m$ 处形成。折线从状态$i+1$ ($-j$) 到状态 $i$ ($1-j$) 的每次上升对应于 (其中的加减运算的结果都会经过模 $N$ 运算)。对任意轨迹 $\xi=(\xi_m)_{0\le m\le n}\in G^{1,+}(1,0,\tilde{k})$,令 $\eta(\xi)=\max\{m: 0\le m\le n-1,\xi_m=1\}$。回顾定义 $G^{1,+}(1,0,\tilde{k})$,轨迹 $(\xi_m)_{0\le m\le n}$ 在时间 $n$ 和状态 $N+1$ 到达集合 $\{N+1,1-N\}$。那么构造 $\xi$ 对应的轨迹 $\tilde{\xi}$(见图 \ref{fig:linechart} (a))
	\begin{equation*}
		\tilde{\xi}_m
		:=\left\{\begin{aligned}
			&\xi_m,    && \text{if } 0\le m\le \eta(\xi),\\
			&\xi_{n+\eta(\xi)-m},    && \text{if }\eta(\xi)<m\le n.\\
		\end{aligned}\right.
	\end{equation*}
    所有 $(\xi_m)_{0\le m\le \eta(\xi)}$ 中形成的环,在$(\tilde{\xi}_m)_{0\le m\le \eta(\xi)}$中也有形成。折线轨迹 $(\xi_m)_{\eta(\xi)\le m\le n}$ 在状态 $i$ 时间 $m$ 的水平移动对应于 $(\tilde{\xi}_m)_{\eta(\xi)\le m\le n}$ 在状态 $i-N$ 时间 $n+\eta(\xi)-m$ 的水平移动。类似的,折线轨迹 $(\xi_m)_{\eta(\xi)\le m\le n}$ 在时间 $m$ 从状态$i+1$ 到状态 $i$ 的下落对应于 $(\tilde{\xi}_m)_{\eta(\xi)\le m\le n}$ 在时间 $n+\eta(\xi)-m$ 从状态 $i-N$ 到状态 $i+1-N$ 的下落。那么所有 $(\xi_m)_{0\le m\le \eta(\xi)}$中形成的单状态和两状态环在 $(\tilde{\xi}_m)_{0\le m\le \eta(\xi)}$ 中也有对应。而且,轨迹 $(\tilde{\xi}_m)_{0\le m\le n}$ 在时间 $n$ 和状态 $1-N$第一次到达集合 $\{N+1,1-N\}$,这说明轨迹 $(\tilde{\xi}_m)_{0\le m\le n}$ 最后形成的环是 $(1,N,\cdots,2)$,使得 $\tilde{\xi}\in G^{1,-}(0,1,\tilde{k})$。那么 $\xi\to \tilde{\xi}$ 是从 $G^{1,+}(1,0,\tilde{k})$ 到 $G^{1,-}(0,1,\tilde{k})$ 的一一映射,这说明 $B^1(\tilde{k})=C^1(\tilde{k})$。

    现在已经证明 $B^1(\tilde{k})=B^j(\tilde{k})$。对任意 $\tilde{k}$,令 $\tilde{G}^i(\tilde{k})$ 为满足下列所有条件路径的集合,1)前 $n$ 步($n=N+\sum_{i \in S}(k^i+2k^{i,i+1})$)是从状态 $i$ 出发。2)边 $\langle i,i\rangle$ 被通过 $k^i$ 次。3)边 $\langle i,i+1\rangle$ 被通过 $k^{i,i+1}$ 次。4)在时间 $n$ 第一次到达状态 $N+i$。

    已经阐明对任意 $j\in S$,$|\tilde{G}^1(\tilde{k})|=|\tilde{G}^j(\tilde{k})|$ 成立。对任意轨迹 $\xi=(\xi_m)_{0\le m\le n}\in \tilde{G}^{1}(\tilde{k})$,令 $\kappa^j(\xi)=\min\{m: 0\le m\le n,\xi_k=j\}$,$\xi$也可以被视为折线图(见图 \ref{fig:linechart} (b))。那么构造 $\xi$ 对应的轨迹 $\tilde{\xi}$ 为:
    \begin{equation*}
		\tilde{\xi}_k
		:=\left\{\begin{aligned}
			&\xi_{\kappa(\xi)+k},    && \text{if } 0\le k\le n-\kappa(\xi),\\
			&\xi_{k+\kappa(\xi)-n},    && \text{if }n-\kappa(\xi)<k\le n.\\
		\end{aligned}\right.
	\end{equation*}
    因此 $(\tilde{\xi}_m)_{0\le m\le n} \in \tilde{G}^i(\tilde{k})$。注意到 $\kappa(\xi)$ 是 $\xi$ 第一次到达状态 $j$,这说明 $n$ 也是 $\tilde{\xi}$ 第一次到达状态 $j+N$,那么$\tilde{\xi}\in \tilde{G}^j(\tilde{k})$。显然,$\xi\to\tilde{\xi}$ 是从 $\tilde{G}^1(\tilde{k})$ 到 $\tilde{G}^j(\tilde{k})$的一一映射,这说明 $|\tilde{G}^1(\tilde{k})|=|\tilde{G}^j(\tilde{k})|$。

    对轨迹 $\xi\in\tilde{G}^j(\tilde{k})$,环 $(i)$,$(i,i+1)$,$(1,\cdots,N)$,$(1,N,\cdots,2)$ 分别形成 $k^i$,$k^{i,i+1}-l$,$l+1$,$l$ 次,其中 $l\le \min_{i\in S}k^{i,i+1}$ 依赖于 $\xi$。注意到 $\xi$ 在 $n$ 时到达状态 $N+j$ ,那么 $\xi$ 形成的最后一个环是 $(1,\cdots,N)$,因此
	\begin{equation*}
		\tilde{G}^j\left(\tilde{k}\right)=\bigsqcup_{l=0}^{\min_{i\in S}k^{i,i+1}}G^{j,+}\left(l+1,l,k^1,\cdots,k^N,k^{12}-l,\cdots,k^{N1}-l\right),
	\end{equation*}
	\begin{equation}\label{equality for AjBj}
		\left|\tilde{G}^j\left(\tilde{k}\right)\right|=\sum_{l=1}^{\min_{i\in S}k^{i,i+1}}\left|G^{j,+}\left(l+1,l,k^1,\cdots,k^N,k^{12}-l,\cdots,k^{N1}-l\right)\right|+B^j\left(\tilde{k}\right).
	\end{equation}
	下面用归纳法证明 $B^1(\tilde{k})=B^j(\tilde{k})$。当 $\min_{i\in S}k^{i,i+1}=0$, 通过方程 \eqref{equality for AjBj},可以得到 
    \begin{equation*}
		B^1(\tilde{k})=|\tilde{G}^1(\tilde{k})|=|\tilde{G}^j(\tilde{k})|=B^j(\tilde{k}).
	\end{equation*} 
	假设等式对 $\min_{i\in S}k^{i,i+1}\le m$ 成立。当 $\min_{i\in S}k^{i,i+1}=m+1$ 时,固定 $1\le l\le \min_{i\in S}k^{i,i+1}$,并且令 $\xi\in G^{j,+}_n(l+1,l,k^1,\cdots,k^N,k^{12}-l,\cdots,k^{N1}-l)$。那么轨迹 $\xi$ 可以被分为 $2l+1$ 个子轨迹, 是的每一个子轨迹最后形成的环是 $(1,\cdots,N)$ 或 $(1,N,\cdots,2)$。注意到最后一个环 $(1,\cdots,N)$ is $\binom{2l}{l}$ 形成前,在状态 1 处插入 $l$ 环 $(1,\cdots,N)$ 和 $l$ 个环 $(1,N,\cdots,2)$ 的排列数是 $\binom{2l}{l}$。固定剩余的环的分区(即 $\sum_{s=1}^{2l+1}\tilde{k}_s=(k^1,\cdots,k^N,k^{12}-l,\cdots,k^{N1}-l)$),因为对任意 $i$ 和 $\tilde{k}$,插入方式的排列数为 $\prod_{s=1}^{2l+1}B^j(\tilde{k}_s)$,那么:
	\begin{align*}
		&\;\left|G^{j,+}\left(l+1,l,k^1,\cdots,k^N,k^{12}-l,\cdots,k^{N1}-l\right)\right|\\
		=&\;\binom{2l}{l}\sum_{\sum_{s=1}^{2l+1}\tilde{k}_s=(k^1,\cdots,k^N,k^{12}-l,\cdots,k^{N1}-l)}\prod_{s=1}^{2l+1}B^j\left(\tilde{k}_s\right).
	\end{align*}
	
	因为 $l\ge 1$,,所以有 $B^1(\tilde{k}_s)=B^j(\tilde{k}_s)$,那么:
	\begin{align*}
		&\left|G^{1,+}\left(l+1,l,k^1,\cdots,k^N,k^{12}-l,\cdots,k^{N1}-l\right)\right|\\
		=&\left|G^{j,+}\left(l+1,l,k^1,\cdots,k^N,k^{12}-l,\cdots,k^{N1}-l\right)\right|.
	\end{align*}
	依据式 \eqref{equality for AjBj},可知$B^1(\tilde{k})=B^j(\tilde{k})$。最后通过归纳法,完成证明。
\end{proof}
\begin{proposition}\label{proposition:E Gin}
	考虑任意 $1\le i,j\le N$, $k\in\mathbb{N}^{2N+2}$ 和导出链 $\eta$,可得下面三条结论:
	\begin{itemize}
		\item[(i)] 
		\begin{equation}\label{EGin}
			\left|G^i\left(k^+,k^-,\tilde{k}\right)\right|=\left|G^i\left(k^-,k^+,\tilde{k}\right)\right|.
		\end{equation}
		\item[(ii)] 
		\begin{equation}\label{E Gi1cdotsNn}
			\left|G^{i,+}(k)\right|=\left|G^{j,+}(k)\right|,\qquad\left|G^{i,-}(k)\right|=\left|G^{j,-}(k)\right|.
		\end{equation}
		\item[(iii)]  
		\begin{equation}\label{EGetan}
			\left|G^{\eta}\left(k^+,k^-,\tilde{k}\right)\right|=\left|G^{\eta}\left(k^-,k^+,\tilde{k}\right)\right|.
		\end{equation}
	\end{itemize}
\end{proposition}
\begin{proof}
	(1) 首先考虑在状态 $i$ 插入环 $k^+$ 个环 $(1,\cdots,N)$ 和 $k^-$ 个环  $(1,N,\cdots,2)$,那么相应的排列数为 $\binom{k^++k^-}{k^+}$。接下来把剩余的环 (i.e. $k^i$ 个环 $(i)$ 和 $k^{i,i+1}$ 个环 $(i,i+1)$)分为 $k^++k^-+1$ 个部分。将剩余环的分区固定为 $\sum_{s=1}^{k^++k^-+1}\tilde{k}_s=\tilde{k}$。通过引理 \ref{lemma:BiBjCiCj},插入的数量为:
	\begin{equation*}
		\left[\prod_{s=1}^{k^++k^-}B^i(\tilde{k}_s)\right]\left|G^i(0,0,\tilde{k}_{k^++k^-+1})\right|,
	\end{equation*}
	其中 $B^i(\tilde{k}_s)$ 是在第 $(s-1)$ 个和 第 $s-$ 个 $N$ 状态环之间插入 $\tilde{k}_s$ 个环的排列数,且 $|G^i(0,0,\tilde{k}_{k^++k^-+1})|$ 是在第 $(k^++k^-)$ 个 $N$ 状态环之后插入 $\tilde{k}_{k^++k^-+1}$ 个环的排列数。在所有分区中增加插入数量 $\sum_{s=1}^{k^++k^-+1}\tilde{k}_s=\tilde{k}$,可以得到:
	\begin{align}\label{E Gin}
		\left|G^i\left(k^+,k^-,\tilde{k}\right)\right|=\binom{k^++k^-}{k^+}\sum_{\sum_{s=1}^{k^++k^-+1}\tilde{k}_s=\tilde{k}}\left[\prod_{s=1}^{k^++k^-}B^i\left(\tilde{k}_s\right)\right]\left|G^i\left(0,0,\tilde{k}_{k^++k^-+1}\right)\right|.
	\end{align}
	注意到 \eqref{E Gin} 是关于 $k^+$ 和 $k^-$ 对称的,故(1)证毕。
	
	(ii) 上面只证明了 \eqref{E Gi1cdotsNn} 中的第一个等式,第二个等式的证明与其相似。但是这里考虑的是轨迹 $\xi\in G^{i,+}_n(k)$,最后的环是 $(1,\cdots,N)$,只需在状态 $i$ 处插入 $k^+-1$ 个环 $(1,\cdots,N)$ 和 $k^-$ 个环 $(1,N,\cdots,2)$,可以得到:
	\begin{align*}
		&\left|G^{i,+}\left(k\right)\right|=\binom{k^++k^--1}{k^+-1}\sum_{\sum_{s=1}^{k^++k^-}\tilde{k}_s=\tilde{k}}\prod_{s=1}^{k^++k^-}B^i\left(\tilde{k}_s\right)\\
		&\qquad\qquad=\binom{k^++k^--1}{k^+-1}\sum_{\sum_{s=1}^{k^++k^-}\tilde{k}_s=\tilde{k}}\prod_{s=1}^{k^++k^-}B^j\left(\tilde{k}_s\right)=\left|G^{j,+}\left(k\right)\right|,
	\end{align*}
	其中第二步出自引理 ~\ref{lemma:BiBjCiCj}。

	(iii) 记 $\eta=[i_0,i_1,\cdots,i_t]$,首先把所有的环分为 $t+1$ 个部分,固定一部分为 $\sum_{s=0}^t k_s=k$,那么 $k^c_s$ 个环 $c$ ($\forall c\in\mathcal{C}$) 将被插入在 $\eta$ 的状态 $i_s$ ($0\le s\le t$)。那么可插入方式的数量为 $\prod_{s=0}^t |G^{i_s}(k_s)|$。增加所有分区的数量 $\sum_{s=0}^t k_s=k$,可以得到
	\begin{equation*}
		\left|G^{\eta}(k)\right|=\sum_{\sum_{s=0}^t k_s=k}\prod_{s=0}^t \left|G^{i_s}(k_s)\right|.
	\end{equation*}
	注意到对任意分区 $\sum_{s=0}^t k_s=k$,可知 $\sum_{s=0}^t (k_s^-,k_s^+,\tilde{k}_s)=(k^-,k^+,\tilde{k})$ 是 $(k^-,k^+,\tilde{k})$ 的一个分区。那么由方程 \eqref{EGin},可得:
	\begin{align*}
		\left|G^{\eta}(k^-,k^+,\tilde{k})\right|=&\sum_{\sum_{s=0}^t (k_s^-,k_s^+,\tilde{k}_s)=(k^-,k^+,\tilde{k})}\prod_{s=0}^t \left|G^{i_s}(k_s^-,k_s^+,\tilde{k}_s)\right|\\
		=&\sum_{\sum_{s=0}^t k_s=k}\prod_{s=0}^t \left|G^{i_s}(k_s)\right|=\left|G^{\eta}(k)\right|.
	\end{align*}
\end{proof}
下面将陈述 $| (G^i(k)|$ 和 $|G^{i,c}(k)|$ 的相关性。
\begin{lemma}\label{lemma:E G1cnk}
	对任意 $k\in \mathbb{N}^{2N+2}$ 和 $i\in S$,令 $c\in \mathcal{C}$ 包含状态 $i$ 的环,那么:
	\begin{align*}
		\left|G^{i,c}(k)\right| = \frac{k^c}{k_i}\left|G^i(k)\right|.
	\end{align*}
\end{lemma}
\begin{proof}
	下面只证明 $i=1$ 时的情况,并不降低一般性。 回顾章节~\ref{LELDP} 中计算 $|G^1(k)|$ 的三个步骤。
	在第一个步骤中固定最后一个环 $c$,并且把排列数转换为
	\begin{equation*}
		\frac{k^c}{k^{1}+k^{12}+k^{1N}+k^{+}+k^{-}}A_1.
	\end{equation*}
	第二,三个步骤不变。在这种情况下,轨迹形成的最后一个环是 $c$,这意味着:
	\begin{align*}
		\left|G^{1,c}(k)\right| = \frac{k^c}{k_1}\left|G^1(k)\right|.
	\end{align*}
\end{proof}
\begin{proposition}\label{corollary:rate function is unrelated to the starting state}
	$I_J$ 不依赖于初始分布的选择 $\xi$.
\end{proposition}
\begin{proof}
	记 $c=(1,\cdots,N)$ 或 $(1,N,\cdots,2)$,通过命题 ~\ref{proposition:E Gin} 和引理 ~\ref{lemma:E G1cnk},易知 $\log|G^1(k)|=\log|G^i(k)|+O(\log n)$,那么由式 \eqref{ratefunction},命题得证。
\end{proof}


\section{Proof of Theorem~\ref{thm:LDP}}\label{appendix:LDP}
这里把大偏差原理的证明 ~\ref{thm:LDP} 分为两部分。首先研究经验环流的速率函数的性质。回顾速率函数 $I_J:\mathcal{V}\to[0,\infty]$ 的形式为:
\begin{equation*}\label{ratefunction1}
	\begin{split}
		I_J(\nu) =&\; \left[h\left(\nu^{12}\right)+h\left(\nu^{1N}\right)
		+h\left(\nu^+\right)+h\left(\nu^-\right)-h\left(\nu^{12}+\nu^{1N}+\nu^++\nu^-\right)\right] \\
		&\;+\inf_{X\in V(\nu)}F_{\nu}(X)+\sum_{i\in S}\left[ h\left(\nu_i-\nu^i\right)+h\left(\nu^i\right)
		-h\left(\nu_i\right)\right]-\sum_{c\in\mathcal{C}}\nu^c\log\gamma^c\\
		:=&\;I_1(\nu)+I_2(\nu)+I_3(\nu)+I_4(\nu).
	\end{split}
\end{equation*}
\begin{proposition}\label{proposition:I}
	速率函数 $I_J$ 是有界,连续的凸函数。
\end{proposition}
\begin{proof}
	易知 $I_j$ 有界。首先证明 $I_J$ 连续。易知 $h$ 是定义在 $[0,\infty)$ 的连续函数。那么 $I_1$ 和 $I_3$ 是连续的。注意到  $I_4$ 是关于 $\nu$ 的连续函数,因此还需证明 $I_2$ 是连续函数。令 $Y(\nu)\in V(\nu)$ 是方程 \eqref{equations} 的解,由于 \eqref{equations} 是一个多项式方程组,因此易知 $Y(\nu)$ 是关于变量 $\nu$ 的连续函数。因为 $h$ 是连续的,再根据 \eqref{formula:F},$\inf_{X\in V(\nu)}F_{\nu}(X)=F_{\nu}(Y(\nu))$ 是关于 $\nu$ 的连续函数。
	
	然后证明 $I_J$ 是凸函数。 回顾 log-sum 不等式,对任意 $a_{1},a_{2},b_{1},b_{2}\ge 0$,有:
	\begin{equation}\label{log sum inequality}
		(a_1+a_2)\log \frac{a_1+a_2}{b_1+b_2}\le a_{1}\log \frac{a_{1}}{b_{1}}+a_{2}\log \frac{a_{2}}{b_{2}} ,
	\end{equation}
	注意到
	\begin{equation}\label{I_1}
		I_1(\nu) = \nu^{12}\log\left(\frac{\nu^{12}}{\hat{\nu}}\right)+\nu^{1N}\log\left(\frac{\nu^{1N}}{\hat{\nu}}\right)+\nu^{+}\log\left(\frac{\nu^{+}}{\hat{\nu}}\right)+\nu^{-}\log\left(\frac{\nu^{-}}{\hat{\nu}}\right),
	\end{equation}
	其中 $\hat{\nu}=\nu^{12}+\nu^{1N}+\nu^++\nu^-$。对任意满足 $\alpha+\beta=1$ 的 $\alpha,\beta\ge 0$ 和 $\nu,\mu\in \mathcal{V}$,通过log-sum 不等式 \eqref{log sum inequality},可以得到:
	\begin{equation*}
		(\alpha \nu^{12}+\beta \mu^{12})\log\left(\frac{\alpha \nu^{12}+\beta \mu^{12}}{\alpha \hat{\nu}+\beta \hat{\mu}}\right)\le \alpha\nu^{12}\log\left(\frac{\nu^{12}}{\hat{\nu}}\right)+\beta\mu^{12}\log\left(\frac{\mu^{12}}{\hat{\mu}}\right),
	\end{equation*}
	其中 $\hat{\mu}=\mu^{12}+\mu^{1N}+\mu^++\mu^-$。这说明 \eqref{I_1} 式的第一项的右边是关于 $\nu$ 的连续函数,同理,其他三项项的右边也是关于 $\nu$ 的连续函数。因此 $I_1(\nu)$ 是凸函数,同理,可以通过log-sum 不等式得到 $I_3(\nu)$也是凸函数。由于 $I_4(\nu)$ 是线性函数,故也是凸函数,最后只需证明 $\inf_{X\in V(\nu)}F_{\nu}(X)$ 是凸函数。 吧\eqref{formula:F} 重写为:
	\begin{align*}
		F_{\nu}(X)=&\;\sum_{i=2}^{N-1}\left[(x^{i-1}+\nu^+)\log\left(\frac{x^{i-1}+\nu^+}{x^{i-1}+x^i+\nu^+}\right)+x^i\log\left(\frac{x^{i}}{x^{i-1}+x^i+\nu^+}\right)\right]\\
		&\;+\sum_{i=2}^{N-1}\left[y^i\log\left(\frac{y^{i}}{y^{i}+y^{i+1}+\nu^-}\right)+(y^{i+1}+\nu^-)\log\left(\frac{y^{i+1}+\nu^-}{y^{i}+y^{i+1}+\nu^-}\right)\right]\\
		:=&\;\sum_{i=2}^{N-1}[A_1^i(\nu,X)+A_2^i(\nu,X)+A_3^i(\nu,X)+A_4^i(\nu,X)].
	\end{align*}
	注意到对任意 $X=(x^i,y^i)\in V(\nu)$ 和 $Z=(z^i,w^i)\in V(\mu)$,有 $\alpha X+\beta Z\in V(\alpha\nu+\beta\mu)$。那么通过 log-sum 不等式,可得:
	\begin{align*}
		&\;A_1^i(\alpha\nu+\beta \mu,\alpha X+\beta Y)\\
		=&\;(\alpha x^{i-1}+\beta z^{i-1}+\alpha\nu^++\beta \mu^+ )\log\left(\frac{\alpha x^{i-1}+\beta z^{i-1}+\alpha\nu^++\beta \mu^+}{\alpha x^{i-1}+\alpha x^i+\beta z^{i-1}+\beta z^i+\alpha\nu^++\beta \mu^+}\right)\\
		=&\;(\alpha (x^{i-1}+\nu^+)+\beta (z^{i-1}+ \mu^+) )\log\left(\frac{\alpha (x^{i-1}+\nu^+)+\beta (z^{i-1}+ \mu^+)}{\alpha (x^{i-1}+ x^i+\nu^+)+\beta (z^{i-1}+ z^i+ \mu^+)}\right)\\
		\le &\; \alpha (x^{i-1}+\nu^+)\log\left(\frac{x^{i-1}+\nu^+}{x^{i-1}+ x^i+\nu^+}\right)+\beta (z^{i-1}+ \mu^+)\log\left(\frac{z^{i-1}+ \mu^+}{z^{i-1}+ z^i+ \mu^+}\right)\\
		=&\;\alpha A^i_1(\nu,X)+\beta A^i_1(\mu,Y).
	\end{align*}
	同理, 对任意 $j=2,3,4$ 有 $A^i_j(\alpha \nu+\beta \mu,\alpha X+\beta Y)\le \alpha A^i_j(\nu,X)+\beta A^i_j (\mu,Y)$。这说明
	\begin{equation*}
		F_{\alpha \nu+\beta \mu}(\alpha X+\beta Y)\le \alpha F_{\nu}(X)+\beta F_{\mu}(Y).
	\end{equation*}
	针对 $X$ 和 $Y$ 取极小,可以得到:
	\begin{align*}
		\inf_{Z\in V(\alpha\nu+\beta\mu)}F_{\alpha\nu+\beta\mu}( Z)\le\alpha \inf_{X\in V(\nu)} F_{\nu}(X)+\beta \inf_{Y\in V(\mu)} F_{\mu}(Y).
	\end{align*}
	证毕。
\end{proof}
下面给出 LDP 原理的严格证明。
\begin{proposition}\label{theorem:LDP}
	经验环流 $(J^c_n)_{c\in\mathcal{C}}$ 满足速率为 $n$ 的大偏差原理,并且相应的速率函数为 $I_J:\mathcal{V}\to [0,\infty]$。此外,它的上界可以提升为:对任意集合 $\Gamma \subset \mathcal{V}$,有:
	\begin{align}\label{upper bound}
		\varlimsup_{n\to+\infty}\;
		\frac 1n \log \mathbb  P \left( (J^c_n)_{c\in \mathcal{C}} \in \Gamma \right)
		\le -\inf_{\nu\in \Gamma} I_J(\nu).
	\end{align}
\end{proposition}
\begin{proof}
记
\begin{align}\label{set:K_n}
K_n := \left\{(k^c)_{c\in \mathcal{C}}\in \mathbb{N}^{2N+2}: \sum_{c \in \mathcal{C}} k^{c} |c| =n \right\}.
\end{align}
接下来的证明将假设马氏链从状态 1 出发,这并不会降低命题的一般性。那么,对任意 $k=(k^c)_{c\in\mathcal{C}}\in K_n$,由 \eqref{joint} 式,可得:
\begin{align}\label{Jnxi}
\mathbb{P}_1\left(J^c_{n}= \frac{k^c}{n},\ \forall c\in \mathcal{C}\right)
= |G_n(k)| \prod_{c\in\mathcal{C}}\left(\gamma^c\right)^{k^c},
\end{align}
令 $\mu_n(k) = k/n\in\mathcal{V}$。对任意 $\Gamma\subset \mathcal{V}$,令
\begin{align*}
Q_n(\Gamma) = \max_{k\in K_n: \mu_n(k) \in \Gamma}
\mathbb{P}_1\left(J^c_{n} = \frac{k^c}{n},\ \forall c\in\mathcal{C}\right).
\end{align*}
显然有:
\begin{align}\label{Qn gamma}
	Q_n(\Gamma)
	\le \mathbb{P}_1\left(J_{n} \in \Gamma\right)
	\le |K_n| Q_n(\Gamma).
\end{align}
易知 $|K_n| \le (2N+2)(n+1)^{2N+3}$,那么有公式 \eqref{Jnxi},\eqref{Qn gamma},\eqref{log A1},\eqref{log A3},和\eqref{log A2}
\begin{equation}\label{1 n log P}
	\begin{split}
	\frac{1}{n}\log\mathbb{P}_1\left(J_{n} \in \Gamma\right)=&O\left(\frac{\log n}{n}\right)+\frac{1}{n}\log Q_n(\Gamma)\\
	=&O\left(\frac{\log n}{n}\right)+\max_{k\in K_n: \mu_n(k) \in \Gamma}\left[\frac{1}{n}\log |G_{n}(k)|+\sum_{c \in \mathcal{C}}\frac{k_c}{n}\log \gamma^c\right]\\
	=&O\left(\frac{\log n}{n}\right)-\min_{k\in K_n: \mu_n(k) \in \Gamma}I_J\left(\mu_{n}(k)\right).
	\end{split}
\end{equation}
由于 $\cup_{n\in\mathbb{N}} \{\mu_{n}(k):k\in K_n\}$ 在 $\mathcal{V}$ 中稠密,且 $\nu\to I_J(\nu)$ 在 $\mathcal{V}$ 中连续(见命题 \ref{proposition:I}),这保证了对每个 $\nu\in \mathcal{V}$,存在序列 $(k_n)_{n\in \mathbb{N}}$,使得:
\begin{equation*}
\lim_{n\to\infty} \|\mu_{n}(k_n)-\nu\| = 0, \qquad \lim_{n\to\infty} I_J\left(\mu_{n}(k_n)\right) = I_J(\nu).
\end{equation*}
那么对任意开集 $U\subset \mathcal{V}$
\begin{equation*}
\varlimsup_{n\to\infty}\min_{k\in K_n: \mu_n(k) \in U}I_J\left(\mu_{n}(k)\right)\le I_J(\nu),\quad\forall\nu\in U.
\end{equation*}
对 $\nu\in U$ 取极小,可得:
\begin{equation}\label{varlimsup}
\varlimsup_{n\to\infty}\min_{k\in K_n: \mu_n(k) \in U}I_J\left(\mu_{n}(k)\right)\le \inf_{\nu\in U}I_J(\nu).
\end{equation}
结合 \eqref{1 n log P} 和 \eqref{varlimsup},可得大偏差原理的下界 \eqref{def1}。此外,对任意$\Gamma \subset \mathcal{V}$,同理可得反向的不等式,即
\begin{equation}\label{varliminf}
\varliminf_{n\to\infty}\min_{k\in K_n: \mu_n(k) \in \Gamma}I_J\left(\mu_{n}(k)\right)\ge \inf_{\nu\in \Gamma}I_J(\nu).
\end{equation}
结合 \eqref{1 n log P} 和 \eqref{varliminf},可得大偏差原理的上界 \eqref{def2}。
\end{proof}


\section{简化速率函数 $I_J$}\label{appendix:threestate}
这里将对两种特殊情况,给出速率函数 $I_J$ 的简化形式,1)三状态马氏链。2)状态 1 到状态 N 的转移概率为 0 的单环马氏系统。回顾速率函数 $I_J:\mathcal{V}\to[0,\infty]$ 的公式为:
\begin{equation*}\label{ratefunction1}
	\begin{split}
		I_J(\nu) =&\; \left[h\left(\nu^{12}\right)+h\left(\nu^{1N}\right)
		+h\left(\nu^+\right)+h\left(\nu^-\right)-h\left(\nu^{12}+\nu^{1N}+\nu^++\nu^-\right)\right] \\
		&\;+\inf_{X\in V(\nu)}F_{\nu}(X)+\sum_{i\in S}\left[ h\left(\nu_i-\nu^i\right)+h\left(\nu^i\right)
		-h\left(\nu_i\right)\right]-\sum_{c\in\mathcal{C}}\nu^c\log\gamma^c\\
		:=&\;I_1(\nu)+I_2(\nu)+I_3(\nu)+I_4(\nu).
	\end{split}
\end{equation*}
\begin{proposition}
三状态马氏链的速率函数为
\begin{align*}
I_J(\nu) =
\sum_{i\in S} \left[\nu^{i}\log \left(\frac{\nu^{i}/\nu_i}{J^i/J_i}\right) + (\nu_i - \nu^i)\log \left(\frac{(\nu_i - \nu^i)/\nu_i}{(J_i - J^i)/J_i} \right)
\right]
+ \sum_{c \in \mathcal{C}, |c|\neq 1} \nu^{c} \log \left(\frac{\nu^{c}/\tilde{\nu}}{J^c/\tilde{J}}\right).
\end{align*}
\end{proposition}
\begin{proof}
	易知 $X=(x^2,y^2)$ 为方程 \eqref{equations} 的解,其中
\begin{equation*}
	x^{2}=\frac{\nu^{23}\left(\nu^{12}+\nu^+\right)}{\nu^{12}+\nu^{13}+\nu^++\nu^-},\quad y^{2}=\frac{\nu^{23}\left(\nu^{13}+\nu^-\right)}{\nu^{12}+\nu^{13}+\nu^++\nu^-}.
\end{equation*}
根据引理 ~\ref{lemma:mininum}
\begin{equation*}\label{Fnu2}
I_2(\nu)=F_{\nu}(X)=\nu^{23}\log\frac{\nu^{23}}{\tilde{\nu}}+\left(\nu^{12}+\nu^{13}+\nu^++\nu^-\right)\log\frac{\tilde{\nu}-\nu^{23}}{\tilde{\nu}}.
\end{equation*}
那么通过计算,可以得到:
\begin{equation}\label{I1I2I3}
	I_1(\nu)+I_2(\nu)+I_3(\nu) = \sum_{i\in S} \left[\nu^{i}\log \frac{\nu^{i}}{\nu_i} + (\nu_i - \nu^i)\log \frac{\nu_i - \nu^i}{\nu_i} 
	\right]
	+ \sum_{c \in \mathcal{C}, |c|\neq 1} \nu^{c} \log \frac{\nu^{c}}{\tilde{\nu}}.
\end{equation}
回顾环流的表达 \cite[Theorem.1.3.3]{jiang2004mathematical},有
\begin{equation}\label{J+J-Jii+1}
J^+=\gamma^+\frac{1}{C},\quad J^-=\gamma^-\frac{1}{C},\quad J^{i,i+1}=\gamma^{i,i+1}\frac{1-p_{i-1,i-1}}{C}, \quad 1\le i\le 3,
\end{equation}
其中ere $C=3+\sum_{i\in S}[p_{ii}p_{i+1,i+1}-p_{i,i+1}p_{i+1,i}-2(p_{ii}+p_{i+1,i+1})]$. 根据 \eqref{decomposition},
\begin{equation}\label{pij}
	p_{ij}=\frac{\sum_{c \ni \langle i,j\rangle}J^c}{\sum_{c\ni i}J^c}.
\end{equation}
结合 \eqref{J+J-Jii+1} 和 \eqref{pij},有:
\begin{equation}\label{nucgamma2}
	\begin{split}
		&\;\sum_{i \in S}\left[\nu^i\log\frac{J^i}{J_i}+\left(\nu_i-\nu^i\right)\log\frac{J_i-J^i}{J_i}\right]+ \sum_{c \in \mathcal{C}, |c|\neq 1} \nu^{c} \log \left(\frac{J^c}{\tilde{J}}\right)\\
		=&\;\sum_{i \in S}\left[\nu^i\log\frac{J^i}{J_i}+\nu^{i,i+1}\log\left(\left(1-\frac{J^i}{J_i}\right)\left(1-\frac{J^{i+1}}{J_{i+1}}\right)\frac{J^{i,i+1}}{\tilde{J}}\right)\right]\\
		&\;+\nu^+\log\left(\frac{J^+}{\tilde{J}}\prod_{i\in\mathcal{C}}\left(1-\frac{J^i}{J_i}\right)\right)+\nu^-\log\left(\frac{J^-}{\tilde{J}}\prod_{i\in\mathcal{C}}\left(1-\frac{J^i}{J_i}\right)\right)\\
		=&\;\sum_{c \in \mathcal{C}}\nu^c \log\gamma^c=-I_4(\nu).
	\end{split}
\end{equation}
结合 \eqref{I1I2I3} 和 \eqref{nucgamma2},得证。
\end{proof}
\begin{proposition}
	状态 1 到状态 N 的转移概率为 0 的单环马氏链的速率函数为:
	\begin{equation*}
		\begin{split}
			I_J(\nu) = \sum_{i\in S}\left[\nu^i\log\left(\frac{\nu^i/\nu_i}{J^i/J_i}\right)
			+\nu^{i,i+1}\log\left(\frac{\nu^{i,i+1}/\nu_i}{J^{i,i+1}/J_i}\right)+\left(\nu^{i-1,i}+\nu^+\right)\log\left(\frac{\left(\nu^{i-1,i}+\nu^+\right)/\nu_i}
			{\left(J^{i-1,i}+J^+\right)/J_i}\right)\right].
		\end{split}
	\end{equation*}
\end{proposition}
\begin{proof}
	在 $p_{1N}=0$ 的条件下,环 $(1,N)$ 和环 $(1,N,\cdots,2)$ 不会被形成。因此,可以得到\eqref{equation} 中的 $\nu^{1N}=\nu^{-}=0$ ,并且 $x^{i}=\nu^{i,i+1}$, $y^{i}=0$ 是方程 \eqref{equation}的解,那么
	\begin{equation*}
		I_2(\nu)=F_{\nu}(x^i,y^i)=\sum_{i=2}^{N-1}\left[-\lambda_{i}\nu^{i,i+1}+\nu^+\log\frac{\nu^{i-1,i}+\nu^+}{\nu^{i-1,i}+\nu^{i,i+1}+\nu^+}\right]+\nu^{12}\log\frac{\nu^{12}+\nu^+}{\nu^{12}+\nu^{23}+\nu^+},
	\end{equation*}
	其中 
	\begin{equation*}
		\lambda_i=-\log\left(\frac{\nu^{i,i+1}}{\nu^{i-1,i}+\nu^{i,i+1}+\nu^+}\,\frac{\nu^{i,i+1}+\nu^+}{\nu^{i,i+1}+\nu^{i+1,i+2}+\nu^+}\right).
	\end{equation*}
	通过 $\nu_i$ 的定义,有
	\begin{equation*}
		\nu_1=\nu^{1}+\nu^{12}+\nu^+,\quad \nu_i=\nu^i+\nu^{i-1,i}+\nu^{i,i+1}+\nu^+,\ 2\le i\le N-1,\quad \nu_N=\nu^N+\nu^{N-1,N}+\nu^+.
	\end{equation*} 
然后通过计算,可以得到:	
\begin{equation}\label{I1lack}
	I_1(\nu)=\nu^{12}\log\frac{\nu^{12}}{\nu_1-\nu^{12}}+\nu^+\log\frac{\nu^+}{\nu_1-\nu^1},
\end{equation}
	\begin{align}\label{I2lack}
	I_2(\nu)
	=\sum_{i=2}^{N}\left[\nu^{i,i+1}\log\frac{\nu^{i,i+1}}{\nu_i-\nu^i}+\nu^+\log\frac{\nu^{i-1,i}+\nu^+}{\nu_i-\nu^i}\right]+\sum_{i=1}^{N}\nu^{i,i+1}\log\frac{\nu^{i,i+1}+\nu^+}{\nu_{i+1}-\nu^{i+1}},
   \end{align}
和
	\begin{equation}\label{I3lack}
		\begin{split}
			I_3(\nu)=\sum_{i\in S}\left[\nu^i\log\frac{\nu^i}{\nu_i}+\nu^+\log\frac{\nu_i-\nu^i}{\nu_i}\right]+\sum_{i\in S}\nu^{i,i+1}\left(\log\frac{\nu_i-\nu^i}{\nu_i}
			+\log\frac{\nu_{i+1}-\nu^{i+1}}{\nu_{i+1}}\right).
		\end{split}
	\end{equation}
    根据 \eqref{pij},可得
	\begin{equation}\label{nucgammac}
		\begin{split}
			&\;\sum_{i \in S}\left[\nu^i\log\frac{J^i}{J_i}+\nu^{i,i+1}\log\frac{J^{i,i+1}}{J_i}+(\nu^{i-1,i}+\nu^+)\log\frac{J^{i-1,i}+J^+}{J_i}\right]\\
			=&\;\sum_{i \in S}\left[\nu^i\log\frac{J^i}{J_i}+\nu^{i,i+1}\log\frac{J^{i,i+1}(J^{i,i+1}+J^+)}{J_{i+1}J_i}\right]+\nu^+\log\frac{\prod_{i=1}^N\left(J^{i,i+1}+J^+\right)}{\prod_{i=1}^N J_i}\\
			=&\;\sum_{c \in \mathcal{C}}\nu^c \log\gamma^c=-I_4(\nu).
		\end{split}
	\end{equation}
	结合 \eqref{I1lack},\eqref{I2lack},\eqref{I3lack},和 \eqref{nucgammac},证毕。
\end{proof}

\section{经验 LE 环流的暂态涨落定理的证明}
在此,将给出无周期条件下,经验 LE 环流的暂态涨落定理的证明。
记 $k=(k^c)_{c\in\mathcal{C}}\in \mathbb{N}^{2N+2}$。在时间步 $n$,轨迹中环 $c$ 形成 $k^c$ 次,因此有:
\begin{equation*}
	n=\sum_{c\in\mathcal{C}}k^c|c|+|\eta|,
\end{equation*}
其中 $\eta$ 表示剩下的导出链,且 $0\le |\eta|\le N-1$。假设系统从状态 1 出发,这并不降低命题的一般性。令 $[\eta]=t$,那么 $\eta=\eta_1$ 或者 $\eta_2$,其中 $\eta_1=[1,2,\cdots,t+1]$ 且 $\eta_2=[1,N,\cdots,N+1-t]$。同时, \eqref{joint} 可以被写为:
\begin{equation}\label{joint2}
	\begin{split}
		\mathbb{P}\left(N^c_n=k^c,\;\forall c\in\mathcal{C}\right)
		=&\;\mathbb{P}\left(N^c_n=k^c,\;\forall c\in\mathcal{C},\tilde{\xi}_n=\eta_1\right)+\mathbb{P}\left(N^c_n=k^c,\;\forall c\in\mathcal{C},\tilde{\xi}_n=\eta_2\right)\\
		=&\;\left|G^{\eta_1}(k)\right|\left[\prod_{c\in\mathcal{C}}\left(\gamma^c\right)^{k^c}\right]\gamma^{\eta_1}+\left|G^{\eta_2}(k)\right|\left[\prod_{c\in\mathcal{C}}\left(\gamma^c\right)^{k^c}\right]\gamma^{\eta_2},
	\end{split}
\end{equation}
其中 $\gamma^{\eta_1}=p_{12}\cdots p_{t,t+1}$ 且 $\gamma^{\eta_2}=p_{1N}\cdots p_{N+2-t,N+1-t}$。那么有
\begin{align*}
	\mathbb{P}\left(N^+_n=k^+,N^-_n=k^-,\cdots\right)
	&= (\gamma^+)^{k^+}(\gamma^-)^{k^-}\prod_{c\neq C^+,C^-}\left(\gamma^c\right)^{k^c}\\
	&\qquad\left[|G^{\eta_1}(k^+,k^-,\cdots)|\gamma^{\eta_1}+|G^{\eta_2}(k^+,k^-,\cdots)|\gamma^{\eta_2}\right].
\end{align*}
同理,如果交换上述方程中的 $k^+$ 和 $k^-$,可以得到:
\begin{align*}
	\mathbb{P}\left(N^+_n=k^-,N^-_n=k^+,\cdots\right)
	&= (\gamma^+)^{k^-}(\gamma^-)^{k^+}\prod_{c\neq C^+,C^-}\left(\gamma^c\right)^{k^c}\\
	&\qquad\left[|G^{\eta_1}(k^-,k^+,\cdots)|\gamma^{\eta_1}+|G^{\eta_2}(k^-,k^+,\cdots)|\gamma^{\eta_2}\right].
\end{align*}
根据 \eqref{EGetan} 式,可以得到下列的暂态涨落定理成立:
\begin{equation*}
	\mathbb{P}\left(N^+_n=k^+,N^-_n=k^-,\cdots\right)=\mathbb{P}\left(N^+_n=k^-,N^-_n=k^+,\cdots\right)\left(\frac{\gamma^+}{\gamma^-}\right)^{k^+-k^-}.
\end{equation*}
