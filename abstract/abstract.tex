% !Mode:: "TeX:UTF-8"

\BiAppendixChapter{摘\quad 要}{Abstract (In Chinese)}
\setcounter{page}{1}
\song\defaultfont

环流是随机热力学中的重要物理量。马氏系统中的环流和净环流可以由环擦除(LE)或生成树(ST)方式定义。本文从大偏差理论和涨落定理的角度,对这两种定义模式进行比较性的研究。首先推导出环拓扑结构系统的LE环流联合分布和大偏差率函数,同时也得到一般系统的ST环流的速率函数。进一步阐明了LE和ST环流的速率函数之间的关系。此外,还研究了LE和ST环流满足的各种类型的涨落定理,并说明了它们的适用范围。本文表明,LE环流和LE净环流皆满足各种强形式下的涨落定理。相比之下,ST环流不满足涨落定理的,ST净环流也仅满足弱形式涨落定理。

\vspace{\baselineskip}

\hangafter=1\hangindent=52.3pt\noindent
{\bfseries\xiaosi\song 关键词:环流;大偏差;涨落定理}
\clearpage

\BiAppendixChapter{Abstract}{Abstract (In English)}

The cycle current is a crucial quantity in stochastic thermodynamics. The absolute and net cycle currents of a Markovian system can be defined in the loop-erased (LE) or the spanning tree (ST) manner. Here we make a comparative study between the large deviations and fluctuation theorems for the LE and ST currents, i.e. cycle currents defined in the LE and ST manners. First, we derive the exact joint distribution and the large deviation rate function for the LE currents of a system with a cyclic topology and also obtain the rate function for the ST currents of a general system. The relationship between the rate functions for the LE and ST currents is clarified. Furthermore, we examine various types of fluctuation theorems satisfied by the LE and ST currents and clarify their ranges of applicability. We show that both the absolute and net LE currents satisfy the strong form of all types of fluctuation theorems. In contrast, the absolute ST currents do not satisfy fluctuation theorems, while the net ST currents only satisfy the weak form of fluctuation theorems.

\vspace{\baselineskip}

\hangafter=1\hangindent=60pt\noindent
{\textbf{\xiaosi Keywords: cycle currents, large deviations, fluctuation theorems}}
\clearpage
