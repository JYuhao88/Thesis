% !Mode:: "TeX:UTF-8"

\BiAppendixChapter{摘\quad 要}{Abstract (In Chinese)}
\setcounter{page}{1}
\song\defaultfont

环流理论一直是随机热力学理论的关键内容。马氏系统中的环流和净环流可以由两种方式定义,即环擦除和生成树。该研究针对两种研究下的大偏差理论和涨落定理进行比较研究。首先研究环拓扑结构系统,并计算出环擦除定义下大偏差速率函数的隐式表达式,然后研究生成树定义下的大偏差速率函数,并阐明两者之间的关系。进一步,检验两个定义下的结论和涨落定理的相容性。经过严格对照,生成树定义下的环流不满足涨落定理,然而生成树的结果满足涨落定理的弱形式。

\vspace{\baselineskip}

\hangafter=1\hangindent=52.3pt\noindent
{\bfseries\xiaosi\song 关键词:环流;大偏差;涨落定理}
\clearpage

\BiAppendixChapter{Abstract}{Abstract (In English)}

The cycle current is a crucial quantity in stochastic thermodynamics. The absolute and net cycle currents of a Markovian system can be defined in the loop-erased (LE) or the spanning tree (ST) manner. Here we make a comparative study between the large deviations and fluctuation theorems for the LE and ST currents, i.e. cycle currents defined in the LE and ST manners. First, we give the explicit expression of the large deviation rate functions for the LE currents of a system with a cyclic topology and for the ST currents of a general system. The relationship between the rate functions for the LE and ST currents are clarified. Furthermore, we examine various types of fluctuation theorems satisfied by the LE and ST currents. We show that both the absolute and net LE currents satisfy the strong form of all types of fluctuation theorems. In contrast, the absolute ST currents do not satisfy fluctuation theorems, while the net ST currents only satisfy the weak form of fluctuation theorems.

\vspace{\baselineskip}

\hangafter=1\hangindent=60pt\noindent
{\textbf{\xiaosi Keywords: cycle currents, large deviations, fluctuation theorems}}
\clearpage
