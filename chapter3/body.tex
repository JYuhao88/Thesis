\BiChapter{环流的大偏差原理}{Large deviations of cycle currents}
大偏差原理关系到随机过程中小概率事件的长时涨落行为 \cite{varadhan1984large,den2000large}。本文在此考察了经验环流的大偏差。既然假设周期边界条件,单环马氏链的经验LE环流$(J_n^c)_{c \in \mathcal{C}}$也要定义在空间:
\begin{equation*}
    \mathcal{V} = \left\{(\nu^c)_{c\in \mathcal{C}}:\;\nu^c\geq 0,\;
    \sum_{c\in \mathcal{C}}|c|\nu^c  = 1\right\},
\end{equation*}
其中$|c|$表示环$c$的长度,即环$c$中状态数量。若满足下列三个条件\cite{varadhan1984large},则称经验LE环流$(J_n^c)_{c \in \mathcal{C}}$满足速率函数为$I_J:\mathcal{V}\rightarrow [0,\infty]$的大偏差原理:
\begin{itemize}
    \item 对于$\forall \alpha \geqslant 0$,水平集$\{x \in \mathcal{V}: I_{J}(x) \leqslant \alpha\}$是紧的。
    \item 对于任意开集$U \subset \mathcal{V}$,
        \begin{equation}\label{def1}
            \varliminf_{n\to\infty}\frac{1}{n}\log\mathbb{P}((J^c_n)_{c\in\mathcal{C}}\in U)\ge-\inf_{x\in U}I_J(x).
        \end{equation}
    \item 对于任意闭集$F \subset \mathcal{V}$,
        \begin{equation}\label{def2}
            \varlimsup_{n\to \infty}\frac{1}{n}\log\mathbb{P}((J^c_n)_{c\in\mathcal{C}}\in F)\le -\inf_{x\in F}I_J(x).
        \end{equation}
\end{itemize}
从中可以看出,定义中的条件$(ii)$和$(iii)$表明对$\forall (\nu^c)_{c\in\mathcal{C}}\in\mathcal{V}$,满足:
\begin{equation}\label{LDP}
    \mathbb{P}(J^c_n=\nu^c,\;\forall c\in\mathcal{C})\propto e^{-n I_J(\nu)},\;\;\;n\to\infty,
\end{equation}
同样地,也可以定义经验ST环流的大偏差$(Q_c^{c_l})_{l \subset \mathcal{L}}$。

\BiSection{单环马氏链LE环流的大偏差}
首先讨论经验LE环流的大偏差原理。一般的马氏链中,速率函数的解析表达式$I_J$很难求出,因此下面将研究单环马氏链。单环系统中所有可能形成的环都已在 (\ref{cycle_space})中列出。

下面假设系统都从状态$1$出发,这样会简化叙述步骤,且不会降低命题的适用性。为了求出速率函数$I_J$,需要计算LE经验环流的联合分布。对任意环$c=(i_1, i_2, \cdots, i_s)$,令$\gamma^c = p_{i_1i_2}p_{i_2i_3}\cdots p_{i_si_1}$表示沿该环所有转移概率的乘积。对任意满足$\sum_{c \subset \mathcal{C}} |c| k^c=n$的非负整数序列$k=(k^c)_{c\subset \mathcal{C}}$,由于周期边界条件,经验LE环流的联合分布为:
\begin{equation}\label{joint}
    \begin{split}
    \mathbb{P}\left(J^c_n=\nu^c,\;\forall c\in\mathcal{C}\right)
    =&\;\mathbb{P}\left(N^c_n=k^c,\;\forall c\in\mathcal{C}\right)\\
    =&\;|G_n(k)|\prod_{c\in\mathcal{C}}\left(\gamma^c\right)^{k^c},
    \end{split}
\end{equation}
其中$\nu^c = k^c/n$,$G_n(k)$表示$n$时刻可能形成的轨道的集合,且称该类轨道为容许轨道。

为书写方便,若$c=(i)$是一状态环,则用$k^i$替代$k^c$;若$c=(i,i+1)$是两状态环,则用$k^{i,i+1}$表示;若$c=(1,2,\cdots,N)$是顺时针$N$状态环,则用$k^{+}$表示;若$c=(1,N,\cdots,2)$是逆时针$N$状态环,则用$k^{-}$表示。类似地,也用$\nu^i, \nu^{i,i+1}, \nu^+, \nu^-$表示相应的经验环流,用$J^i, J^{i, i+1}, J^{+}, J^{-}$表示相应的经验净环流。例如,对于三状态马氏链,如果序列$k=(k^c)_{c\subset \mathcal{C}}$为:
\begin{equation} \label{trajectory_ex}
    k^3 = k^{12} = k^{23} = k^- = 1, ~~~ k^1 = k^2 = k^{13} = k^+= 0
\end{equation}
那么在时刻$n=8$,有8个容许轨道,如表\ref{table:all possible trajectories}所示。

\begin{table}[htb!]
    \renewcommand\arraystretch{1.2}
    \begin{tabular}{cccccccccc}
    \hline
   $m$   & 0 & 1 & 2 & 3 & 4 & 5 & 6 & 7 & 8 \\\hline
   $\xi_m$& 1 & 3 & 3 & 2 & 3 & 2 & 1 & 2 & 1 \\\hline
   $\xi_m$& 1 & 3 & 2 & 3 & 3 & 2 & 1 & 2 & 1 \\\hline
   $\xi_m$& 1 & 3 & 3 & 2 & 1 & 2 & 3 & 2 & 1 \\\hline
   $\xi_m$& 1 & 3 & 2 & 1 & 2 & 3 & 3 & 2 & 1 \\\hline
   $\xi_m$& 1 & 2 & 3 & 3 & 2 & 1 & 3 & 2 & 1 \\\hline
   $\xi_m$& 1 & 2 & 3 & 2 & 1 & 3 & 3 & 2 & 1 \\\hline
   $\xi_m$& 1 & 2 & 1 & 3 & 3 & 2 & 3 & 2 & 1 \\\hline
   $\xi_m$& 1 & 2 & 1 & 3 & 2 & 3 & 3 & 2 & 1 \\\hline
    \end{tabular}\centering
    \caption{三状态马氏链中,8个容许轨道,环$(3), (12), (23)$和$(1,3,2)$形成一次,环$(1), (2), (13)$和$(1,2,3)$没有形成过}
    \label{table:all possible trajectories}
\end{table}

接下来计算容许轨道的数量$G_n(k)$。基本的思路是在近似有序的轨道中插入各种环,插入方式的数量将会是容许轨道的数量。计算过程分为三个步骤:

1)由于系统从状态$1$出发,作为第一步,选出所有包含初始状态$1$的环,即$(1)$,$(1,2)$,$(1,N)$,$(1,2,\cdots,N)$,$(1,N,\cdots,2)$,并且插入到轨道中。因为环$c$形成$k^c$次,所以步骤1)所有可能的插入方式的数量,也就是环的排列数为:
\begin{equation*}\label{formula:A1}
    A_1 = \binom{k^1+k^{12}+k^{1N}+k^{+}+k^{-}}{k^1,k^{12},k^{1N},k^{+},k^{-}}
    := \frac{(k^1+k^{12}+k^{1N}+k^{+}+k^{-})!}{k^1!\;k^{12}!\;k^{1N}!\;k^{+}!\;k^{-}!}.
\end{equation*}
对于 \ref{trajectory_ex}中的例子,步骤1)中所有可能的插入方式为图 \ref{figure:insertion}左部分所示。
\begin{figure}[htb!]
\centering
\includegraphics[scale=0.6]{chart/insertiongraph.pdf}
\caption{构建所有容允许轨道的环插入法示意图。依然使用\ref{trajectory_ex}中的例子。环插入法分为三步:首先我们将所有包含初始状态的环插入轨道,接下来我们将所有剩余的两元环插入轨道,最后我们将所有剩余的一元环插入轨道。经过这三步的环插入,找到了所有八个容许轨道,这与表中列出的轨道完全吻合。}
\label{figure:insertion}
\end{figure}

2)在轨道中插入剩余的二元环。仔细观察系统形成环$(i,i+1)$的情况,可能是在状态$i$,也可能是$i+1$。例如,轨道$\{1, 3, 2, 3, \cdots\}$,形成环$(2,3)$时,导出链是$[1, 3]$。对此,称该环是在状态$3$处形成。作为对比,轨道$\{1,2,3,2, \cdots\}$,形成环$(2,3)$时,导出链是$[1, 2]$,因此称该环是在状态$2$处形成的。

考虑两状态环$(i, i+1), 2\leqslant i \leqslant N-1$,记$l^i$和$m^i$分别表示在状态$i$和$i+1$处形成环$(i, i+1)$的数量,显然$l^i + m^i = k^{i, i+1}$。固定$l^i$和$m^i$的值,相应的容许轨道数量可以计算得到。首先,在状态$2$处插入$l^2$个环$(2, 3)$,总共有$k^+ + k^{12}$个位置可以插入,然而环$(1,2, \cdots, N)$和环$(1, 2)$已经在步骤1)中考虑。既然这些位置没有包含环$(1, N, \cdots, 2)$中的状态$2$,如果插入$(2,3)$,那么这个环将会在状态$3$处形成,而不是状态$2$。因此可能的插入方式数量是:
\begin{equation} \label{binom1}
    \binom{k^++k^{12}+l^{2}-1}{l^{2}}.
\end{equation}
接下来,考虑状态$3 \leqslant i \leqslant N$对应的$l^i$个环,即$(i, i+1)$。把它们插入轨道中,其中每个状态$i$有$k^+ + l^{i-1}$个可能的位置插入,也就是环$(1, 2, \cdots, N)$和环$(i-1, i)$中的位置。因此总共插入方式数量是:
\begin{equation} \label{binom2}
    \binom{k^++l^{i-1}+l^{i}-1}{l^{i}},\;\;\;3\le i\le N-1.
\end{equation}
目前,已经插入$l^i$个环$(i,i+1)$。结合 \ref{binom1}和 \ref{binom2},所有可能的插入方式的数量:
\begin{equation*}
    \prod_{i=2}^{N-1}\binom{l^{i}+l^{i-1}+k^{+}-1}{l^{i}}
\end{equation*}
其中$l^1:=k^{12}$。

同理,把状态$2 \leqslant i \leqslant N-1$对应的$m^i$个环$(i, i+1)$插入轨道,所有可能的插入方式的数量:
\begin{equation*}
    \prod_{i=2}^{N-1}\binom{m^{i}+m^{i+1}+k^{-}-1}{m^{i}},
\end{equation*}
其中$m^N:=k^{N1}$。所有两状态的环就已经完全被插入。步骤 2)中所有可能的插入方式数量为:
\begin{equation*}
    A_2 = \sum_{l^{2}+m^{2}=k^{23}}\dots\sum_{l^{N-1}+m^{N-1}=k^{N-1,N}}
    \prod_{i=2}^{N-1}\binom{l^{i}+l^{i-1}+k^{+}-1}{l^{i}}\prod_{i=2}^{N-1}\binom{m^{i}+m^{i+1}+k^{-}-1}{m^{i}},
\end{equation*}
例子 \ref{trajectory_ex}中,步骤2)中的插入方式在图 \ref{figure:insertion}的中间部分。

3)最终把剩下的所有一元环插入轨道中。对每个环$(i), 2 \leqslant i \leqslant N$,总共有$\sum_{c\ni i}k^c-k^i$个可选择的位置插入。因此步骤 3)总共的可能的插入方式数量为:
\begin{equation*}\label{formula:A3}
    A_3 = \prod_{i=2}^N\binom{\sum_{c\ni i}k^{c}-1}{k^{i}}.
\end{equation*}
例子 \ref{trajectory_ex}中,步骤3)的插入方式在图 \ref{figure:insertion}的右部分。

结合上述三个步骤,最终可以得到容许轨道的数量为:$|G_n(k)|=A_1A_2A_3$。因此LE经验环流的联合分布为:
\begin{equation} \label{trajectories}
    \mathbb{P}\left(J^c_n=\nu^c,\;\forall c\in\mathcal{C}\right)
    = A_1 A_2 A_3 \Pi_{c\in\mathcal{C}} (\gamma^c)^{K^c}
\end{equation}
为了得到更明确的速率函数表达式$I_J$,先回顾$Stirling$公式:
\begin{equation*}
    \log n! = n\log n-n+O(\log n)=h(n)-n+O(\log n),
\end{equation*}
其中$h(x)=x \log x, x \geqslant 0$。记$k_i=\sum_{c\ni i}k^c$且$\nu_i=\sum_{c\ni i}\nu^c$,注意到$k_i$和$k^i$的定义是不同的,因此有$Stirling$可得:
\begin{equation}\label{log A1}
    \begin{split}
    \log A_1&=\log\frac{(k^1+k^{12}+k^{1N}+k^{+}+k^{-})!}{k^1!\;k^{12}!\;k^{1N}!\;k^{+}!\;k^{-}!}\\
    &= h(k_1)-h(k^1)-h(k^{12})-h(k^{1N})-h(k^+)-h(k^-)+O(\log n)\\
    &= n\left[h(\nu_1)-h(\nu^1)-h(\nu^{12})-h(\nu^{1N})-h(\nu^+)-h(\nu^-)\right]+O(\log n).
    \end{split}
\end{equation}
类似地,有:
\begin{equation}\label{log A3}
    \begin{split}
    \log A_3&=\log\left[\prod_{i=2}^N\binom{k_i-1}{k^{i}}\right]
    =\sum_{i=2}^N\log\left(\frac{\left(k_i\right)!}{\left(k^i\right)!\left(k_i-k^i\right)!}\right)+O(\log n)\\
    &=\sum_{i=2}^N\left[h(k_i)-h(k^i)-h(k_i-k^i)\right]+O(\log n)\\
    &=\sum_{i=2}^Nn\left[h(\nu_i )-h(\nu^i)-h(\nu_i-\nu^i)\right]+O(\log n).
    \end{split}
\end{equation}
最后,估计$\log A_2$,令$D = \{(l^i,m^i)_{2\le i\le N-1}:\;l^i,m^i\in\mathbb{N},\;l^i+m^i=k^{i,i+1}\}$ 表示$l^i$和$m^i$可能的组合形成的集合。记$L = (l^i,m^i)\in D$, 令
\begin{equation*}
    B_L=\prod_{i=2}^{N-1}\binom{l^{i}+l^{i-1}+k^{+}-1}{l^{i}}\binom{m^{i}+m^{i+1}+k^{-}-1}{m^{i}}.
\end{equation*}
表示在给定$l^i$和$m^i$时,步骤 2)中插入方式的数量。易知$|D| \leqslant n^{N-2}$,因此,可得:
\begin{equation}\label{inequality}
    \max_{L\in D}B_L \le A_2 \le (n+1)^{N-2} \max_{L\in D}B_L,
\end{equation}
该式中还运用了$A_2 = \sum_{L\in D}B_L$。类似于 \ref{log A3} 式,有:
\begin{equation}\label{log BL}
    \begin{split}
    \log B_L =&\;\sum_{i=2}^{N-1}[h(l^i+l^{i-1}+k^+)-h(l^i)-h(l^{i-1}+k^+)]\\
    &\;+\sum_{i=2}^{N-1}[h(m^i+m^{i+1}+k^-)-h(m^i)-h(m^{i+1}+k^-)]+O(\log n)\\
    =&\;\sum_{i=2}^{N-1}n[h(x^i+x^{i-1}+\nu^+)-h(x^i)-h(x^{i-1}+\nu^+)]\\
    &\;+\sum_{i=2}^{N-1}n[h(y^i+y^{i+1}+\nu^-)-h(y^i)-h(y^{i+1}+\nu^-)]+O(\log n),
    \end{split}
\end{equation}
其中$x^i = l^i/n$和$y^i = m^i/n$。对于任意的$\nu \in \mathcal{V}$,考虑空间:
\begin{equation*}
    V(\nu) = \left\{\left(x^{i},y^{i}\right)_{2\le i\le N-1}:\;x^i,y^i\geq0,\;x^{i}+y^{i}=\nu^{i,i+1}\right\},
\end{equation*}
且对任意$X = (x^i, y^i) \in V(\nu)$。定义函数:
\begin{equation}\label{formula:F}
    \begin{split}
    F_{\nu}(X)
    =&\sum_{i=2}^{N-1}\left[ h\left(x^{i-1}+\nu^+\right) + h\left(x^{i}\right) - h\left(x^{i-1}+x^{i}+\nu^+\right) \right] \\
    &+ \sum_{i=2}^{N-1} \left[h\left(y^{i}\right) + h\left(y^{i+1} +\nu^-\right)-h\left(y^{i} +y^{i+1} +\nu^-\right)\right].
    \end{split}
\end{equation}
其中$x^1=\nu^{12}$,$y^N=\nu^{N1}$,再由式联系 \ref{inequality}的结论,可得:
\begin{equation}\label{log A2}
    \log A_2 = \max_{L\in D}\log B_L+O(\log n)
    = n\sup_{X\in V(\nu)}F_{\nu}(X)+O(\log n).
\end{equation}
结合\eqref{LDP}和\eqref{trajectories},可得:
\begin{equation*}
    \begin{split}
    I_J(\nu) &= -\lim_{n\to\infty}\frac{1}{n}\log\mathbb{P}\left(J^c_n=\nu^c,\;\forall c\in\mathcal{C}\right)\\
    &= -\lim_{n\to\infty}\frac{1}{n}\left[\log A_1+\log A_2+\log A_3+\sum_{c\in\mathcal{C}}k^c\log\gamma^c\right].
    \end{split}
\end{equation*}
再联系\eqref{log A1}, \eqref{log A3},和 \eqref{log A2}式,可得:
\begin{equation}\label{ratefunction}
    \begin{split}
    I_J(\nu) =&\; \left[h\left(\nu^{12}\right)+h\left(\nu^{1N}\right)
    +h\left(\nu^+\right)+h\left(\nu^-\right)-h\left(\nu^{12}+\nu^{1N}+\nu^++\nu^-\right)\right] \\
    &\;+\inf_{X\in V(\nu)}F_{\nu}(X)+\sum_{i\in S}\left[ h\left(\nu_i-\nu^i\right)+h\left(\nu^i\right)
    -h\left(\nu_i\right)\right]-\sum_{c\in\mathcal{C}}\nu^c\log\gamma^c,
    \end{split}
\end{equation}
其中$h(x) = x \log x$,$\nu_i=\sum_{c\ni i}\nu^c$。这就是LE经验完整的环流速率函数表达式。该式中的$\inf_{X\in V(\nu)}F_{\nu}(X)$难以直接计算,不过可以通过拉格朗日乘子法得到。

下面将给出证明:
% 在附录 A 中,证明了
\begin{equation*}
    \inf_{X\in V(\nu)}F_{\nu}(X) = F_{\nu}(x^i,y^i),
\end{equation*}
其中$(x^i,y^i)_{2\leq i\leq N-1}$是下面代数方程的解:
\begin{equation}\label{equation}
    \begin{split}
    \frac{x^{i}}{x^{i-1}+x^{i}+\nu^+}\cdot\frac{x^{i}+\nu^+}{x^{i}+x^{i+1}+\nu^+}
    &= \frac{y^{i}+\nu^-}{y^{i-1}+y^{i}+\nu^-}\cdot\frac{y^{i}}{y^{i}+y^{i+1}+\nu^-},\\
    x^{i} + y^{i} &= \nu^{i,i+1},
    \end{split}
\end{equation}
这里$x^1=\nu^{12}$,$x^N=0$,$y^1=0$,且$y^N=\nu^{1N}$。

%%%%%%%%%%%%%%%%%%%%%%%%%%%%%%%%%%%%%%%%%%%%%%%%%%%%%%%%%%%%%%%%%%%%%%%%%%%%%%
% \subsubsection{附录 A:单环马氏链速率函数 $I_J$ 的表达式 }{} \label{appendix:explicit}
% 回顾单环系统,LE经验净环流$(J_n^c)_{c \in \mathcal{C}}$的速率函数为:
% \begin{equation*}\label{ratefunction2}
% 	\begin{split}
% 		I_J(\nu) =&\; \left[h\left(\nu^{12}\right)+h\left(\nu^{1N}\right)
% 		+h\left(\nu^+\right)+h\left(\nu^-\right)-h\left(\nu^{12}+\nu^{1N}+\nu^++\nu^-\right)\right] \\
% 		&\;+\inf_{X\in V(\nu)}F_{\nu}(X)+\sum_{i\in S}\left[ h\left(\nu_i-\nu^i\right)+h\left(\nu^i\right)
% 		-h\left(\nu_i\right)\right]-\sum_{c\in\mathcal{C}}\nu^c\log\gamma^c,
% 	\end{split}
% \end{equation*}
% % 下面将使用拉格朗日乘子法化简该式。固定 $\nu\in\mathcal{V}$,令拉格朗日函数 \cite{brian1990optimization} $\mathcal{A}_{\nu}:V(\nu)\times \mathbb{R}^{N-2}\to \mathbb{R}$ 为:
% 这里$\inf_{X\in V(\nu)}F_{\nu}(X)$不是闭形式,下面利用拉格朗日乘子法求解$\inf_{X\in V(\nu)}F_{\nu}(X)$。

对任意$\nu\in\mathcal{V}$,定义拉格朗日函数为:
\begin{align*}
    \mathcal{A}_{\nu}(X,\lambda) = F_{\nu}(X) + \sum_{i=2}^{N-1} \lambda_i \left(x^{i} + y^{i} - \nu^{i,i+1}\right),
\end{align*}
其中 $X=(x^i,y^i)_{2\le i\le N-1}\in V(\nu)$ 并且 $\lambda=(\lambda_i)_{2\le i\le N-1}\in \mathbb{R}^{N-2}$。分别对 $x^{i}$,$y^{i}$ 和 $\lambda_i$求导,可以得到下列方程:
\begin{equation}\label{equation1}
	\begin{split}
		\log\left(x^{i}\right) - \log\left(x^{i-1}+x^{i}+\nu^{+}\right)  + \log\left(x^{i}+\nu^{+}\right) -\log\left(x^{i}+x^{i+1}+\nu^{+}\right)+\lambda_i  &= 0, \\
		\log\left(y^{i}+\nu^{-}\right) -\log\left(y^{i-1}+y^{i}+\nu^{-}\right)  + \log\left(y^{i}\right) - \log\left(y^{i}+y^{i+1}+\nu^{-}\right) +\lambda_i &= 0, \\
		x^{i} + y^{i} = \nu^{i,i+1},\qquad 2\le i\le N-1.\qquad\qquad\qquad
	\end{split}
\end{equation}
而且,可以把方程组 \eqref{equation1} 写为:
\begin{equation}\label{equations}
    \begin{split}
    \frac{x^{i}}{x^{i-1}+x^{i}+\nu^+}
    \,\frac{x^{i}+\nu^+}{x^{i}+x^{i+1}+\nu^+}
    &= \frac{y^{i}+\nu^-}{y^{i-1}+y^{i}+\nu^-}
    \,\frac{y^{i}}{y^{i}+y^{i+1}+\nu^-}=e^{-\lambda_i},\\
    x^{i} + y^{i} &= \nu^{i,i+1},\qquad 2\le i\le N-1.
    \end{split}
\end{equation}
其中 $x^1=\nu^{12}$,$x^N=0$,$y^1=0$ 和 $y^N=\nu^{1N}$。

\begin{lemma}\label{lemma:existence for equations solution}
	方程 \eqref{equations} 有解 $X=(x^i,y^i)\in V(\nu)$。
\end{lemma}
\begin{proof}
    若对某些 $2\le k\le N-1$ 存在 $\nu^{k,k+1}=0$,则有 $x^{k}=y^{k}=0$ 成立。那么依据指标 $2\le i\le k-1$ 和 $k+1\le i\le N-1$,方程 \eqref{equations} 可以被分为两个方程。因此对 $\nu^{k,k+1}>0, \forall k, 2\le k\le N-1$ 证明引理,下面会从三种不同情况考虑这个引理。

    情况 1:$\nu^{12}=\nu^{+}=\nu^{1N}=\nu^-=0$。
    易知,对每个 $\alpha\in (0,1)$,
    \begin{equation*}
        x^{i}=\alpha\nu^{i,i+1},\quad y^{i}=(1-\alpha)\nu^{i,i+1},\quad 2\le i\le N-1,
    \end{equation*}
    都是 \eqref{equations} 的解。

    情况 2:$\nu^{12}=\nu^{+}=0,\nu^{1N}+\nu^{-}>0$ 或 $\nu^{1N}=\nu^{-}=0,\nu^{12}+\nu^{+}>0$。容易验证若 $\nu^{12}=\nu^{+}=0,\nu^{1N}+\nu^{-}>0$成立,则$x^{i}=0,y^{i}=\nu^{i,i+1}$ 是方程 \eqref{equations} 的解。

    情况 3:$\nu^{12}+\nu^{+}>0$ 或 $\nu^{1N}+\nu^{-}>0$。上述已证明对任意给定的 $x^{k+1}\ge 0$,$y^{k+1}>0$,和 $x^{k+1}+y^{k+1}=\nu^{k+1,k+2}$,下列方程 \eqref{equations2} 满足 $x^{i},y^{i}> 0, \forall i, 2\le i\le k$。
    \begin{equation}\label{equations2}
        \begin{split}
            \frac{x^{i}}{y^{i}}
            =&\frac{x^{i}+x^{i+1}+\nu^+}{x^{i}+\nu^+}\, \frac{y^{i}+\nu^-}{y^{i-1}+y^{i}+\nu^-}
            \,\frac{x^{i-1}+x^{i}+\nu^+}{y^{i}+y^{i+1}+\nu^-},\\
            &\qquad x^{i} + y^{i} = \nu^{i,i+1},\qquad 2\le i\le k.
        \end{split}
    \end{equation}

    下面通过归纳法证明。若 $k=2$,方程 \eqref{equations2} 可以简化为:
    \begin{equation*}\label{equation k=2}
        \frac{x^{2}}{\nu^{23}-x^2}
        =\frac{x^{2}+x^{3}+\nu^+}{x^{2}+\nu^+}\, \frac{\nu^{12}+x^{2}+\nu^{+}}{\nu^{23}-x^2+y^3+\nu^-}.
        %,\qquad \nu^+_{23} + \nu^-_{23} = \nu_{23}.
    \end{equation*}
    易得:
    \begin{equation*}
        \lim_{x^{2}\downarrow 0}\frac{x^{2}}{\nu^{23}-x^2} = 0,\qquad \lim_{x^{2}\downarrow 0}\frac{x^{2}+x^{3}+\nu^+}{x^{2}+\nu^+}\, \frac{\nu^{12}+x^{2}+\nu^{+}}{\nu^{23}-x^2+y^3+\nu^-} \ge  \frac{\nu^{12}+\nu^{+}}{\nu^{23}+y^{3}+\nu^-} > 0.
    \end{equation*} 
    另外,
    \begin{equation*}
        \lim_{x^2\uparrow \nu^{23}}\frac{x^{2}}{\nu^{23}-x^2} = \infty,\qquad \lim_{x^{2}\uparrow \nu^{23}}\frac{x^{2}+x^{3}+\nu^+}{x^{2}+\nu^+}\, \frac{\nu^{12}+x^{2}+\nu^{+}}{\nu^{23}-x^2+y^3+\nu^-} =  \frac{\nu^{23}+x^{3}+\nu^+}{\nu^{23}+\nu^+}\frac{\nu^{12}+\nu^{23}+\nu^{+}}{y^{3}+\nu^-} < \infty.
    \end{equation*}
    通过中值定理,可以找到满足 $x^{i},y^i>0$ 的方程 \eqref{equations2} 的解,并且有 $x^{i}+y^{i}=\nu^{i,i+1}$ 。假设对 $k=n-1$ 命题成立,那么可以考虑方程:
    \begin{equation*}
        \frac{x^{n}}{\nu^{n,n+1}-x^{n}} = \frac{x^{n}+x^{n+1}+\nu^+}{x^{n}+\nu^+}\,\frac{\nu^{n,n+1}-x^{n}+\nu^-}{y^{n-1}+\nu^{n,n+1}-x^{n}+\nu^-}\,\frac{x^{n-1}+x^{n}+\nu^+}{\nu^{n,n+1}-x^{n}+y^{n+1}+\nu^-},
    \end{equation*}
    其中 $x^{n-1}$, $y^{n-1}$ 是方程 \eqref{equations2} 在 $k=n-1$ 时的解。可以得到:
    \begin{equation}\label{limit1}
        \lim_{x^{n}\downarrow 0}\frac{x^{n}}{\nu^{n,n+1}-x^{n}}=0,
    \end{equation}
    并且
    \begin{equation}\label{limit2}
        \begin{split}
            &\;\lim_{x^{n}\downarrow 0}\frac{x^{n}+x^{n+1}+\nu^+}{x^{n}+\nu^+}\,\frac{\nu^{n,n+1}-x^{n}+\nu^-}{y^{n-1}+\nu^{n,n+1}-x^{n}+\nu^-}\,\frac{x^{n-1}+x^{n}+\nu^+}{\nu^{n,n+1}-x^{n}+y^{n+1}+\nu^-}\\
            \ge &\;\frac{\nu^{n,n+1}+\nu^-}{\lim_{x^{n}\downarrow 0}(y^{n-1})+\nu^{n,n+1}+\nu^-}\,\frac{\lim_{x^{n}\downarrow 0}(x^{n-1})+\nu^+}{\nu^{n,n+1}+y^{n+1}+\nu^-}>0.
        \end{split}
    \end{equation}
    利用归纳法得出的结论 $\lim_{x^{n}\downarrow 0}(x^{n-1})>0$,可以得到下面两个式子成立。
    \begin{equation}\label{limit3}
        \lim_{x^{n}\uparrow \nu^{n,n+1}}\frac{x^{n}}{\nu^{n,n+1}-x^{n}}=\infty,
    \end{equation}
    和
    \begin{equation}\label{limit4}
        \begin{split}
            &\;\lim_{x^{n}\uparrow \nu^{n,n+1}}\frac{x^{n}+x^{n+1}+\nu^+}{x^{n}+\nu^+}\,\frac{\nu^{n,n+1}-x^{n}+\nu^-}{y^{n-1}+\nu^{n,n+1}-x^{n}+\nu^-}\,\frac{x^{n-1}+x^{n}+\nu^+}{\nu^{n,n+1}-x^{n}+y^{n+1}+\nu^-}\\
            \le &\;\frac{\nu^{n,n+1}+x^{n+1}+\nu^+}{\nu^{n,n+1}+\nu^+}\,\frac{\lim_{x^{n}\uparrow \nu^{n,n+1}}x^{n-1}+\nu^{n,n+1}+\nu^+}{y^{n+1}+\nu^-}<\infty.
        \end{split}
    \end{equation}
    那么通过中值定理,可以找到方程 \eqref{equations2} 满足条件 $x^{i},y^i>0, x^{i}+y^{i}=\nu^{i,i+1}, \forall i=2,3, \cdots, n$ 的解,就是说上述命题对 $k=n$ 成立。

    当 $k=N-1$ 时,有 $y^{k+1}=y^N=\nu^{1N}$ 成立。注意到 $\nu^{1N}+\nu^->0$,那么可以得到 \eqref{limit1},\eqref{limit2},\eqref{limit3} 和 \eqref{limit4} 成立。通过中值定理,可以找到方程 \eqref{equations2} 满足条件 $x^{i},y^i>0, x^{i}+y^{i}=\nu^{i,i+1}, \forall i=2,3, \cdots, n$ 的解。当 $k=N-1$ 时,方程 \eqref{equations2} 与 方程 \eqref{equations} 等价。证毕。

\end{proof}
\begin{lemma}\label{lemma:mininum}
    令 $X=(x^i,y^i)\in V(\nu)$ 是引理 ~\ref{lemma:existence for equations solution} 的解,那么 $X$ 是 $F_{\nu}(\cdot)$ 定义在 $V(\nu)$ 下的最小值点。 
\end{lemma}
\begin{proof}
    对任意 $a_1,a_2,b_1,b_2\ge 0$,通过 log-sum 不等式(见 \eqref{log sum inequality}),可以得到:
    \begin{align}\label{log sum inequality 2}
        a_1\log\frac{a_1}{a_1+a_2}+a_2\log\frac{a_2}{a_1+a_2}\ge a_1\log\frac{b_1}{b_1+b_2}+a_2\log\frac{b_2}{b_1+b_2}.
    \end{align}
    对任意 $Z=(z^i,w^i)\in V(\nu)$,可以把 \eqref{formula:F} 写为:
    \begin{align*}
        F_{\nu}(Z)
        =&\;\sum_{i=2}^{N-1}\left[ \left(z^{i-1}+\nu^+\right)\log\frac{z^{i-1}+\nu^+}{z^{i-1}+z^{i}+\nu^+} + z^{i}\log\frac{z^{i}}{z^{i-1}+z^{i}+\nu^+}  \right]\\
        &\;+\sum_{i=2}^{N-1}  \left[w^{i}\log\frac{w^{i}}{w^{i} +w^{i+1} +\nu^-} + \left(w^{i+1} +\nu^-\right)\log\frac{w^{i+1} +\nu^-}{w^{i} +w^{i+1} +\nu^-}\right],
    \end{align*}
    其中 $z^{1}=\nu^{12},w^{N}=\nu^{1N}$。再根据方程 \eqref{equations} 和 \eqref{log sum inequality 2}
    \begin{align*}
        F_{\nu}(Z)\ge &\;\sum_{i=2}^{N-1}\left[ \left(z^{i-1}+\nu^+\right)\log\frac{x^{i-1}+\nu^+}{x^{i-1}+x^{i}+\nu^+} + z^{i}\log\frac{x^{i}}{x^{i-1}+x^{i}+\nu^+}  \right]\\
        &\;+\sum_{i=2}^{N-1}  \left[w^{i}\log\frac{y^{i}}{y^{i} +y^{i+1} +\nu^-} + \left(w^{i+1} +\nu^-\right)\log\frac{y^{i+1} +\nu^-}{y^{i} +y^{i+1} +\nu^-}\right]\\
        =&\;\sum_{i=2}^{N-1}\left[ \nu^+\log\frac{x^{i-1}+\nu^+}{x^{i-1}+x^{i}+\nu^+} + z^{i}\log\frac{x^{i}+\nu^{+}}{x^{i-1}+x^{i}+\nu^{+}}\frac{x^{i}}{x^{i-1}+x^{i}+\nu^+}  \right]\\
        &\;+\sum_{i=2}^{N-1}  \left[\nu^-\log\frac{y^{i+1} +\nu^-}{y^{i} +y^{i+1} +\nu^-}+w^{i}\log\frac{y^{i}+\nu^{-}}{y^{i-1}+y^{i}+\nu^{-}}\frac{y^{i}}{y^{i} +y^{i+1} +\nu^-}  \right]\\
        &\;+\nu^{12}\log\frac{\nu^{12}+\nu^{+}}{\nu^{12}+x^{2}+\nu^{+}}+\nu^{1N}\log\frac{\nu^{1N}+\nu^{-}}{\nu^{1N}+y^{N-1}+\nu^{-}}\\
        =&\;\sum_{i=2}^{N-1}\left[-\lambda_i\nu^{i,i+1}+\nu^+\log\frac{x^{i-1}+\nu^+}{x^{i-1}+x^{i}+\nu^+}+\nu^-\log\frac{y^{i+1} +\nu^-}{y^{i} +y^{i+1} +\nu^-}\right]\\
        &\;+\nu^{12}\log\frac{\nu^{12}+\nu^{+}}{\nu^{12}+x^{2}+\nu^{+}}+\nu^{1N}\log\frac{\nu^{1N}+\nu^{-}}{\nu^{1N}+y^{N-1}+\nu^{-}}\\
        =&\;\sum_{i=2}^{N-1}\left[ \left(x^{i-1}+\nu^+\right)\log\frac{x^{i-1}+\nu^+}{x^{i-1}+x^{i}+\nu^+} + x^{i}\log\frac{x^{i}}{x^{i-1}+x^{i}+\nu^+}  \right]\\
        &\;+\sum_{i=2}^{N-1}  \left[y^{i}\log\frac{y^{i}}{y^{i} +y^{i+1} +\nu^-} + \left(y^{i+1} +\nu^-\right)\log\frac{y^{i+1} +\nu^-}{y^{i} +y^{i+1} +\nu^-}\right]\\
        =&\;F_{\nu}(X),
        \end{align*}
        其中 $\lambda_i$ 被表示为 \eqref{equations}。证毕。
\end{proof}
%%%%%%%%%%%%%%%%%%%%%%%%%%%%%%%%%%%%%%%%%%%%%%%%%%%%%%%%%%%%%%%%%%%%%%%%%%%%%%%%%

上述论证,都是基于系统的初始状态是$1$,自然问到其他初始分布会不会影响速率函数的表达式。由于证明过程非常复杂,所以在附录部分给出了速率函数不依赖初始状态的证明。由 \eqref{ratefunction} 的形式,可以看出这是一个极不平凡的结论。

为建立经验LE环流大偏差原理,依然要证明速率函数的水平集是紧的,也就是验证条件 \eqref{def1} 和 \eqref{def2}。

% 大偏差原理严格的证明在附录 B 中给出,经过整理后,得到下面的定理
下面给出大偏差原理的严格证明:
\begin{theorem}\label{thm:LDP}
    单环马氏链的经验环流$(J^c_n)_{c\in\mathcal{C}}$满足大偏差原理,并且相应的速率函数$I_J:\mathcal{V}\to [0,\infty]$满足上式 \eqref{ratefunction}. 速率函数$I_J$满足有界性,连续性和凸性。并且,速率函数$I_J$并不依赖初始分布的选择。
\end{theorem}

%%%%%%%%%%%%%%%%%%%%%%%%%%%%%%%%%%%%%%%%%%%%%%%%%%%%%%%%%%%%%%%%%%%%%%%%%%%%%%%%%
% \subsubsection{Proof of Theorem~\ref{thm:LDP}}\label{appendix:LDP}

这里把大偏差原理的证明 ~\ref{thm:LDP} 分为两部分。首先研究经验环流的速率函数的性质。回顾速率函数 $I_J:\mathcal{V}\to[0,\infty]$ 的形式为:
\begin{equation*}\label{ratefunction1}
	\begin{split}
		I_J(\nu) =&\; \left[h\left(\nu^{12}\right)+h\left(\nu^{1N}\right)
		+h\left(\nu^+\right)+h\left(\nu^-\right)-h\left(\nu^{12}+\nu^{1N}+\nu^++\nu^-\right)\right] \\
		&\;+\inf_{X\in V(\nu)}F_{\nu}(X)+\sum_{i\in S}\left[ h\left(\nu_i-\nu^i\right)+h\left(\nu^i\right)
		-h\left(\nu_i\right)\right]-\sum_{c\in\mathcal{C}}\nu^c\log\gamma^c\\
		:=&\;I_1(\nu)+I_2(\nu)+I_3(\nu)+I_4(\nu).
	\end{split}
\end{equation*}
\begin{proposition}\label{proposition:I}
	速率函数 $I_J$ 是有界,连续的凸函数。
\end{proposition}
\begin{proof}
	易知 $I_j$ 有界。首先证明 $I_J$ 连续。易知 $h$ 是定义在 $[0,\infty)$ 的连续函数。那么 $I_1$ 和 $I_3$ 是连续的。注意到  $I_4$ 是关于 $\nu$ 的连续函数,因此还需证明 $I_2$ 是连续函数。令 $Y(\nu)\in V(\nu)$ 是方程 \eqref{equations} 的解,由于 \eqref{equations} 是一个多项式方程组,因此易知 $Y(\nu)$ 是关于变量 $\nu$ 的连续函数。因为 $h$ 是连续的,再根据 \eqref{formula:F},$\inf_{X\in V(\nu)}F_{\nu}(X)=F_{\nu}(Y(\nu))$ 是关于 $\nu$ 的连续函数。
	
	然后证明 $I_J$ 是凸函数。 回顾 log-sum 不等式,对任意 $a_{1},a_{2},b_{1},b_{2}\ge 0$,有:
	\begin{equation}\label{log sum inequality}
		(a_1+a_2)\log \frac{a_1+a_2}{b_1+b_2}\le a_{1}\log \frac{a_{1}}{b_{1}}+a_{2}\log \frac{a_{2}}{b_{2}} ,
	\end{equation}
	注意到
	\begin{equation}\label{I_1}
		I_1(\nu) = \nu^{12}\log\left(\frac{\nu^{12}}{\hat{\nu}}\right)+\nu^{1N}\log\left(\frac{\nu^{1N}}{\hat{\nu}}\right)+\nu^{+}\log\left(\frac{\nu^{+}}{\hat{\nu}}\right)+\nu^{-}\log\left(\frac{\nu^{-}}{\hat{\nu}}\right),
	\end{equation}
	其中 $\hat{\nu}=\nu^{12}+\nu^{1N}+\nu^++\nu^-$。对任意满足 $\alpha+\beta=1$ 的 $\alpha,\beta\ge 0$ 和 $\nu,\mu\in \mathcal{V}$,通过log-sum 不等式 \eqref{log sum inequality},可以得到:
	\begin{equation*}
		(\alpha \nu^{12}+\beta \mu^{12})\log\left(\frac{\alpha \nu^{12}+\beta \mu^{12}}{\alpha \hat{\nu}+\beta \hat{\mu}}\right)\le \alpha\nu^{12}\log\left(\frac{\nu^{12}}{\hat{\nu}}\right)+\beta\mu^{12}\log\left(\frac{\mu^{12}}{\hat{\mu}}\right),
	\end{equation*}
	其中 $\hat{\mu}=\mu^{12}+\mu^{1N}+\mu^++\mu^-$。这说明 \eqref{I_1} 式的第一项的右边是关于 $\nu$ 的连续函数,同理,其他三项项的右边也是关于 $\nu$ 的连续函数。因此 $I_1(\nu)$ 是凸函数,同理,可以通过log-sum 不等式得到 $I_3(\nu)$也是凸函数。由于 $I_4(\nu)$ 是线性函数,故也是凸函数,最后只需证明 $\inf_{X\in V(\nu)}F_{\nu}(X)$ 是凸函数。 吧\eqref{formula:F} 重写为:
	\begin{align*}
		F_{\nu}(X)=&\;\sum_{i=2}^{N-1}\left[(x^{i-1}+\nu^+)\log\left(\frac{x^{i-1}+\nu^+}{x^{i-1}+x^i+\nu^+}\right)+x^i\log\left(\frac{x^{i}}{x^{i-1}+x^i+\nu^+}\right)\right]\\
		&\;+\sum_{i=2}^{N-1}\left[y^i\log\left(\frac{y^{i}}{y^{i}+y^{i+1}+\nu^-}\right)+(y^{i+1}+\nu^-)\log\left(\frac{y^{i+1}+\nu^-}{y^{i}+y^{i+1}+\nu^-}\right)\right]\\
		:=&\;\sum_{i=2}^{N-1}[A_1^i(\nu,X)+A_2^i(\nu,X)+A_3^i(\nu,X)+A_4^i(\nu,X)].
	\end{align*}
	注意到对任意 $X=(x^i,y^i)\in V(\nu)$ 和 $Z=(z^i,w^i)\in V(\mu)$,有 $\alpha X+\beta Z\in V(\alpha\nu+\beta\mu)$。那么通过 log-sum 不等式,可得:
	\begin{align*}
		&\;A_1^i(\alpha\nu+\beta \mu,\alpha X+\beta Y)\\
		=&\;(\alpha x^{i-1}+\beta z^{i-1}+\alpha\nu^++\beta \mu^+ )\log\left(\frac{\alpha x^{i-1}+\beta z^{i-1}+\alpha\nu^++\beta \mu^+}{\alpha x^{i-1}+\alpha x^i+\beta z^{i-1}+\beta z^i+\alpha\nu^++\beta \mu^+}\right)\\
		=&\;(\alpha (x^{i-1}+\nu^+)+\beta (z^{i-1}+ \mu^+) )\log\left(\frac{\alpha (x^{i-1}+\nu^+)+\beta (z^{i-1}+ \mu^+)}{\alpha (x^{i-1}+ x^i+\nu^+)+\beta (z^{i-1}+ z^i+ \mu^+)}\right)\\
		\le &\; \alpha (x^{i-1}+\nu^+)\log\left(\frac{x^{i-1}+\nu^+}{x^{i-1}+ x^i+\nu^+}\right)+\beta (z^{i-1}+ \mu^+)\log\left(\frac{z^{i-1}+ \mu^+}{z^{i-1}+ z^i+ \mu^+}\right)\\
		=&\;\alpha A^i_1(\nu,X)+\beta A^i_1(\mu,Y).
	\end{align*}
	同理, 对任意 $j=2,3,4$ 有 $A^i_j(\alpha \nu+\beta \mu,\alpha X+\beta Y)\le \alpha A^i_j(\nu,X)+\beta A^i_j (\mu,Y)$。这说明
	\begin{equation*}
		F_{\alpha \nu+\beta \mu}(\alpha X+\beta Y)\le \alpha F_{\nu}(X)+\beta F_{\mu}(Y).
	\end{equation*}
	针对 $X$ 和 $Y$ 取极小,可以得到:
	\begin{align*}
		\inf_{Z\in V(\alpha\nu+\beta\mu)}F_{\alpha\nu+\beta\mu}( Z)\le\alpha \inf_{X\in V(\nu)} F_{\nu}(X)+\beta \inf_{Y\in V(\mu)} F_{\mu}(Y).
	\end{align*}
	证毕。
\end{proof}
下面给出 LDP 原理的严格证明。
\begin{proposition}\label{theorem:LDP}
	经验环流 $(J^c_n)_{c\in\mathcal{C}}$ 满足速率为 $n$ 的大偏差原理,并且相应的速率函数为 $I_J:\mathcal{V}\to [0,\infty]$。此外,它的上界可以提升为:对任意集合 $\Gamma \subset \mathcal{V}$,有:
	\begin{align}\label{upper bound}
		\varlimsup_{n\to+\infty}\;
		\frac 1n \log \mathbb  P \left( (J^c_n)_{c\in \mathcal{C}} \in \Gamma \right)
		\le -\inf_{\nu\in \Gamma} I_J(\nu).
	\end{align}
\end{proposition}
\begin{proof}
记
\begin{align}\label{set:K_n}
K_n := \left\{(k^c)_{c\in \mathcal{C}}\in \mathbb{N}^{2N+2}: \sum_{c \in \mathcal{C}} k^{c} |c| =n \right\}.
\end{align}
接下来的证明将假设马氏链从状态 1 出发,这并不会降低命题的一般性。那么,对任意 $k=(k^c)_{c\in\mathcal{C}}\in K_n$,由 \eqref{joint} 式,可得:
\begin{align}\label{Jnxi}
\mathbb{P}_1\left(J^c_{n}= \frac{k^c}{n},\ \forall c\in \mathcal{C}\right)
= |G_n(k)| \prod_{c\in\mathcal{C}}\left(\gamma^c\right)^{k^c},
\end{align}
令 $\mu_n(k) = k/n\in\mathcal{V}$。对任意 $\Gamma\subset \mathcal{V}$,令
\begin{align*}
Q_n(\Gamma) = \max_{k\in K_n: \mu_n(k) \in \Gamma}
\mathbb{P}_1\left(J^c_{n} = \frac{k^c}{n},\ \forall c\in\mathcal{C}\right).
\end{align*}
显然有:
\begin{align}\label{Qn gamma}
	Q_n(\Gamma)
	\le \mathbb{P}_1\left(J_{n} \in \Gamma\right)
	\le |K_n| Q_n(\Gamma).
\end{align}
易知 $|K_n| \le (2N+2)(n+1)^{2N+3}$,那么有公式 \eqref{Jnxi},\eqref{Qn gamma},\eqref{log A1},\eqref{log A3},和\eqref{log A2}
\begin{equation}\label{1 n log P}
	\begin{split}
	\frac{1}{n}\log\mathbb{P}_1\left(J_{n} \in \Gamma\right)=&O\left(\frac{\log n}{n}\right)+\frac{1}{n}\log Q_n(\Gamma)\\
	=&O\left(\frac{\log n}{n}\right)+\max_{k\in K_n: \mu_n(k) \in \Gamma}\left[\frac{1}{n}\log |G_{n}(k)|+\sum_{c \in \mathcal{C}}\frac{k_c}{n}\log \gamma^c\right]\\
	=&O\left(\frac{\log n}{n}\right)-\min_{k\in K_n: \mu_n(k) \in \Gamma}I_J\left(\mu_{n}(k)\right).
	\end{split}
\end{equation}
由于 $\cup_{n\in\mathbb{N}} \{\mu_{n}(k):k\in K_n\}$ 在 $\mathcal{V}$ 中稠密,且 $\nu\to I_J(\nu)$ 在 $\mathcal{V}$ 中连续(见命题 \ref{proposition:I}),这保证了对每个 $\nu\in \mathcal{V}$,存在序列 $(k_n)_{n\in \mathbb{N}}$,使得:
\begin{equation*}
\lim_{n\to\infty} \|\mu_{n}(k_n)-\nu\| = 0, \qquad \lim_{n\to\infty} I_J\left(\mu_{n}(k_n)\right) = I_J(\nu).
\end{equation*}
那么对任意开集 $U\subset \mathcal{V}$
\begin{equation*}
\varlimsup_{n\to\infty}\min_{k\in K_n: \mu_n(k) \in U}I_J\left(\mu_{n}(k)\right)\le I_J(\nu),\quad\forall\nu\in U.
\end{equation*}
对 $\nu\in U$ 取极小,可得:
\begin{equation}\label{varlimsup}
\varlimsup_{n\to\infty}\min_{k\in K_n: \mu_n(k) \in U}I_J\left(\mu_{n}(k)\right)\le \inf_{\nu\in U}I_J(\nu).
\end{equation}
结合 \eqref{1 n log P} 和 \eqref{varlimsup},可得大偏差原理的下界 \eqref{def1}。此外,对任意$\Gamma \subset \mathcal{V}$,同理可得反向的不等式,即
\begin{equation}\label{varliminf}
\varliminf_{n\to\infty}\min_{k\in K_n: \mu_n(k) \in \Gamma}I_J\left(\mu_{n}(k)\right)\ge \inf_{\nu\in \Gamma}I_J(\nu).
\end{equation}
结合 \eqref{1 n log P} 和 \eqref{varliminf},可得大偏差原理的上界 \eqref{def2}。
\end{proof}
%%%%%%%%%%%%%%%%%%%%%%%%%%%%%%%%%%%%%%%%%%%%%%%%%%%%%%%%%%%%%%%%%%%%%%%%%%%%%%%%%%%%

一般单环马氏链的速率函数表达式 \eqref{ratefunction} 十分复杂。不过,如果令状态$1$到状态$N$的转移概率为 0,或者只考虑三状态的马氏链,速率函数表达式会被简化。

%%%%%%%%%%%%%%%%%%%%%%%%%%%%%%%%%%%%%%%%%%%%%%%%%%%%%%%%%%%%%%%%%%%%%%
% \subsubsection{简化速率函数 $I_J$}\label{appendix:threestate}

这里将对两种特殊情况,给出速率函数 $I_J$ 的简化形式,1)三状态马氏链。2)状态 1 到状态 N 的转移概率为 0 的单环马氏系统。回顾速率函数 $I_J:\mathcal{V}\to[0,\infty]$ 的公式为:
\begin{equation*}\label{ratefunction1}
	\begin{split}
		I_J(\nu) =&\; \left[h\left(\nu^{12}\right)+h\left(\nu^{1N}\right)
		+h\left(\nu^+\right)+h\left(\nu^-\right)-h\left(\nu^{12}+\nu^{1N}+\nu^++\nu^-\right)\right] \\
		&\;+\inf_{X\in V(\nu)}F_{\nu}(X)+\sum_{i\in S}\left[ h\left(\nu_i-\nu^i\right)+h\left(\nu^i\right)
		-h\left(\nu_i\right)\right]-\sum_{c\in\mathcal{C}}\nu^c\log\gamma^c\\
		:=&\;I_1(\nu)+I_2(\nu)+I_3(\nu)+I_4(\nu).
	\end{split}
\end{equation*}
\begin{proposition}
三状态马氏链的速率函数为
\begin{align*}
I_J(\nu) =
\sum_{i\in S} \left[\nu^{i}\log \left(\frac{\nu^{i}/\nu_i}{J^i/J_i}\right) + (\nu_i - \nu^i)\log \left(\frac{(\nu_i - \nu^i)/\nu_i}{(J_i - J^i)/J_i} \right)
\right]
+ \sum_{c \in \mathcal{C}, |c|\neq 1} \nu^{c} \log \left(\frac{\nu^{c}/\tilde{\nu}}{J^c/\tilde{J}}\right).
\end{align*}
\end{proposition}
\begin{proof}
	易知 $X=(x^2,y^2)$ 为方程 \eqref{equations} 的解,其中
\begin{equation*}
	x^{2}=\frac{\nu^{23}\left(\nu^{12}+\nu^+\right)}{\nu^{12}+\nu^{13}+\nu^++\nu^-},\quad y^{2}=\frac{\nu^{23}\left(\nu^{13}+\nu^-\right)}{\nu^{12}+\nu^{13}+\nu^++\nu^-}.
\end{equation*}
根据引理 ~\ref{lemma:mininum}
\begin{equation*}\label{Fnu2}
I_2(\nu)=F_{\nu}(X)=\nu^{23}\log\frac{\nu^{23}}{\tilde{\nu}}+\left(\nu^{12}+\nu^{13}+\nu^++\nu^-\right)\log\frac{\tilde{\nu}-\nu^{23}}{\tilde{\nu}}.
\end{equation*}
那么通过计算,可以得到:
\begin{equation}\label{I1I2I3}
	I_1(\nu)+I_2(\nu)+I_3(\nu) = \sum_{i\in S} \left[\nu^{i}\log \frac{\nu^{i}}{\nu_i} + (\nu_i - \nu^i)\log \frac{\nu_i - \nu^i}{\nu_i} 
	\right]
	+ \sum_{c \in \mathcal{C}, |c|\neq 1} \nu^{c} \log \frac{\nu^{c}}{\tilde{\nu}}.
\end{equation}
回顾环流的表达 \cite[Theorem.1.3.3]{jiang2004mathematical},有
\begin{equation}\label{J+J-Jii+1}
J^+=\gamma^+\frac{1}{C},\quad J^-=\gamma^-\frac{1}{C},\quad J^{i,i+1}=\gamma^{i,i+1}\frac{1-p_{i-1,i-1}}{C}, \quad 1\le i\le 3,
\end{equation}
其中ere $C=3+\sum_{i\in S}[p_{ii}p_{i+1,i+1}-p_{i,i+1}p_{i+1,i}-2(p_{ii}+p_{i+1,i+1})]$. 根据 \eqref{decomposition},
\begin{equation}\label{pij}
	p_{ij}=\frac{\sum_{c \ni \langle i,j\rangle}J^c}{\sum_{c\ni i}J^c}.
\end{equation}
结合 \eqref{J+J-Jii+1} 和 \eqref{pij},有:
\begin{equation}\label{nucgamma2}
	\begin{split}
		&\;\sum_{i \in S}\left[\nu^i\log\frac{J^i}{J_i}+\left(\nu_i-\nu^i\right)\log\frac{J_i-J^i}{J_i}\right]+ \sum_{c \in \mathcal{C}, |c|\neq 1} \nu^{c} \log \left(\frac{J^c}{\tilde{J}}\right)\\
		=&\;\sum_{i \in S}\left[\nu^i\log\frac{J^i}{J_i}+\nu^{i,i+1}\log\left(\left(1-\frac{J^i}{J_i}\right)\left(1-\frac{J^{i+1}}{J_{i+1}}\right)\frac{J^{i,i+1}}{\tilde{J}}\right)\right]\\
		&\;+\nu^+\log\left(\frac{J^+}{\tilde{J}}\prod_{i\in\mathcal{C}}\left(1-\frac{J^i}{J_i}\right)\right)+\nu^-\log\left(\frac{J^-}{\tilde{J}}\prod_{i\in\mathcal{C}}\left(1-\frac{J^i}{J_i}\right)\right)\\
		=&\;\sum_{c \in \mathcal{C}}\nu^c \log\gamma^c=-I_4(\nu).
	\end{split}
\end{equation}
结合 \eqref{I1I2I3} 和 \eqref{nucgamma2},得证。
\end{proof}
\begin{proposition}
	状态 1 到状态 N 的转移概率为 0 的单环马氏链的速率函数为:
	\begin{equation*}
		\begin{split}
			I_J(\nu) = \sum_{i\in S}\left[\nu^i\log\left(\frac{\nu^i/\nu_i}{J^i/J_i}\right)
			+\nu^{i,i+1}\log\left(\frac{\nu^{i,i+1}/\nu_i}{J^{i,i+1}/J_i}\right)+\left(\nu^{i-1,i}+\nu^+\right)\log\left(\frac{\left(\nu^{i-1,i}+\nu^+\right)/\nu_i}
			{\left(J^{i-1,i}+J^+\right)/J_i}\right)\right].
		\end{split}
	\end{equation*}
\end{proposition}
\begin{proof}
	在 $p_{1N}=0$ 的条件下,环 $(1,N)$ 和环 $(1,N,\cdots,2)$ 不会被形成。因此,可以得到\eqref{equation} 中的 $\nu^{1N}=\nu^{-}=0$ ,并且 $x^{i}=\nu^{i,i+1}$, $y^{i}=0$ 是方程 \eqref{equation}的解,那么
	\begin{equation*}
		I_2(\nu)=F_{\nu}(x^i,y^i)=\sum_{i=2}^{N-1}\left[-\lambda_{i}\nu^{i,i+1}+\nu^+\log\frac{\nu^{i-1,i}+\nu^+}{\nu^{i-1,i}+\nu^{i,i+1}+\nu^+}\right]+\nu^{12}\log\frac{\nu^{12}+\nu^+}{\nu^{12}+\nu^{23}+\nu^+},
	\end{equation*}
	其中 
	\begin{equation*}
		\lambda_i=-\log\left(\frac{\nu^{i,i+1}}{\nu^{i-1,i}+\nu^{i,i+1}+\nu^+}\,\frac{\nu^{i,i+1}+\nu^+}{\nu^{i,i+1}+\nu^{i+1,i+2}+\nu^+}\right).
	\end{equation*}
	通过 $\nu_i$ 的定义,有
	\begin{equation*}
		\nu_1=\nu^{1}+\nu^{12}+\nu^+,\quad \nu_i=\nu^i+\nu^{i-1,i}+\nu^{i,i+1}+\nu^+,\ 2\le i\le N-1,\quad \nu_N=\nu^N+\nu^{N-1,N}+\nu^+.
	\end{equation*} 
然后通过计算,可以得到:	
\begin{equation}\label{I1lack}
	I_1(\nu)=\nu^{12}\log\frac{\nu^{12}}{\nu_1-\nu^{12}}+\nu^+\log\frac{\nu^+}{\nu_1-\nu^1},
\end{equation}
	\begin{align}\label{I2lack}
	I_2(\nu)
	=\sum_{i=2}^{N}\left[\nu^{i,i+1}\log\frac{\nu^{i,i+1}}{\nu_i-\nu^i}+\nu^+\log\frac{\nu^{i-1,i}+\nu^+}{\nu_i-\nu^i}\right]+\sum_{i=1}^{N}\nu^{i,i+1}\log\frac{\nu^{i,i+1}+\nu^+}{\nu_{i+1}-\nu^{i+1}},
   \end{align}
和
	\begin{equation}\label{I3lack}
		\begin{split}
			I_3(\nu)=\sum_{i\in S}\left[\nu^i\log\frac{\nu^i}{\nu_i}+\nu^+\log\frac{\nu_i-\nu^i}{\nu_i}\right]+\sum_{i\in S}\nu^{i,i+1}\left(\log\frac{\nu_i-\nu^i}{\nu_i}
			+\log\frac{\nu_{i+1}-\nu^{i+1}}{\nu_{i+1}}\right).
		\end{split}
	\end{equation}
    根据 \eqref{pij},可得
	\begin{equation}\label{nucgammac}
		\begin{split}
			&\;\sum_{i \in S}\left[\nu^i\log\frac{J^i}{J_i}+\nu^{i,i+1}\log\frac{J^{i,i+1}}{J_i}+(\nu^{i-1,i}+\nu^+)\log\frac{J^{i-1,i}+J^+}{J_i}\right]\\
			=&\;\sum_{i \in S}\left[\nu^i\log\frac{J^i}{J_i}+\nu^{i,i+1}\log\frac{J^{i,i+1}(J^{i,i+1}+J^+)}{J_{i+1}J_i}\right]+\nu^+\log\frac{\prod_{i=1}^N\left(J^{i,i+1}+J^+\right)}{\prod_{i=1}^N J_i}\\
			=&\;\sum_{c \in \mathcal{C}}\nu^c \log\gamma^c=-I_4(\nu).
		\end{split}
	\end{equation}
	结合 \eqref{I1lack},\eqref{I2lack},\eqref{I3lack},和 \eqref{nucgammac},证毕。
\end{proof}
%%%%%%%%%%%%%%%%%%%%%%%%%%%%%%%%%%%%%%%%%%%%%%%%%%%%%%%%%%%%%%%%%%%%%%%%%

对于三状态马氏链,则速率函数可以化简为:(证明细节见 \ref{appendix:threestate})
\begin{align*}
    I_J(\nu) =
    \sum_{i\in S} \left[\nu^{i}\log \left(\frac{\nu^{i}/\nu_i}{J^i/J_i}\right) + (\nu_i - \nu^i)\log \left(\frac{(\nu_i - \nu^i)/\nu_i}{(J_i - J^i)/J_i} \right)
    \right]
    + \sum_{c\in\mathcal{C},|c|\neq 1} \nu^{c} \log \left(\frac{\nu^{c}/\tilde{\nu}}{J^c/\tilde{J}}\right) ,
\end{align*}
其中
\begin{align*}
    \tilde{\nu} &=\sum_{c\in\mathcal{C},|c|\neq 1}\nu^{c}
    = \nu^{12}+\nu^{13}+\nu^{23}+\nu^++\nu^-,\\
    \tilde{J} &=\sum_{c\in\mathcal{C},|c|\neq 1}J^{c}
    = J^{12}+J^{13}+J^{23}+J^++J^-.
\end{align*}
对于N状态单环马氏链,且$p_{1N}=0$,速率函数可以化简为(证明细节见 \ref{appendix:threestate}):
\begin{equation}\label{lack}
    \begin{split}
        I_J(\nu) = \sum_{i\in S}\Bigg[&\;\nu^i\log\left(\frac{\nu^i/\nu_i}{J^i/J_i}\right)
        +\nu^{i,i+1}\log\left(\frac{\nu^{i,i+1}/\nu_i}{J^{i,i+1}/J_i}\right)\\
        &\;+\left(\nu^{i-1,i}+\nu^+\right)\log\left(\frac{\left(\nu^{i-1,i}+\nu^+\right)/\nu_i}
        {\left(J^{i-1,i}+J^+\right)/J_i}\right)\Bigg].
    \end{split}
\end{equation}
注意到上述两种情况下的速率函数表达式更简单更对称 \eqref{ratefunction}。易知,上述表达式有着完美的对称性,因此与初始状态的选择无关。这也验证了定理\ref{thm:LDP}的相关结论。

LE经验环流$(J^{c}_n)_{c\in\mathcal{C}}$的大偏差原理结果可以直接应用到经验LE净环流$(\tilde{J}^{c}_n)_{c\in\mathcal{C}}$。因为对一状态和两状态马氏链$\tilde{J}^c_n = 0$且$\tilde{J}^+_n = -\ {J}^-_n$,所以只需要考虑环$(1, 2, \cdots ,N)$的经验净环流$\tilde{J}^+_n$。那么由收缩原理可以得到:
\begin{equation}\label{tilde I J}
	\begin{split}
		\mathbb{P}\left(\tilde{J}^{+}_n = x\right)
		&=\;\mathbb{P}\left(J^{+}_n-J^{-}_n = x\right)\\
		&=\;\sum_{\nu^{+}-\nu^{-}=x}\mathbb{P}\left(J^{c}_n=\nu^{c},\forall c\in\mathcal{C}\right)\\
		&\propto\sum_{\nu^{+}-\nu^{-}=x} e^{-nI_J(\nu)}.
	\end{split}
\end{equation}
由此说明$\tilde{J}^+_n$满足大偏差原理, 相应的速率函数$I_{\tilde{J}}$为:
\begin{equation*}
	I_{\tilde{J}}(x)=\inf_{\{\nu\in\mathcal{V}:\nu^{+}-\nu^{-}= x\}}I_J(\nu).
\end{equation*} 



\subsection{一般马氏链的ST环流的大偏差}

经验ST净环流的大偏差,以及速率函数的对称性已经在文献 \cite{bertini2015flows} 中有过相关介绍。接下来将着重研究经验ST环流的大偏差原理,涉及的对经验测度内容可参考 \cite{den2000large}。

$n$时刻的对经验测度$R_n:E\rightarrow[0,1]$,可以定义为
\begin{equation*}
R_n(i,j) = \frac{1}{n}\sum_{m=1}^n1_{\{\xi_{m-1}=i,\xi_m=j\}},
\end{equation*}
显然$R_n(i,j)$表示边$\langle i,j\rangle$形成的速度。注意到在周期边界条件下,对经验测度$R_n$处于空间
\begin{equation*}
\mathcal{M} = \left\{R:E\rightarrow[0,1]:\;\sum_{i,j\in S}R(i,j) = 1,\;
\sum_{j\in S}R(i,j)=\sum_{j\in S}R(j,i)\right\}.
\end{equation*}
众所周知,对经验测度满足下面的大偏差原理:
\begin{equation*}
\mathbb{P}(R_n(i,j)=R(i,j),\;\forall\langle i,j\rangle\in E)\propto e^{-nI_{\mathrm{pair}}(R)},\;\;\;n\to\infty,
\end{equation*}
上式中的速率函数$I_{\mathrm{pair}}:\mathcal{M}\rightarrow[0,\infty]$的表达式为
\begin{equation*}
I_{\mathrm{pair}}(R) = \sum_{\langle i,j\rangle\in E}R(i,j)\log\frac{R(i,j)}{R(i)p_{ij}}
\end{equation*}
其中$R(i)=\sum_{j\in S}R(i,j)$,可以看到,对经验测度的速率函数是相对熵形式。给定生成树$T$,定义在空间$E$上的函数$H^{c_l}$为:
\begin{equation*}\label{cycle function2}
H^{c_l}(i,j)
    =\left\{\begin{aligned}
    1, &   && \text{if } \langle i,j\rangle \in c_l \text{ and }\langle i,j\rangle \in T, \text{ or } \langle i,j\rangle=l,\\
    -1,&   && \text{if } \langle i,j\rangle\notin c_l,\langle j,i\rangle \in c_l,\text{ and }\langle i,j\rangle \in T,\\
    0, &   && \text{otherwise}.\\
    \end{aligned}\right.
\end{equation*}
对经验测度可以被分解为下面ST环流的加权和:
\begin{equation*}
R_n(i,j) = \sum_{c_l\in\mathcal{L}}H^{c_l}(i,j)Q^{c_l}_n,
\end{equation*}
且由文献 \cite{kalpazidou2007cycle}可知,该分解是唯一的。也就是说,如果$R_n =\sum_{c_l \in \mathcal{L}\nu^{c_l}H^{c_l}}$,那么对任意$c_l \in \mathcal{L}$会有$\nu^{c_l}=Q_n^{c_l}$。由上述表达式的唯一性可以得到:
\begin{equation*}
    \mathbb{P}(Q_n^{c_l}=\mu^{c_l},\;\forall c_l\in\mathcal{L})
    =\mathbb{P}\left(R_n(i,j)=\sum_{c_l\in\mathcal{L}}\mu^{c_l}H^{c_l}(i,j),\;\forall\langle i,j\rangle\in E\right)
    \propto e^{-n I_{\mathrm{pair}}\left(\sum_{c_l\in\mathcal{L}}\mu^{c_l}H^{c_l}\right)}.
\end{equation*}
这表明ST经验环流$(Q_n^{c_l})_{c_l\in\mathcal{L}}$满足大偏差原理,相应的速率函数$I_Q:\mathcal{M}\rightarrow[0,\infty]$为:
\begin{equation}\label{formula:I_Q}
I_Q(\mu)=I_{\mathrm{pair}}\left(\sum_{c_l\in\mathcal{L}}\mu^{c_l}H^{c_l}\right).
\end{equation}

目前,已经得到了单环马氏链经验环流的速率函数,和一般马氏链的经验ST环流的速率函数。自然会想到两种速率函数的关系。前面也说过 ST 环流可以通过 LE 表示为$Q_n^{c_l} = \sum_{c\in\mathcal{C}}J^c_n1_{\{l\in c\}}$。从收缩原理中,可以得到:
\begin{align*}
	\mathbb{P}\left(Q_n^{c_l}=\mu^{c_l},\;\forall l\in\mathcal{L}\right)
	&= \mathbb{P}\left(\sum_{c\in\mathcal{C}}J^c_n1_{\{l\in c\}}=\mu^{c_l},\;\forall l\in\mathcal{L}\right)\\
	&= \sum_{\{\nu\in\mathcal{V}:\;\sum_c\nu^c1_{\{l\in c\}}=\mu^{c_l}\}}
	\mathbb{P}\left(J^c_n=\nu^c,\;\forall c\in\mathcal{C}\right)\\
	&\propto \sum_{\{\nu\in\mathcal{V}:\;\sum_c\nu^c1_{\{l\in c\}}=\mu^{c_l}\}}e^{-nI_J(\nu)}.
\end{align*}
这表明 LE 和 ST 经验环流的内部关系为:
\begin{equation*}
	I_Q(\mu) = \inf_{\{\nu\in\mathcal{V}:\;\sum_c\nu^c1_{\{l\in c\}}=\mu^{c_l}\}}I_J(\nu).
\end{equation*}
易知上式中的$I_Q$和式 \eqref{formula:I_Q} 中的一致。

前面论述已得出单环马氏链 LE 经验环流的速率函数,和一般马氏链的 ST 经验环流的速率函数,下面将探讨两者之间的关系。前面也说过 ST 环流可以通过 LE 表示为$Q_n^{c_l} = \sum_{c\in\mathcal{C}}J^c_n1_{\{l\in c\}}$。从收缩原理中,可以得到:
\begin{align*}
	\mathbb{P}\left(Q_n^{c_l}=\mu^{c_l},\;\forall l\in\mathcal{L}\right)
	&= \mathbb{P}\left(\sum_{c\in\mathcal{C}}J^c_n1_{\{l\in c\}}=\mu^{c_l},\;\forall l\in\mathcal{L}\right)\\
	&= \sum_{\{\nu\in\mathcal{V}:\;\sum_c\nu^c1_{\{l\in c\}}=\mu^{c_l}\}}
	\mathbb{P}\left(J^c_n=\nu^c,\;\forall c\in\mathcal{C}\right)\\
	&\propto \sum_{\{\nu\in\mathcal{V}:\;\sum_c\nu^c1_{\{l\in c\}}=\mu^{c_l}\}}e^{-nI_J(\nu)}.
\end{align*}
这表明 LE 和 ST 经验环流的内在关系为:
\begin{equation*}
	I_Q(\mu) = \inf_{\{\nu\in\mathcal{V}:\;\sum_c\nu^c1_{\{l\in c\}}=\mu^{c_l}\}}I_J(\nu).
\end{equation*}
易知上式中的$I_Q$和式 \eqref{formula:I_Q} 中的一致,故该式的证明在此省略。

% 由于 LE 环流的定义相比于 ST 环流更为精确,所以多数情况下,LE 经验环流的速率函数有别于 ST 环流。然而,对于 图 \ref{figure:transitiongraph}(d) 所示的单环系统,基本集$\mathcal{L}$与环空间$\mathcal{C}$完全一致。那么结合 \eqref{same cycle current} 式,可得 LE 经验环流$(J^{c_l})_{c_l\in\mathcal{L}}$和 ST 经验环流$(Q^{c_l})_{c_l\in\mathcal{L}}$恰好相等,并且 
% \begin{equation*}
% 	I_J(\nu) = I_Q(\nu),
% \end{equation*}
% 其中$I_J(\nu)$已经在 \eqref{lack} 中给出。

收缩原理也可以建立ST经验净环流$(\tilde{Q}^{c_l}_n)_{c_l\in\mathcal{L}}$的大偏差原理。对于一状态和两状态环,ST经验净环流为0,因此只需考虑三状态或者更多状态的环。令$\{c_{l_1}$,$c_{l_2}$,$\cdots$,$c_{l_s}\} \subset \mathcal{L}$ 表示基本集中所有三状态及更多状态的环,那么环$c$及相应的反环$c^-$只出现一次。再令$l_1$,$l_1-$,$l_2$,$l_2-$,$\cdots$,$l_s$,$l_s-$为基本集中环对应的弦。那么ST经验净环流$(\tilde{Q}^{c_{l_i}})_{1\le i\le s}$满足大偏差原理,且相应的速率函数$I_{\tilde{Q}}$为:
\begin{equation}\label{I_Q2}
	I_{\tilde{Q}}(x)=\inf_{\{\mu\in\mathcal{M}:\mu^{c_{l_{i}}}-\mu^{c_{l_{i}}-}= x_i,\forall 1\le i\le s\}}I_Q(\mu).
\end{equation}