% !Mode:: "TeX:UTF-8"

\BiChapter{总结与展望}{Conclusions and Perspectives}

本文基于深度学习方法,从初始化的角度出发,介绍了一种新的多尺度网络结构。我们利用尺度变换的技术和紧支集的激活函数,构造了一种可以求解高频震荡椭圆方程的多尺度网络。本文的主要创新在于:在通常的全连接网络下,求解高频震荡的椭圆方程比较困难,而多尺度网络克服了这一缺点。大量的实验证明,多尺度网络是一种无网格、易于实现的椭圆方程求解方法。

相比于传统的有限元和有限差分方法,深度学习方法具有如下优势:首先,在复杂的区域上,由于需要生成网格和求解大型线性方程组,传统的有限元和有限差分法求解方法可能代价高昂。而深度学习方法只需要在边界和内部生成随机点,无需生成网格。此外,深度学习方法是本质非线性的,虽然本文中只讨论了线性方程的情形,这些方法都可以毫无难度地推广到非线性方程的情况。最后,深度学习方法采用了类似蒙特卡洛的抽样,因此可以克服维度灾难,更加快速地求解高维问题。

本文的末尾通过深度学习方法求解了椭圆特征值问题,指出深度学习方法在求解特征值问题中还有很大的改进空间。未来我们可以针对特征值问题设计对应的神经网络,希望可以提高求解效率。在进一步的工作中,我们还将探索在多尺度网络中,充分利用小波激活函数的思想,进一步提高多尺度网络的频率局部化和分离能力。此外,我们将尝试将多尺度网络应用于大规模计算工程问题,特别是与有限元法和有限差分法的比较。更进一步,我们将研究将多尺度网络应用于高维,更复杂的方程,如多体量子系统的薛定谔方程。在这些问题中,方程往往是非常高维的,用深度学习方法求解会为我们带来很大的便利。
